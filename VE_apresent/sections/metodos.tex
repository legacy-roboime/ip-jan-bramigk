\section{Métodos}
\frame{
	\frametitle{Métodos}
	\begin{block}{}
		Falar aqui o porque de estudar as eurísticas
		e algorítimos que estudamos
	\end{block}
}

\frame{
	\frametitle{Lógica Nebulosa}
	\begin{block}{}
		Regras relacionam os conjuntos $A_j$ e $B_j$, gerando o caminho difuso $A_j x B_j$. Na prática,
		é utilizado o produto para definir $ a_j x b_j (x,y) = a_j(x).b_j(y)$. Esta é a parte "padrão" no SAM.
		A parte "aditiva" se refere ao fato de a entrada $x$ acionar a $j$-ésima regra em um grau $a_j(x)$ e o sistema 
		soma os acionamentos ou partes escaladas dos conjuntos escalados $a_j(x)B_j$, \cite{kosko1997fuzzy}:
		
		\begin{eqnarray}
		F(x) = \frac{\sum w_i.a_i(x).V_i.c_i}{\sum w_j.a_j(x).V_j}
		\end{eqnarray}
		
		Com o volume/área $V_j$ e o centroide $c_j$ são dados por:
		
		\begin{eqnarray}
		V_j = \int{b_j(y_1,...,y_p)}_{\Re^{p}}.dy_1...dy_p > 0\\
		c_j = \frac{\int{y.b_j(y_1,...,y_p)}_{\Re^{p}}.dy_1...dy_p}{V_j}
		\end{eqnarray}
	\end{block}
}

\frame{
	\frametitle{Otimização da Colonia de Formigas}
	\begin{block}{}
		\subsection{Pseudo código da meta-heurística do ACO}
		%Algoritmo
		\begin{algorithm}[H]
		
		%Macros
		\SetKwBlock{AgendarAtividade}{AgendarAtividade}{fim}
		\SetKwBlock{Procedimento}{Procedimento}{fim}
		
		\Procedimento{
  			\Enqto{$n < N_{MAX\_IT}$}{
    				%\tcp*[f]{$N_{MAX\_IT}$ é o número máximo de iterações\\}
    				\AgendarAtividade{
      					ConstruirSolucoesFormigas\\
      					AtualizarFeromonios\\
      					%\tcp*[f]{opcional}\\
      					\tcp{opcional:}
      					AcoesGlobais
    				}
  			}
		}

		\caption{Pseudo código da meta-heurística do ACO\label{lst:meta-heuristica_aco}}
		\end{algorithm}
	\end{block}
}

\frame{
	\frametitle{Recozimento Simulado}
	\begin{block}{}
%Algoritmo
\begin{algorithm}[H]
%Macros
\SetKwBlock{Procedimento}{Procedimento}{fim}
\SetKwBlock{EscolherVizinho}{EscolherVizinho}{fim}
\SetKwBlock{CalcTransicao}{CalcTransicao}{fim}

\Procedimento{
  SetarValoresInicias\;
  \Para{$n = 1$ até $N_{MAX\_IT}$ ou $J(x^*) \le TOL$ }{
    \Para{$k = 1$ até $N_{MAX\_IT}$ ou a solução convergir}{
      \EscolherVizinho{
        selecionar algum $j \in S(i)$\;
      }

      \CalcTransicao{
        $\Delta J \leftarrow J(j)-J(i)$\;
        \Se{$Delta J \le 0$}{
          $x(t+1) \leftarrow j$\;
          $x^* \leftarrow j$\;
        }
        \Senao{
          %$q_{ij} \leftarrow exp^{\left\{-\frac{\Delta J}{T(t)} \rigth\}}$\;\\
          $q_{ij} \leftarrow exp^{ -\frac{\Delta J}{T(t)} } $\;
          \lSe{$random() < q_{ij}$}{$x(t+1) \leftarrow j$}
          \lSenao{$x(t+1) \leftarrow i$}
        }
      }
    }
    AtualizarTemperetura\;
  }
}

\caption{Pseudo código da meta-heurística do SA\label{lst:meta-heuristica_sa}}
\end{algorithm}
	\end{block}
}

\frame{
	\frametitle{Algoritmo Genético}
	\begin{block}{}
	\end{block}
}

\frame{
	\frametitle{Redes Neurais}
	\begin{block}{}
	\end{block}
}

