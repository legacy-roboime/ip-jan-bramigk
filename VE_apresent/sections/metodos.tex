\section{Métodos}
\frame{
	\frametitle{Métodos}
	\begin{block}{}
		\begin{center}
			Como modelar a função $F$?
		\end{center}
	\end{block}
}

\frame{
	\frametitle{Lógica Nebulosa (SAM)}
	\begin{block}{}
		$$
			F(x) = \frac{\sum w_i.a_i(x).V_i.c_i}{\sum w_j.a_j(x).V_j}
		$$
		
		Com o volume/área $V_j$ e o centroide $c_j$ são dados por:
		
		$$
			V_j = \int{b_j(y_1,...,y_p)}_{\Re^{p}}.dy_1...dy_p > 0
		$$

		$$
			c_j = \frac{\int{y.b_j(y_1,...,y_p)}_{\Re^{p}}.dy_1...dy_p}{V_j}
		$$	
	\end{block}
}

\frame{
	\frametitle{Otimização da Colonia de Formigas}
	\begin{block}{}
		\subsection{Pseudo código da meta-heurística do ACO}
		%Algoritmo
		\begin{algorithm}[H]
		
		%Macros
		\SetKwBlock{AgendarAtividade}{AgendarAtividade}{fim}
		\SetKwBlock{Procedimento}{Procedimento}{fim}
		
		\Procedimento{
  			\Enqto{$n < N_{MAX\_IT}$}{
    				%\tcp*[f]{$N_{MAX\_IT}$ é o número máximo de iterações\\}
    				\AgendarAtividade{
      					ConstruirSolucoesFormigas\\
      					AtualizarFeromonios\\
      					%\tcp*[f]{opcional}\\
      					\tcp{opcional:}
      					AcoesGlobais
    				}
  			}
		}

		\caption{Pseudo código da meta-heurística do ACO\label{lst:meta-heuristica_aco}}
		\end{algorithm}
	\end{block}
}

\frame{
	\frametitle{Recozimento Simulado}
	\begin{block}{}
		%Algoritmo
		\begin{algorithm}[H]
			%Macros
			\SetKwBlock{Procedimento}{Procedimento}{fim}
			
			\Procedimento{
  			SetarValoresInicias\;
      			EscolherVizinho\;
			
      			CalcTransicao\;
    			Atualizar Temperatura\;
			}
		
			\caption{Pseudo código da meta-heurística do SA\label{lst:meta-heuristica_sa}}
		\end{algorithm}
	\end{block}
}

\frame{
	\frametitle{Algoritmo Genético}
	\begin{block}{}
\subsection{Pseudo código de um Algorítimo Genético}

%\begin{lstlisting}
\begin{algorithm}[H]
\SetKwBlock{Procedimento}{Procedimento}{fim}

%Algorithm: GA(n, \ki, \mu)
\Procedimento{
  %// Initialise generation 0:
  $k \leftarrow 0$, $P_k \leftarrow $ população aleatoriamente\;
  %// EvaluatePk:
  %\Para{cada $i$ em $P_k$}
  %Compute fitness(i) for each i ∈ Pk;
  %Computar a $avaliacao(i)$ para cada $i$ em $P_k$\;
  %while fitness of fittest individual in Pk is not high enough;
  \Enqto{a $avaliacao(i)$ de cada $i$ em $P_k$ não for boa o suficiente}{
    %// Create generation k + 1:
    %// 1. Copy:
    %Select (1−χ)×n members ofPk and insert into Pk+1;
    Selecionar os $(1 - \chi) \times n$ membros com maior $avaliacao(i)$ de $P_k$ e inserir em $P_{k+1}$\;
    %// 2. Crossover:
    %Select χ×n members of Pk; pair them up; produce offspring; insert the offspring into Pk+1;
    Selecionar $\chi \times n$ membros de $P_k$, pareá-los e inserir a cria em $P_{k+1}$\;
    %// 3. Mutate:
    %Select µ×n members of Pk+1; invert a randomly-selected bit in each;
    Selecionar os $\mu \times n$ membros de $P_{k+1}$ com maior $avaliacao(i)$ e inverter um bit aleatório de cada membro\;
    %// Evaluate Pk+1:
    %Compute fitness(i) for each i ∈ Pk;
    %Computar a $avaliacao(i)$ para cada $i$ em $P_{k+1}$\;
    %// Increment:
    %k := k + 1;
    $k \leftarrow k + 1$\;
  }
%return the fittest individual from Pk;ut your code here.
  \Retorna{membro $i$ em $P_k$ com maior $avaliacao(i)$}
}
\end{algorithm}
	\end{block}
}

\frame{
	\frametitle{Redes Neurais}
	\begin{block}{}
	\end{block}
}

