% TODO: Descrever mais os capítulos

% textual -> contextualizar
% temática -> idéia principal
% interpretativa

\chapter{Revisão da Literatura}\label{cap:rev_bibliografica}


\citet{russellnorvig} apresenta os conceitos básicos dos agentes em IA,
inclusive seus diversos tipos. Em \cite{haykin2001redes}, é referencia fundamental
em ANN. \cite{kosko1997fuzzy} apresenta os conceitos básicos que deram início
a lógica nebulosa, inclusive seus fundamentos. Os métodos desenvolvidos
atualmente baseados em lógica nebulosa são também apresentados, inclusive o
modelo aditivo padrão, que foi utilizado neste.

Em \cite{passos2005datamining} são apresentados os conceitos de mineração de
dados. Entretanto, somente são apresentados os conceitos básicos. Para esta
pesquisa, foi necessário recorrer a outras referências que tratassem de
mineração de dados em sequência.

\cite{doringo2004ant} é a referência base para fundamentação teórica e para
aplicação do ACO, contém uma ilustração de aplicação deste método a problemas
de classificação. Entretanto, novamente foi necessário um aprofundamento maior
para estudar a aplicação do ACO para dados em sequência.


\cite{bertsimas1993simulated} apresenta o SA.

% Em\cite{yoneyama2004ia} é apresentado uma abordagem de problemas de controle,
% utilizando de IA\@. Destaca-se a abordagem utilizando algoritmos de otimização.

Em\cite{felixnavarro} é apresentada arquitetura para o ambiente da SSL/F180.
Essa arquitetura foi base para a arquitetura descrito no
capítulo~\ref{cap:def_problema}.

Em\cite{zickler} é apresentado uma modelagem de um planejador robótico baseado
em física para ambientes dinâmicos. Essa modelagem foi base para a enunciação
do problema descrita no capítulo~\ref{cap:def_problema}, juntamente com
arquitetura orientada a \textit{Skills, Tactics and Plays} descrita em
\cite{bowling2003plays} e com a teoria dos agentes apresentada em
\cite{russellnorvig}.

Em\cite{vail2008crf} é apresentada uma modelagem do problema considerado neste
trabalho utilizando-se \textit{Conditional Random Fields}. É interessante
ressaltar que essa modelagem foi realizada utilizando-se um time $T_1$
(definição~\ref{def:time}) conhecido no qual os dados de treinamento que foram
coletados da partida $p$ sabendo-se $p.A_1^{i}$, e não somente $X_{ob}^{i}$.

Em\cite{sheng2005motionprediction} é apresentada uma modelagem considerando
apenas $X_{ob}^{i}$ de um time $T$. Essa modelagem utiliza redes neurais de
três camadas. Foi utilizada uma função de ativação sigmoidal para a camada
oculta e linear para a função de ativação da saída. O algoritmo de treinamento
utilizado na rede neural foi o \textit{standard back-propagation algorithm},
descrito em\cite{haykin2001redes}.
