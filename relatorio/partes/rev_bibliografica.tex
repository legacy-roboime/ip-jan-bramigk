\chapter{Revisão Bibliográfica}\label{cap:rev_bibliografica}


Em \cite{russellnorvig}, \cite{haykin2001redes}, \cite{kosko1997fuzzy}, \cite{passos2005datamining},
\cite{doringo2004ant} e \cite{bertsimas1993simulated} são descritas as heurísticas
mais comumente aplicadas a problemas envolvendo Inteligência Artificial (IA).

Em \cite{yoneyama2004ia} é apresentado uma abordagem de problemas de controle, utilizando de IA. Destaca-se
a abordagem utilizando algoritmos de otimização.

Em \cite{felixnavarro} é apresentada arquitetura para o ambiente da SSL/F180. Essa arquitetura foi base
para a arquitetura descrito no capítulo \ref{cap:def_problema}.

Em \cite{zickler} é apresentado uma modelagem de um planejador robótico baseado em física para ambientes dinâmicos. 
Essa modelagem foi base para a enunciação do problema descrita no capítulo \ref{cap:def_problema}, juntamente com 
arquitetura orientada a \textit{Skills, Tactics and Plays} descrita em \cite{bowling2003plays} e com a teoria dos 
agentes apresentada em \cite{russellnorvig}.

Em \cite{vail2008crf} é apresentada uma modelagem do problema considerado neste trabalho
utilizando-se \textit{Conditional Random Fields}. É interesante resaltar que essa modelagem
foi realizada utilizando-se um time $T_1$ (definição \ref{def:time}) conhecido no qual os dados
de treinamento que foram coletados da partida $p$ sabendo-se $p.A_1^{i}$, e não somente $X_{ob}^{i}$.

Em \cite{sheng2005motionprediction} é apresentada uma modelagem considerando apenas $X_{ob}^{i}$
de um time $T$. Essa modelagem utiliza redes neurais de três camadas. Foi utilizada
uma função de ativação sigmoidal para a camada oculta e linear para a função de 
ativação da saída. O algoritmo de treinamento utilizado na rede neural foi o \textit{standard back-propagation
algorithm}, descrito em \cite{haykin2001redes}.
