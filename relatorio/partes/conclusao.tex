\chapter{Conclusão}\label{cap:conclusao}

Este trabalho descreve uma modelagem pesquisada para o jogo de futebol de robôs
da categoria SSL da competição RoboCup e introduz métodos que podem contribuir
para a confecção da solução final. Concluiu-se que nem todas as variáveis
desejadas durante  processo de predição são observáveis, tornando o problema
mais complexo do que realmente se imaginava. Os ACO, SA,
lógica \textit{fuzzy} e o AG não são apropriadas para resolver diretamente o
problema. O ACO e SA podem ser empregados utilizando um classificador 
modelado com um CRF, onde atuariam na minimização do vetor de pesos associado
a essa estrutura. A modelagem da solução utilizando lógica \textit{fuzzy} se
mostrou muito complexa para este trabalho. Entretanto, de acordo com a
bibliografia, ela confere mais robustez ao sistema. A rede neural
parece ser uma boa abordagem para o problema, mas é necessário tratar melhor os
dados dos \textit{logs} para prosseguir com uma implementação significativa.
O futuro desse trabalho pode ser desenvolver uma inteligencia artificial baseada
no \textit{MiniMax} que use a heurística pesquisada para simular o comportamento
do adversário.

% TODO[vbramigk] Refazer conclusão:
% - Recontextualizar
% - Listar objetivos (com linha de onde foram cumpridos)
% - Retificando o que forem concluindo
% - Apresentar as contribuições
% - Trabalhos futuros

% vim: tw=80
