\chapter{Conclusão}\label{cap:conclusao}

Este trabalho visa modelar o time adversário do jogo de futebol de robôs
da categoria SSL da competição RoboCup e introduz métodos que podem contribuir
para a confecção da solução final.

No capítulo~\ref{cap:heuristicas}, foram estudados os algorítmos ACO, SA, GA,
ANN assim como métodos da lógica fuzzy. Para cada um desses algorítimos, foi
apresentado
um exemplo de aplicação para se obter um melhor domínio do respectivo
algorítimo.

Com o objetivo de compreender melhor o problema e verificar se a IA de um time
de
futebol de robôs pode ser modelada utilizando esses algorítmos, no
capítulo~\ref{cap:def_problema} foi desenvolvido o enunciado formal do problema
a ser resolvido.
Apesar de não ser o objetivo inicial, foi descoberto que o jogo da SSL não é
completamente observável. Esse fato é de suma importância para o desenvolvimento
de estratégias chava para este jogo e, como consequência, para se modelar a
inteligência inimiga partindo-se dos dados coletados durante o jogo.

No capítulo~\ref{cap:anal_abordagens}, as heurísticas estudadas foram analisadas
criticamente, com o objetivo de determinar se são adequadas para modelar a IA
do time adversário com base nas informações contidas nos logs de um jogo da SSL
da RoboCup. Conclui-se que o método da lógica fuzzy estudado não é adequado para
abordar o problema devido a complexidade de incorporar o conceito de
temporalidade.
Verificou-se também que métodos ACO, SA, GA servem primariamente para apoio de
um
gorítimo baseado em ANN\@.

Como a ANN se mostrou ser o método mais adequado dentre os métodos estudados,
foi desenvolvido um teste de conceito na secção~\ref{cap:abordagem_rede_neural}.
Concluiu-se que, apesar de ser necessário um pré-processamento, esse
método é apto para resolver o problema utilizando os \textit{logs}.

Como trabalho futuro, é sugerido que seja implementado preditor baseado em
algoritmo com ANN para modelar o time adversário em uma IA que utilize o
algoritmo minimax para efetuar a movimentação do time advesário.



 %Este trabalho descreve uma modelagem pesquisada para o jogo de futebol de robôs
 %da categoria SSL da competição RoboCup e introduz métodos que podem contribuir
 %para a confecção da solução final. Concluiu-se que nem todas as variáveis
 %desejadas durante  processo de predição são observáveis, tornando o problema
 %mais complexo do que realmente se imaginava. Os ACO, SA,
 %lógica \textit{fuzzy} e o AG não são apropriadas para resolver diretamente o
 %problema. O ACO e SA podem ser empregados utilizando um classificador 
 %modelado com um CRF, onde atuariam na minimização do vetor de pesos associado
 %a essa estrutura. A modelagem da solução utilizando lógica \textit{fuzzy} se
 %mostrou muito complexa para este trabalho. Entretanto, de acordo com a
 %bibliografia, ela confere mais robustez ao sistema. A rede neural
 %parece ser uma boa abordagem para o problema, mas é necessário tratar melhor os
 %dados dos \textit{logs} para prosseguir com uma implementação significativa.
 %O futuro desse trabalho pode ser desenvolver uma inteligencia artificial baseada
 %no \textit{MiniMax} que use a heurística pesquisada para simular o comportamento
 %do adversário.

% TODO[vbramigk] Refazer conclusão:
% - Recontextualizar
% - Listar objetivos (com linha de onde foram cumpridos)
% - Retificando o que forem concluindo
% - Apresentar as contribuições
% - Trabalhos futuros

% vim: tw=80
