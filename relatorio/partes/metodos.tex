\chapter{Heurísticas para Solução do Problema}\label{cap:heuristicas}

\section{Introdução}

Conforme descrito no capítulo anterior, o problema a ser resolvido será
encarado como um problema de classificação temporal. O sistema deve mapear uma
sequência temporal de observações para um conjunto de regras associadas a
subconjuntos de robôs. Essas regras definem o comportamento dos robôs.

Outro problema relevante é definir o comportamento dos robôs a partir do estado
atual do jogo e das regras associadas aos robôs. Entretanto, devido ao tempo
destinado a esta pesquisa será modelado somente uma solução para o mapeamento das
observações às regras. Também, o módulo de inteligência necessita dos dados
desse mapeamento para poder prever de maneira coerente estados futuros do jogo. Uma
solução ingênua é utilizar os modelos de táticas e \textit{skills} do próprio time.
Isso não representa a realidade, pois é pouco provável que dois times desenvolvam
todos os algoritmos que controlam os robôs de tal maneira que eles tenham o mesmo
resultado.

\subsection{Regras}

Em princípio cada equipe utiliza seu próprio conjunto de regras.
Na prática, entretanto, devido aos trabalhos publicados na comunidade científica
relacionados ao problema de futebol de robôs, muitas regras são compartilhadas
entre os times. Outra razão plausível é que muitas heurísticas e regras mimicam
as regras dos times de futebol jogados por humanos. Exemplos de regras típicas são
apresentados na tabela~\ref{regras}.

\begin{table}
  \begin{center}
    \begin{tabular}{|c|c|}
      \hline
      Regra                  & Descrição \\
no campo, várias funções características podem ser instanciadas.
Tipicamente chega-se à ordem de 100 mil instâncias. Faz-se necessário
selecionar quais funções efetivamente são relevantes para o problema.

\subsection{Seleção de Características}

Apesar de incorporar informação ao modelo, a inclusão de muitas características
pode fazer com que o modelo incorpore informações particulares dos dados de
treinamento. Isso é chamado de \textit{overfitting}. Esse fenômeno reduz a precisão
do modelo final por incorporar dados particulares do conjunto de dados de
treinamento. Devido a isso é necessário selecionar quais características devem
ser incorporadas ao modelo final.

A seleção de características reduz o tamanho do modelo, bem como o custo
computacional da classificação. Isso permite que as regras sejam reconhecidas
em tempo viável para serem utilizadas pelo módulo da inteligência durante o
jogo.

\subsection{Modelagens}

Várias estruturas são utilizadas em problemas de classificação. Dentre elas:
\textit{Hidden Markov Models} (HMMs), \textit{Conditional Random Fields} (CRFs),
Redes Neurais Artificiais (RNAs), \textit{Support Vector Machines}.

As modelagens estudadas nesse trabalho envolvem as RNAs e CRFs.
