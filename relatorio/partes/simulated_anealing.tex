\section{Recozimento Simulado}

No processo de recozimento de um metal, a quantidade de energia interna livre esta intrinsecamente
relacionada ao processo de resfriamento em que o metal é submetido. Quanto mais rápido se 
resfriam um metal mais energia é armazenada internamente. Isso pode ser explicado considerando que
o tempo que a estrutura leva para atingir o estado de menor energia é maior que o disponível devido
a redução da mobilidade dos átomos com o decaimento da temperatura. Com efeito, quanto maior a taxa de
resfriamento maior o número de defeitos na estrutura do sólido e menor o tamanho médio dos grãos.
Quando se reduz a taxa de resfriamento, há uma maior chance de se atingir configurações mais estáveis.
Como resultado, a energia interna é reduzida. De acordo com \cite{bertsimas1993simulated},
pode-se modelar a probabilidade $p_{ij}$ de uma configuração atômica $\{r_i\}$ com energia $E\{r_i\}$ 
passar para a configuração $\{r_j\}$ com energia $E\{r_j\}$ na temperatura $T$ como:

\begin{equation}
\mbox{$p_{ij}$}=\left\{
	\begin{array}{rl}
	1 & \mbox{se $E\{r_j\} \le E\{r_i\}$} \\
	exp\left\{-\frac{(E\{r_j\}-E\{r_i\})}{k_B.T}\right\} & \mbox{se $E\{r_j\} > E\{r_i\}$}
\end{array} \right.
\end{equation}

Onde $k_B$ é a constante de Boltzmann. Para se reduzir a energia livre, é necessário que uma
rotina de resfriamento seja escolhida de acordo com o tipo de material a ser resfriado.

Conforme proposto por Kirikpartrick, Gellett e Vechin (1983) e Cerny (1985), pode-se desenvolver
uma heurística probabilística para se encontrar o mínimo global de uma função custo que possua
vários mínimos locais fazendo-se uma analogia com o fenômeno físico descrito acima. A meta-heurística
induzida por este processo é chamada de meta-heurística \textit{Simulated Annealing}(Recozimento 
Simulado), ou SA, apresentado a seguir.

\subsection{Meta-heurística do SA}

De acordo com \cite{bertsimas1993simulated}, os elementos básicos da meta-heurística do SA 
para a resolução de um problema combinatório são:

\begin{enumerate}
 \item Um conjunto finito $S$.
 \item Um função custo $J$ de imagem real definida em $S$. Seja $S^* \subset S$ o conjunto de todos os mínimos globais da
 função $J$, suposto subconjunto próprio.
 \item Para cada $i \in S$ um conjunto $S(i) \subset S - \{i\}$, chamado de conjunto das vizinhos de $i$.
 \item Para cada $i$, uma coleção de coeficientes positivos $q_{ij}$, $j \in S(i)$, tal que $\sum_{j \in S(i)} q{ij} = 1$.
 \item Uma função não crescente $T: \textbf{N} \rightarrow (0,\infty)$, chamada de rotina de resfriamento. Aqui \textbf{N}
 representa o conjunto de inteiros positivos, e $T(t)$ é chamada de \textit{temperatura} no tempo $t$.
 \item Um estado inicial $x(0) \in S$.
\end{enumerate}

Com base na definições acima, tem-se o seguinte pseudo código para a meta-heurística do SA:

%Algoritmo
\begin{algorithm}[H]
%Macros
\SetKwBlock{Procedimento}{Procedimento}{fim}
\SetKwBlock{EscolherVizinho}{EscolherVizinho}{fim}
\SetKwBlock{CalcTransicao}{CalcTransicao}{fim}

\Procedimento{
  SetarValoresInicias\;
  \Para{$n = 1$ até $N_{MAX\_IT}$ ou $J(x^*) \le TOL$ }{
    \Para{$k = 1$ até $N_{MAX\_IT}$ ou a solução convergir}{
      \EscolherVizinho{
        selecionar algum $j \in S(i)$\;
      }

      \CalcTransicao{
        $\Delta J \leftarrow J(j)-J(i)$\;
        \Se{$Delta J \le 0$}{
          $x(t+1) \leftarrow j$\;
          $x^* \leftarrow j$\;
        }
        \Senao{
          %$q_{ij} \leftarrow exp^{\left\{-\frac{\Delta J}{T(t)} \rigth\}}$\;\\
          $q_{ij} \leftarrow exp^{ -\frac{\Delta J}{T(t)} } $\;
          \lSe{$random() < q_{ij}$}{$x(t+1) \leftarrow j$}
          \lSenao{$x(t+1) \leftarrow i$}
        }
      }
    }
    AtualizarTemperetura\;
  }
}

\caption{Pseudo código da meta-heurística do SA\label{lst:meta-heuristica_sa}}
\end{algorithm}

No algorítimo~\ref{lst:meta-heuristica_sa}, o procedimento \textit{AtualizarTemperetura} executa a
rotina de resfriamento através da função $T(t)$ definida anteriormente. Já o procedimento
\textit{EscolherVisinho} escolhe aleatoriamente um dos elementos da vizinhança do vértice atual $i$.
