\section{Recozimento Simulado}

No processo de recozimento de um metal, a quantidade de energia interna livre
esta intrinsecamente relacionada ao processo de resfriamento em que o metal é
submetido. Quanto mais rápido se resfriam um metal mais energia é armazenada
internamente. Isso pode ser explicado considerando que o tempo que a estrutura
leva para atingir o estado de menor energia é maior que o disponível devido
a redução da mobilidade dos átomos com o decaimento da temperatura. Com efeito,
quanto maior a taxa de resfriamento maior o número de defeitos na estrutura do
sólido e menor o tamanho médio dos grãos. Quando se reduz a taxa de resfriamento, 
há uma maior chance de se atingir configurações mais estáveis.
Como resultado, a energia interna é reduzida. De acordo com
\cite{bertsimas1993simulated}, pode-se modelar a probabilidade $p_{ij}$ de uma
configuração atômica $\{r_i\}$ com energia $E\{r_i\}$ passar para a configuração $\{r_j\}$ com energia $E\{r_j\}$ na temperatura $T$ como:

\begin{equation}
\mbox{$p_{ij}$}=\left\{
	\begin{array}{rl}
	1 & \mbox{se $E\{r_j\} \le E\{r_i\}$} \\
	exp\left\{-\frac{(E\{r_j\}-E\{r_i\})}{k_B.T}\right\} & \mbox{se $E\{r_j\} > E\{r_i\}$}
\end{array} \right.
\end{equation}

Onde $k_B$ é a constante de Boltzmann. Para se reduzir a energia livre, é necessário que uma
rotina de resfriamento seja escolhida de acordo com o tipo de material a ser resfriado.

Conforme proposto por Kirikpartrick, Gellett e Vechin (1983) e Cerny (1985), pode-se desenvolver
uma heurística probabilística para se encontrar o mínimo global de uma função custo que possua
vários mínimos locais fazendo-se uma analogia com o fenômeno físico descrito acima. A meta-heurística
induzida por este processo é chamada de meta-heurística \textit{Simulated Annealing} (Recozimento
Simulado), ou SA, apresentada a seguir.

\subsection{Meta-heurística do SA}

De acordo com \cite{bertsimas1993simulated}, os elementos básicos da meta-heurística do SA
para a resolução de um problema combinatório são:

\begin{enumerate}
 \item Um conjunto finito $S$.
 \item Um função custo $J$ de imagem real definida em $S$. Seja $S^* \subset S$ o conjunto de todos os mínimos globais da
 função $J$, suposto subconjunto próprio.
 \item Para cada $i \in S$ um conjunto $S(i) \subset S - \{i\}$, chamado de conjunto dos vizinhos de $i$.
 \item Para cada $i$, uma coleção de coeficientes positivos $q_{ij}$, $j \in S(i)$, tal que $\sum_{j \in S(i)} q_{ij} = 1$.
 \item Uma função não crescente $T: \textbf{N} \rightarrow (0,\infty)$, chamada de rotina de resfriamento. Aqui \textbf{N}
 representa o conjunto de inteiros positivos, e $T(t)$ é chamada de \textit{temperatura} no tempo $t$.
 \item Um estado inicial $x(0) \in S$.
\end{enumerate}

Com base nas definições acima, tem-se o pseudo código para a meta-heurística do
SA apresentado no algoritmo~\ref{lst:meta-heuristica_sa}.

%Algoritmo
\begin{algorithm}
%Macros
\SetKwBlock{Procedimento}{Procedimento}{fim}
\SetKwBlock{EscolherVizinho}{EscolherVizinho}{fim}
\SetKwBlock{CalcTransicao}{CalcTransicao}{fim}

\Procedimento{
  SetarValoresInicias\;
  \Para{$n = 1$ até $N_{MAX\_IT}$ ou $J(x^*) \le TOL$ }{
    \Para{$k = 1$ até $N_{MAX\_IT}$ ou a solução convergir}{
      \EscolherVizinho{
        selecionar algum $j \in S(i)$\;
      }

      \CalcTransicao{
        $\Delta J \leftarrow J(j)-J(i)$\;
        \Se{$Delta J \le 0$}{
          $x(t+1) \leftarrow j$\;
          $x^* \leftarrow j$\;
        }
        \Senao{
          %$q_{ij} \leftarrow exp^{\left\{-\frac{\Delta J}{T(t)} \rigth\}}$\;\\
          $q_{ij} \leftarrow exp^{ -\frac{\Delta J}{T(t)} } $\;
          \lSe{$random() < q_{ij}$}{$x(t+1) \leftarrow j$}
          \lSenao{$x(t+1) \leftarrow i$}
        }
      }
    }
    AtualizarTemperetura\;
  }
}

\caption{Pseudo código da meta-heurística do SA}
\label{lst:meta-heuristica_sa}
\end{algorithm}

No algoritmo~\ref{lst:meta-heuristica_sa}, o procedimento \textit{AtualizarTemperetura} executa a
rotina de resfriamento através da função $T(t)$ definida anteriormente. Já o procedimento
\textit{EscolherVisinho} escolhe aleatoriamente um dos elementos da vizinhança do vértice atual $i$.

\subsection{Exemplo de Aplicação}

O método SA pode ser aplicado para otimizar a busca em árvores de decisão.
Com efeito, devido a complexidade de determinados jogos, não é possível
avaliar todos os possíveis resultados de cada jogada em tempo hábil. A
tabela~\ref{table:games} apresenta a complexidade de alguns jogos de tabuleiro.
Na tabela, \textbf{espaço-de-estado} denota o número de posições legais
atingíveis a partir da posição inicial e \textbf{árvore-do-jogo} denota o
número total de jogos que podem ser jogados, i.e., o número de folhas na árvore
de jogo cuja raiz é a posição inicial do jogo.


\begin{table}
  \begin{center}
    \begin{tabular}{|c|c|c|}
      \hline
                        &                              &                      \\
 \textbf{Jogo} & log(\textbf{espaço-de-estado}) & log(\textbf{árvore-do-jogo})\\
                        &                              &                      \\
      \hline
        Jogo da velha   &             3               &          5           \\
      \hline
        Trilha          &             10              &          50          \\
      \hline
        Oware           &             16              &          32          \\
      \hline
        Pentaminó       &             12              &          18          \\
      \hline
        Lig 4           &             14              &          21          \\
      \hline
        Damas           &             33              &          50          \\
      \hline
        Lines of Action &             24              &          56          \\
      \hline
        Reversi         &             28              &          58          \\
      \hline
        Gamão           &             20              &          144         \\
      \hline
        Quoridor        &             42              &          162         \\
      \hline
        Xadrez          &             46              &          123         \\
      \hline
        Xadrez Chinês   &             52              &          150         \\
      \hline
        Arimaa          &             42              &          190         \\
      \hline
        Shogi           &             71              &          226         \\
      \hline
        Connect6        &             172             &          140         \\
      \hline
        Go              &             171             &          360         \\
      \hline
    \end{tabular}
    \caption{Complexidades do espaço de estado e da árvore do jogo de alguns
             jogos \cite{mertens2006quoridor}}
  \label{table:games}
  \end{center}
\end{table}


%%%%Applying Genetic Algorithms to
%%%Quoridor Game Search Trees for Next-Move Selection
Como o SA é uma otimização de busca local guloso, assumi-se que o adversário
selecionará a melhor jogada de acordo com uma função de avaliação comum aos dois
jogadores. Então a seleção de nós é feita utilizando-se o SA tradicional.

Isso conceitualmente traduz no jogador SA ter olhado para baixo três níveis de
uma determinada posição na árvore de jogo para avaliar a posição atual. A mesma
função heurística de avaliação linear é usada para a avaliação da própria
posição de jogo. No caso do jogo quoridor essa heurística leva em consideração
tanto o número de paredes restantes para cada jogador quanto a distância de cada
jogador ao lado objetivo.

A função objetivo é responsável por determinar a probabilidade de movimentos
piores serem tomados a partir de uma determinada posição. Uma possível função
objetivo é da forma:

\begin{equation}
  exp\left\{\frac{h(atual)- h(vizinho)}{tempo}\right\}
\end{equation}

em que $h$ é uma função de avaliação heurística que é tanto menor quanto melhor
for o estado do jogo. O parâmetro de controle aleatoriedade contra a busca
gananciosa pura é o tempo, e não um constante. Esse é o aspecto do SA que evita
ficar preso em ótimos locais.

Quando se atinge estágios posteriores na árvore de busca jogo onde o jogador
esta perto de seu estado objetivo, a diferença $h(atual)- h(vizinho)$
será relativamente grande (sendo relativamente pequena no início do jogo). Isto
sugere que o uso de um parâmetro constante no lugar do tempo resultaria em um
jogador que seleciona movimentos ruins mais frequentemente perto do fim.
Como o tempo também seria um valor relativamente grande perto do final do jogo,
isso iria compensar essa grande diferença e levar-nos ao estado objetivo mais
rápido \cite{mcdermid2003gaquoridor}.

