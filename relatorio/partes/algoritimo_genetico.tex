\section{Algorítimo Genético}

Um \emph{algorítimo genético} é uma heurística de busca que procura
imitar a seleção natural que ocorre no processo evolucionário dos
organismos vivos.

Nessa heurística, uma população de soluções (também chamadas de
indivíduos ou fenótipos) para problemas de otimização é evoluída para
conseguir soluções melhores. Cada solução possui um conjunto de
propriedades (cromossomos ou genótipos) que podem ser mutados ou
alterados.

Os requerimentos são, típicamente:

\begin{itemize}
\itemsep1pt\parskip0pt\parsep0pt
\item
  uma representação genética da solução
\item
  uma função de aptidão para avaliação da solução
\end{itemize}

\subsection{O processo}

O processo é iniciado com uma população com propriedades geradas
aleatóreamente.

A iteração da heurística se da em 3 etapas:

\begin{itemize}
\itemsep1pt\parskip0pt\parsep0pt
\item
  procriação: indivíduos são pareados e é aplicada a operação de
  cruzamento (\emph{crossover})
\item
  mutação: alguns indíviduos são selecionados e é aplicada a operação de
  mutação (\emph{mutation})
\item
  seleção: é usada a funçao de aptidão para descartar os indivíduos
  menos aptos restando as soluções que de fato trouxeram alguma melhora.
\end{itemize}

As condições mais comuns para terminação do processo são as seguintes:

\begin{itemize}
\itemsep1pt\parskip0pt\parsep0pt
\item
  encontrada uma solução que atende os requisitos mínimos
\item
  número fixo de gerações alcançado
\item
  recursos alocados (tempo ou dinheiro) alcançados
\item
  a melhor solução alcançou um patamar estável em que mais iterações não
  produzem soluções melhores
\item
  inspeção manual
\end{itemize}

\subsection{Limitações}

As limitações mais comuns no emprego de um algorítimo genético são:

\begin{itemize}
\itemsep1pt\parskip0pt\parsep0pt
\item
  Funções de avaliação computacionalmente caras tornam essa heurística
  ineficiente.
\item
  Não escala bem com a complexidade, isto é, quando o número de
  elementos expostos a mutação é grande o espaço de busca cresce
  exponencialmente. Por isso, na prática algorítimos genéticos são
  usados para, por exemplo, projetar uma hélice e não um motor.
\item
  A melhor solução é relativa às outras soluções, por isso o critério de
  parada não é muito claro em alguns problemas.
\item
  Em muitos problemas os algorítimos genéticos tendem a convergir para
  um ótimo local ou as vezes pontos arbitrários em vez do ótimo global.
\item
  É difícil aplicar algorítimos genéticos para conjunto de dados
  dinâmicos. Pois as soluções podem começar a convergir para um conjunto
  de dados que já não é mais válido.
\item
  Algorítimos genéticos não conseguem resolver eficientemente problemas
  em que a avaliação é binária (certo/errado), como em problemas de
  decisão. Nesse caso buscas aleatórias convergem tão rápido quanto essa
  heurística.
\item
  Para problemas mais específicos existem outras heurísticas que
  encontram a solução mais rapidamente.
\end{itemize}

\subsection{Referências}

\begin{itemize}
\itemsep1pt\parskip0pt\parsep0pt
\item
  \href{http://en.wikipedia.org/wiki/Genetic\_algorithm}{Genetic algorithm}
\end{itemize}
