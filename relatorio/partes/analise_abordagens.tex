\chapter{Análise das Possíveis Abordagens}\label{cap:anal_abordagens}

Uma característica desejada no método a ser empregado para prever próximo
movimento do adversário é não requerer um modelo de classificação. Isso para
deixa a implementação muito complexa, o que dificulta a manutenibilidade do
software final.

Também é desejado que tenha uma estrutura de fácil análise. Essa característica
tem o objetivo de incorporar informação heurística no modelo de modo a facilitar
a modelagem. Apesar de ser desejáda, inicialmente essa característica é menos
importante que a anterior. Entretanto, no longo prazo ela é mais relevante.
Portanto, como o objetivo de se adquirir uma experiencia, essa característica é
menos importante que a anterior.

Um outro problema identificado na análise anterior é a degradação do modelo
devido a discretização. Isso reduz a presição do modelo, então é importante que
o modelo não seja degradado com a discretização.

O pré-processamento é uma tarefa importate do processo de KDD. É desejável que
essa etapa seja a mais fácil possível, para poder agilizar a automação dessa
etapa. No caso dos algorítimos que necessitam de modelo de classificação, fica
evidente que eles necessitam de de um pré-processamento maior, pois esses
modelos necessítam de um conjunto de dados classificados para treinamento.
Isso implica que mais informação que os \textit{logs} será nessária, tornado
esses algorítmos menos adequandos que a ANN.

A etapa de pós-processamento é o processamento que será necessário para que os
dados do algorítmos determinem $x_{ob}^{i+1}$. Essa também é uma etapa na qual
os algorítmos que necessitam de um modelo de classificação são inadequados, pois
é necessario desenvolver outra camada de software para processar a classificação
de cada robô de modo a determinar o próximo movinto dos robôs do time
adversário.

Também é desejável, mas não mandatório, que o modelo treinado possa ser
modificado e reutilizado para modelar diferentes IAs. Isso visa
agilizar a modelagem durante as partidas da SSL da Robocup, pois no intervalo
de um jogo é permitido aos times modificar suas respectivas IAs.

\subsection{Comparação dos Métodos}

Os algorítimos abordados se enquadram em duas classes: algorítmos que são
completos e que necessitam de uma estrutura de classificação que são o ACO, SA,
AG e a algorítmos que são não necessitam, que é o caso da ANN\@. O caso da
lógica fuzzy será discutido na próxima seção com mais detalhes. A
tabela~\ref{table:metodos} apresenta as principais diferenças dos métodos
apresentados anteriormente. Conforme pode ser visto nessa tabela, a rede neural
é a que se mais adequa às caraterísticas desejádas.


\section{Rede Neural}\label{cap:abordagem_rede_neural}

Essa aborgem consiste em usar uma Rede Neural que tem como entrada o estado do
jogo (todas as posições, orientações, velocidades e o comando do juiz) e como
saída o estado do time adversário (todas as posições, orientações e velocidades
dos robôs do time adversário), que visa prever as ações imediatas do adversário.
Essa rede deve ser treinada para prever um time específico usando os
\textit{logs} das partidas do torneio de 2013 da \textit{RoboCup}.

% vim: tw=80


 \begin{table}
   \begin{center}
     \begin{tabular}{|c|c|c|c|c|c|}
       \hline
                         &          &           &              &             &          \\
       Critério          &  ACO     &    SA     & Lógica Fuzzy & Rede Neural & Desejado \\
                         &          &           &              &             &          \\
       \hline                                                                           
                         &          &           &              &             &          \\
       Requer modelo     &   Sim    &    Sim    &     Sim      &    Não      &   Não    \\
       de classificação  &          &           &              &             &          \\
                         &          &           &              &             &          \\
       \hline                                                                           
                         &          &           &              &             &          \\
       Estrutura de      &   Sim    &    Sim    &     Sim      &    Não      &   Sim    \\
       fácil análise     &          &           &              &             &          \\
                         &          &           &              &             &          \\
       \hline                                                                           
                         &          &           &              &             &          \\
       Degradação devido &   Sim    &    Não    &     Não      &    Não      &   Não    \\
       a discretização   &          &           &              &             &          \\
                         &          &           &              &             &          \\
       \hline                                                                           
                         &          &           &              &             &          \\
       Fácil             &   Não    &    Não    &     Não      &    Sim      &   Sim    \\
       pré-processamento &          &           &              &             &          \\
                         &          &           &              &             &          \\
       \hline                                                                           
                         &          &           &              &             &          \\
       Fácil             &   Não    &    Não    &     Não      &    Sim      &   Sim    \\
       pós-processamento &          &           &              &             &          \\
                         &          &           &              &             &          \\
       \hline                                                                           
                         &          &           &              &             &          \\
       Característica    & Modelo   & Parte das & Parte das    & Somente     & Modelo   \\
       reutilizável      & treinado & regras    & regras       & topologia   & treinado \\
                         &          &           &              &             &          \\
       \hline
     \end{tabular}
   \caption{Tabela comparativa dos métodos}
   \label{table:metodos}
   \end{center}
 \end{table}

\section{Lógica Fuzzy}

O método da Lógica Fuzzy, conforme exposto anteriormente, necessita que um
conjunto de regras seja definido. Um exemplo dessas regras foi apresentado na
seção \ref{sec_regras}. Essas regras poder ser geradas a partir de uma
análise mais detalhada do problema, mas também podem ser obtidas utilizando
algum algorítimo de aprendizagem de regras. Para aplicar esse método ao problema
analisado neste trabalho é necessário definir as regras e as distribuições das
variáveis.

A vantagem dos conjuntos difusos é que eles tornam o modelo mais robusto. A
lógica fuzzy tenta melhorar a classificação e os sistemas de decisão.

A principal desvantagem deste método é a modelagem necessária para encaixar os
conceitos descritos acima. Isso, pois o conceito de conjuntos nebulosos ainda
estão em desenvolvimento para o problema abordado neste trabalho. Essa modelagem
não é imediata, pois o problema é de classificação temporal. Não basta que as
características do ambiente sejam associadas aos conjuntos nebulosos de
características. É necessário que regras sejam especificadas estática ou
dinamicamente. No caso estático, elas seriam incorporadas ao modelo através de
especialistas. No caso dinâmico, uma solução é utilizar um classificador para
deduzir as regras.

% XXX: Adicionar classificação temporal e viabilidade
% \begin{table}
%   \begin{center}
%     \begin{tabular}{|c|c|c|c|c|c|}
%       \hline
%                         &      &           &          &              &            \\
%       Critério          & CRF  & CRF + ACO & CRF + SA & Lógica Fuzzy & Rede Neural\\
%                         &      &           &          &              &            \\
%       \hline
%                         &      &           &          &              &            \\
%       Requer modelo     &  X   &     X     &    X     &      X       &      -     \\
%       de classificação  &      &           &          &              &            \\
%                         &      &           &          &              &            \\
%       \hline
%                         &      &           &          &              &            \\
%       Estrutura de      &  X   &     X     &    X     &      X       &      -     \\
%       fácil análise     &      &           &          &              &            \\
%                         &      &           &          &              &            \\
%       \hline
%                         &      &           &          &              &            \\
%       Degradação devido &  -   &     X     &    -     &      -       &      -     \\
%       a discretização   &      &           &          &              &            \\
%                         &      &           &          &              &            \\
%       \hline
%                         &      &           &          &              &            \\
%       Fácil             &  -   &     -     &    -     &      -       &      X     \\
%       pré-processamento &      &           &          &              &            \\
%                         &      &           &          &              &            \\
%       \hline
%                         &      &           &          &              &            \\
%       Fácil             &  -   &     -     &    -     &      -       &      X     \\
%       pós-processamento &      &           &          &              &            \\
%                         &      &           &          &              &            \\
%       \hline
%                         &          &          &           &           &          \\
%       Característica    & Modelo   & Modelo   & Parte das & Parte das & Somente  \\
%       reutilizável      & treinado & treinado & regras    & regras    & topologia\\
%                         &          &          &           &           &          \\
%       \hline
%     \end{tabular}
%   \caption{Tabela comparativa dos métodos}
%   \label{regras}
%   \end{center}
% \end{table}
