\chapter{Análise das Possíveis Abordagens}\label{cap:anal_abordagens}

% TODO(depois): está muito obscuro, ligar com o final de cada heurística, não deve mudar muito esse capítulo.

Como os métodos AG (Algorítimo Genético),
SA e ACO são heurísticas de otimização, é necessário que o problema da aproximação
de uma função seja reduzido a um problema de otimização. Uma abordagem é escolher uma
base finita de funções ortonormadas e minimizar a soma dos quadrados da norma da diferença
entre os valores de treinamento e uma combinação linear das funções da base escolhida.
Foi suposto, por simplicidade, que o espaço dos vetores é um espaço vetorial com norma. Isso restringe o
espaço solução, uma vez que apenas funções que são combinação linear das funções da base serão solução.

O método da Lógica Fuzzy, conforme exposto anteriormente, necessita que um conjunto de regras seja definido.
Essas regras poder ser geradas a partir de uma análise mais detalhada do problema, mas também podem ser
aprendidas via rede neural.

Já a Rede Neural define implicitamente a estrutura interna
que minimiza a diferença entre a saída real e a saída desejada.
Entretanto, é necessário definir a topologia mais adequada para o
problema em questão. Também, apesar de não ser necessário, pode-se
decompor o problema em subproblemas e "atribuir redes neurais um
subconjunto de tarefas que coincidem com suas capacidades
inerentes" \cite[pag. 29]{haykin2001redes} com o objetivo de aumentar a
adaptabilidade da rede. Apesar de ser uma modelagem para a arquitetura
da inteligência a ser mapeada, não representa uma restrição tão
considerável quando a das abordagens citadas anteriormente.

Com base nisso, as seguintes abordagens foram estabelecidas:

\begin{enumerate}
 \item Utilizar Lógica Fuzzy com regras geradas através de uma rede neural;
 \item Utilizar Rede Neural com uma topologia mista.
\end{enumerate}

Na primeira abordagem, pretende-se utilizar os métodos de otimização para se
encontrar a melhor topologia de rede neural que melhor otimiza o conjunto de
regras buscado.

Pretende-se também, na segunda abordagem, utilizar os métodos de otimização
descritos anteriormente para se encontrar a melhor topologia que aproxima da
melhor maneira.
