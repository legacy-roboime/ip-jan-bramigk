\chapter{Análise das Possíveis Abordagens}\label{cap:anal_abordagens}

% TODO(depois): está muito obscuro, ligar com o final de cada heurística, não deve mudar muito esse capítulo.

\section{CRF}

Como os métodos AG (Algoritmo Genético), SA e ACO são heurísticas de otimização,
é necessário que o problema da aproximação de uma função seja reduzido a um problema
de otimização. A abordagem que se escolheu foi utilizar os \textit{conditional random
fields} para modelar um classificador para identificar as regras, ou táticas na
abstração da STP\@. Os métodos de otimização estudados anteriormente podem ser
aplicados para encontra os pesos do CRF.

\section{Lógica Fuzzy}

O método da Lógica Fuzzy, conforme exposto anteriormente, necessita que um conjunto
de regras seja definido. Essas regras poder ser geradas a partir de uma análise mais
detalhada do problema, mas também podem ser obtidas utilizando algum algorítimo de
aprendizagem de regras. Para aplicar esse método ao problema analisado neste trabalho
é necessário definir as regras e as distribuições das variáveis.

A vantagem dos conjuntos difusos é que eles tornam o modelo mais robusto. A lógica fuzzy
tenta melhorar a classificação e os sistemas de decisão.

A principal desvantagem deste método a modelagem necessária para encaixar os conceitos
descritos acima. Isso, pois o conceito de conjuntos nebulosos ainda estão em desenvolvimento
para o problema abordado neste trabalho. Essa modelagem não é imediata, pois o problema é de
classificação temporal. Não basta que as características do ambiente sejam associadas aos
conjuntos nebulosos de características. É necessário que regras sejam especificadas estática
ou dinamicamente. No caso estático, elas seriam incorporadas ao modelo através de especialistas.
No caso dinâmico, uma solução é utilizar um classificador para deduzir as regras.

\section{Rede Neural}\label{cap:abordagem_rede_neural}

Essa aborgem consiste em usar uma Rede Neural que tem como entrada o estado do
jogo (todas as posições, orientações, velocidades e o comando do juiz) e como
saída o estado do time adversário (todas as posições, orientações e velocidades
dos robôs do time adversário), que visa prever as ações imediatas do adversário.
Essa rede deve ser treinada para prever um time específico usando os
\textit{logs} das partidas do torneio de 2013 da \textit{RoboCup}.

% vim: tw=80

