\chapter{Análise das Possíveis Abordagens}\label{cap:anal_abordagens}

% TODO(depois): está muito obscuro, ligar com o final de cada heurística, não deve mudar muito esse capítulo.

\section{CRF}

Como os métodos AG (Algoritmo Genético),
SA e ACO são heurísticas de otimização, é necessário que o problema da aproximação
de uma função seja reduzido a um problema de otimização. A abordagem que se escolheu
foi utilizar os \textit{conditional random fields} para modelar um classificador
para identificar as regras, ou táticas na abstração da STP\@. Os métodos de otimização
estudados anteriormente podem ser aplicados para encontra os pesos do CRF.

\section{Lógica Fuzzy}

O método da Lógica Fuzzy, conforme exposto anteriormente, necessita que um conjunto de regras seja definido.
Essas regras poder ser geradas a partir de uma análise mais detalhada do problema, mas também podem ser
aprendidas via rede neural.

Já a Rede Neural define implicitamente a estrutura interna
que minimiza a diferença entre a saída real e a saída desejada.
Entretanto, é necessário definir a topologia mais adequada para o
problema em questão. Também, apesar de não ser necessário, pode-se
decompor o problema em subproblemas e "atribuir redes neurais um
subconjunto de tarefas que coincidem com suas capacidades
inerentes" \cite[pag. 29]{haykin2001redes} com o objetivo de aumentar a
adaptabilidade da rede. Apesar de ser uma modelagem para a arquitetura
da inteligência a ser mapeada, não representa uma restrição tão
considerável quando a das abordagens citadas anteriormente.

\section{Rede Neural}\label{cap:abordagem_rede_neural}

Essa aborgem consiste em usar uma Rede Neural que tem como entrada o estado do
jogo (todas as posições, orientações, velocidades e o comando do juiz) e como
saída o estado do time adversário (todas as posições, orientações e velocidades
dos robôs do time adversário), que visa prever as ações imediatas do adversário.
Essa rede deve ser treinada para prever um time específico usando os
\textit{logs} das partidas do torneio de 2013 da \textit{RoboCup}.

% vim: tw=80

