\section{Rede Neural}

O termo mais apropriado é rede neural aritificial, já que apenas rede
neural pode se referir ao sistema biológico de nervos, no entando dado o
contexto desse texto e o uso consagrado do termo ``rede neural'', esse
será usado no lugar da versão mais explícita ``rede neural artificial''.

Uma rede neural é um sistema inspirado no sistema nervoso central (em
especial o cérebro) encontrado em muitos animais. A ideia básica é ter
um grafo acíclico direcionado, DAG (do inglês \emph{Directed Acyclic
Graph}), em que cada nó abstrai um neurônio e é representado como uma
função, alguns desses nós são responsáveis pela observação e outros pela
saída e os nós de entrada alimentam os próximos nós até chegar nos nós
de saída.

\ldots{}

fontes:

\begin{itemize}
\itemsep1pt\parskip0pt\parsep0pt
\item
  http://en.wikipedia.org/wiki/Neural\_network
\item
  http://en.wikipedia.org/wiki/Artificial\_neural\_network
\item
  HAYKIN, S. Redes neurais princípios e prática.
\end{itemize}
