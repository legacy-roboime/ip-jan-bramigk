\section{Rede Neural}

O termo mais apropriado é rede neural aritificial, já que apenas rede
neural pode se referir ao sistema biológico de nervos, no entando dado o
contexto desse texto e o uso consagrado do termo ``rede neural'', esse
será usado no lugar da versão mais explícita ``rede neural artificial''.

Uma rede neural é um sistema inspirado no sistema nervoso central (em
especial o cérebro) encontrado em muitos animais. A ideia básica é ter
um grafo em que cada nó abstrai um neurônio e é representado como uma
função, alguns desses nós são responsáveis pela observação e outros pela
saída e os nós de entrada alimentam os próximos nós até chegar nos nós
de saída. \cite{haykin2001redes}

\begin{figure}[H]
  \centering
  \includegraphics[width=10cm]{figuras/rede_neural_grafo}
  \caption{Rede neural com três camadas.}\label{fig:rede_neural_grafo}
\end{figure}

A figura \ref{fig:rede_neural_grafo} exemplifica uma rede neural \emph{feedforward},
que é baseada num grafo direcionado acíclico, em que podem ser vistas 3 camadas
a primeira é chamada de camada de entrada, a última, de saída e as intermediárias,
de escondidas. \cite{shiffman2012nature}

Um dos diferencias da rede neural é a capacidade de aprender, essa heurística
forma um sistem adaptativo. Existem três tipos de aprendizados:

\begin{itemize}
\item
  Aprendizado supervisionado: alimentar a rede com um problema cuja a solução é conhecida
  e depois fornecer a resposta certa para que a rede possa se ajustar.
\item
  Aprendizado não supervisionado: consiste em buscar padrões não conhecidos, não se conhece
  a resposta certa ou se uma resposta é certa ou não.
\item
  Aprendizado por reforço: alimentar a rede com um problema cuja a solução pode ser avaliada
  em boa ou má. Esse tipo de aprendizado é comum em robótica onde o robô caminha por um ambiente
  e tem o reforço negativo ou positivo de colodir ou encontrar o objetivo.
\end{itemize}

\subsection{O Neurônio}

O bloco de construção básico de uma rede neural são os neurônios.

\begin{figure}[H]
  \centering
  \includegraphics[width=10cm]{figuras/rede_neural_perceptron}
  \caption{Perceptron de duas entradas e uma saída.}\label{fig:rede_neural_perceptron}
\end{figure}

A figura~\ref{fig:rede_neural_perceptron} mostra um perceptron.

%\ldots{}

%\begin{itemize}
%\itemsep1pt\parskip0pt\parsep0pt
%\item
%  http://en.wikipedia.org/wiki/Neural\_network
%\item
%  http://en.wikipedia.org/wiki/Artificial\_neural\_network
%\item
%  HAYKIN, S. Redes neurais princípios e prática.
%\end{itemize}
