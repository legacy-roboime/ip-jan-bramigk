\section{Logica Nebulosa} 

\subsection{Introdução}

Sistemas nebulosos aproximam funções. Eles são aproximadores universais se usarem regras suficientes. 
Neste sentido sistemas difusos podem modelar qualquer função ou sistema contínuos. Aqueles sistemas podem 
vir tanto da física quanto da sociologia, bem como da teoria do controle ou do processamento de sinais.
[ref KOSKO]

A qualidade da aproximação difusa depende da qualidade das regras. Na prática especialistas sugerem regras
difusas ou aprendem-nas através de esquemas neurais através de dados e ajustam as regras com novos dados.
Os resultados sempre aproximam alguma função não-linear desconhecida que pode mudar com o tempo. Melhores 
cérebros e melhores redes neurais resultam em melhores aproximações. [ref KOSKO]

\subsection{Modelo Aditivo Padrão(SAM)}

O sistema difuso $F:\Re^n \rightarrow \Re^p$ é em si uma árvore de regras rasa e extensa. É um aproximador
por antecipação. Existem $m$ regras da forma "Se $X$ é conjunto difuso $A$ então $Y$ é conjunto difuso $B$".
A partir desse nível o sistema depende cada vez menos em palavras. 

Cada entrada $x$ aciona parcialmente todas as regras em paralelo. Então o sistema age como um processador 
associativo a medida que calcula a saída
$F(x)$. 
%Definir a_j(x) e b_j(y)
Essas regras relacionam os conjuntos $A_j$ e $B_j$, gerando o caminho difuso $A_j x B_j$. Na prática,
é utilizado o produto para definir $ a_j x b_j (x,y) = a_j(x).b_j(y)$. Esta é a parte "padrão" no SAM.
A parte "aditiva" se refere ao fato de a entrada $x$ acionar a $j$-ésima regra em um grau $a_j(x)$ e o sistema 
soma os acionamentos ou partes escaladas dos conjuntos escalados $a_j(x)B_j$:

\begin{eqnarray}
F(x) = \frac{\sum w_i.a_i(x).V_i.c_i}{\sum w_j.a_j(x).V_j}
\end{eqnarray}

Com o volume/área $V_j$ e o centróide $c_j$ são dados por:

\begin{eqnarray}
V_j = \int{b_j(y_1,...,y_p)}_{\Re^{p}}.dy_1...dy_p > 0\\
c_j = \frac{\int{y.b_j(y_1,...,y_p)}_{\Re^{p}}.dy_1...dy_p}{V_j}
\end{eqnarray}



(KOSKO, BART; \textbf{Fuzzy Engineering}, New Jersey, USA 1997) pág. 17
