\section{Modelagem}

Pode-se observar que, como os métodos \textbf{Algorítmos Genéticos},
\textbf{\emph{Simulated Annealing}} e \textbf{\emph{Ant Colony
Optmization}} necessitam que um problema de otimazação seja definido.
Logo é necessário definir um modelo do problema e uma função associada
ao modelo para serem implementados. Isso implica que é necessário
definir a arquitetura da inteligência a ser modelada. Logo, para que
seja possível a modelagem de diferentes inteligências, será necessário
analizar a eficiência da otmização de cada arquitetura para que seja
escolhida a mais eficiente.

O método da Lógica Fuzzy também necessita que uma modelagem para o
sistema a ser considerado seja definida. Isso também restringe a
generalidade dos modelos que se enquadram nessa modelagem.

Já o a \textbf{Rede Neural} define implicitamente a estrutura interna
que mnimiza a diferença entre a saída real e a saída desejada.
Entretanto, é necessário definir a topologia mais adequada para o
problema em questão. Também, apezar de não ser necessário, pode-se
decompor o problema em subproblemas e ``atribuir redes neurais um
subconjunto de tarefas que coincidem com suas capacidades
inerentes''(pag. 29 HAYKIN, 2001) com o objetivo de aumentar a
adaptabilidade da rede. Apesar de ser uma modelagem para a arquitetura
da inteligência a ser mapeada, não representa uma restrição tão
consideravel quando a das abordagens citadas anteriormente.
