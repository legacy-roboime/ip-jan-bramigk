\section{Método da Colônia de Formigas}

Na busca por alimento, as formigas utilizam de feromônios para encontrar o melhor caminho.
Isso acontece da seguinte maneira: cada formiga deposita feromônio ao se deslocar. A partir
da avaliação da quantidade de feromônio depositada por formigas que já passaram pelo local,
formigas subsequentes tem mais probabilidade de se mover em rotas que tem mais feromônios. Ao
decorrer do tempo os feromônios vão evaporando, apagando rastros que não foram reforçados. 
Com isso, caminhos que são percorridos por mais formigas tem mais chance de serem 
percorridos por outras formigas do que aqueles que foram percorridos por menos formigas e 
caminhos que foram percorridos á pouco tempo tem mais chance de serem percorridos que caminhos
percorridos a muito tempo. A quantidade de feromônio depositado é mais intensa no trajeto de volta,
quando a comida foi encontrada. Outro fator que é levado em consideração é a qualidade da comida
encontrada, de maneira que mais feromônio é depositado quanto melhor for a fonte de alimento encontrada.
A medida que as formigas exploram o local e encontram alimento, esse procedimento tende a otimizar o
trajeto a medida cada vez mais.

Apesar dessa heurística utilizada pelas formigas ser interessante para se resolver problemas combinatórios 
do tipo NP(i.e., com complexidade não polinomial), são necessários algumas adaptações na construção
de um algoritmo computacional. 

A seguir é apresentado a meta-heurística do ACO(\textit{Ant Coloni Optmization}) algoritmo, juntamente com observações relacionadas as diferenças
entre a heurística do ACO e o comportamento natural das formigas descrito anteriormente.

\subsectio{Pseudo código do ACO}

relativas as novas trajetórias. Assim, ao decorrer das iterações deve-se convergir para o grafo apresentado na figura(\ref{graf_4}).
