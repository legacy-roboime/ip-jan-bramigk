\section{Otimização da Colônia de Formigas}

Na busca por alimento, as formigas utilizam de feromônios para encontrar o melhor caminho.
Isso acontece da seguinte maneira: cada formiga deposita feromônio ao se deslocar. A partir
da avaliação da quantidade de feromônio depositada por formigas que já passaram pelo local,
formigas subsequentes tem mais probabilidade de se mover em rotas que tem mais feromônios. Ao
decorrer do tempo os feromônios vão evaporando, apagando rastros que não foram reforçados. 
Com isso, caminhos que são percorridos por mais formigas tem mais chance de serem 
percorridos por outras formigas do que aqueles que foram percorridos por menos formigas e 
caminhos que foram percorridos á pouco tempo tem mais chance de serem percorridos que caminhos
percorridos a muito tempo. A quantidade de feromônio depositado é mais intensa no trajeto de volta,
quando a comida foi encontrada. Outro fator que é levado em consideração é a qualidade da comida
encontrada, de maneira que mais feromônio é depositado quanto melhor for a fonte de alimento encontrada.
A medida que mais formigas exploram o local e encontram alimento, esse procedimento tende a otimizar o
trajeto entre a fonte de alimento e a colônia.

Apesar dessa heurística utilizada pelas formigas ser interessante para se resolver problemas combinatórios 
do tipo NP(i.e., com complexidade não polinomial), são necessários algumas adaptações na construção
de um algoritmo computacional.

A seguir é apresentado a meta-heurística do ACO(\textit{Ant Colony Optimization}) algoritmo, juntamente com observações relacionadas as diferenças
entre a heurística do ACO e o comportamento natural das formigas descrito anteriormente.

\subsection{Pseudo código da meta-heurística do ACO}
%Algoritmo
\begin{algorithm}[H]
%Macros
\SetKwBlock{AgendarAtividade}{AgendarAtividade}{fim}
\SetKwBlock{Procedimento}{Procedimento}{fim}

\Procedimento{
  \Enqto{$n < N_{MAX\_IT}$}{
    %\tcp*[f]{$N_{MAX\_IT}$ é o número máximo de iterações\\}
    \AgendarAtividade{
      ConstruirSolucoesFormigas\\
      AtualizarFeromonios\\
      %\tcp*[f]{opcional}\\
      \tcp{opcional:}
      AcoesGlobais
    }
  }
}
%}

\caption{Pseudo código da meta-heurística do ACO}\label{meta-heuristica_aco}
\end{algorithm}
%\\
%\\

A meta-heurística do ACO pode ser subdividida em três partes, 
conforme proposto por (Dorigo, Marco; 2004): \textit{ConstruirSolucoesFormigas}, 
\textit{AtualizarFeromonios} e \textit{AcoesGlobais}.

\textit{ConstruirSolucoesFormigas} gerencia a movimentação de uma colônia de formigas
em torno dos nós vizinhos. A escolha do próximo nó é feita através de uma decisão 
estocástica que é função da quantidade de feromônio no nós vizinhos e informação heurística.
Quando uma formiga encontra uma solução, ou enquanto a solução é construída, esta avalia a
qualidade da solução(completa ou parcial) que será utilizada pelo procedimento
\textit{AtualizarFeromonios} para decidir a quantidade de feromônio que será depositada.
Outro procedimento relevante na construção da solução é a eliminação de possíveis ciclos, utilizado
por exemplo, no problema do caixeiro viajante.

\textit{AtualizarFeromonios} é o processo que atualiza os traços de feromônio depositados pelas
formigas no espaço de busca. Os traços de feromônio podem aumentar, caso uma formiga tenha visitado
o nó/conexão em questão, ou diminuir, devido ao processo de evaporação do feromônio. Esse procedimento faz com 
que nós/conexões que foram visitados por muitas formigas ou por uma formiga e que tenha levado em
uma solução boa aumentem a probabilidade de serem visitados por futuras formigas. Semelhantemente, reduz 
a probabilidade de que nós que não foram visitados por novas formigas por muitas iterações sejam visitados
novamente. Logo, este procedimento evita a convergência a caminhos sub ótimos, favorecendo também a exploração
de novas regiões do espaço de busca.

Por fim, o procedimento \textit{AcoesGlobais} é utilizado para centralizar ações que não podem ser executadas
pelas formigas individualmente. Um exemplo de ações desse tipo é a filtragem de soluções ou o favorecimento de
regiões por meio de informações globais.

O procedimento \textit{AgendarAtividade} não necessariamente é uma instrução sequencial. Pode-se, portanto,
implementá-lo de maneira sequencial ou paralela, síncrona ou assincronamente. O tipo de abordagem que será
utilizada depende das características do problema que se deseja resolver.
