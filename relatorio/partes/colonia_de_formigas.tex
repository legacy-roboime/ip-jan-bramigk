# Método da Colônia de Formigas

Na busca por alimento, as formigas se utilizam de feromônios para encontrar o melhor caminho.
Isso acontece da seguinte maneira: cada formiga deposita feromônio ao se deslocar. A partir
da avaliação da quantidade de feromônio depositada por formigas que já passaram pelo local,
formigas subsequentes tem mais probabilidade de se mover em rotas que tem mais feromônios. Ao
decorrer do tempo os feromônios vão evaporando, reduzindo a quantidade percebida por outras
formigas. Com isso, caminhos que são percorridos por mais formigas tem mais chance de serem 
percorridos por outras formigas do que aqueles que foram percorridos por menos formigas e 
caminhos que foram percorridos á pouco tempo tem mais chance de serem percorridos que caminhos
percorridos a muito tempo. É necessário também, para que aconteça a convergência, que as formigas depositem
feromônios no tragéto de ida e de volta.

A seguir é apresentado uma ilustração da implementação deste algoritmo a solução do TSP(Traveling Salesman Problem).

## Implementação do MCF a solução de um caso do TSP

Deseja-se encontrar o menor caminho do grafo não direcionado com pesos iguais da figura (\ref{graf_1}) do nó A ao nó B.

A primeira iteração do algoritmo acontece, regando grafo da figura (\ref{graf_2}), onde ? representa a quantidade de feromônios.
Inicialmente a probabilidade da formiga escolher os nós segue uma distribuição probabilística uniforme. A parti da segunda iteração,
esta escolha levará em conta a quantidade de feromônio em cada aresta do grafo. Na figura (\ref{graf_3}, pode-se ver que no primeiro
passo da trajetória a presença de feromônio não levou a escolha do mesmo nó da trajetória da formiga anterior, já no segundo e último
passo sim. Ao final de cada trajetória o nível de feromônio é atualizado, eliminando feromônios antigos e adicionando as contribuições
relativas as novas trajetórias. Assim, ao decorrer das iterações deve-se convergir para o grafo apresentado na figura(\ref{graf_4}).


