\chapter{Introdução}

\section{Contextualização}

\par O Laboratório de robótica da seção de computação do IME possui um time que
participa de uma competição de futebol de robôs oficializada pela \textit{RoboCup}.
Essa competição consiste em partidas de futebol autônomas entre robôs com formato
de cilindro de 18cm de diâmetro e 15cm de altura em um campo de 6m por 4m. Além disso
o sistema de visão computacional é padronizado e centralizado. O importante é notar
que a competição é autônoma e por isso há vários desafios envolvidos.
Um dos desafios é modelar o inimigo, de maneira geral modelar um time desconhecido
mas com o intuito de conseguir prever com certa confiança suas decisões para tomar
decisões melhores durante o jogo e conseguir um desempenho melhor.
As partidas são registradas e por isso podem ser feitas análises pré jogo além
de em tempo real. A principal vantagem é a liberdade sobre quão eficiente é o método
de análise.

\par O objeto a ser estudado é, portanto, a análise dos registros dessas partidas 
com foco principal em modelar um time desconhecido, denominados \textit{log}. O modelo 
gerado será utilizado para prever os movimentos do time adversário. Isso deve tornar 
as jogadas planejadas mais  eficazes, uma vez que já levarão em conta os movimentos 
futuros do time adversário.

\section{Problema}

O problema descrito anteriormente pode ser enunciado da seguinte maneira:

Dado um conjunto $E_p$ de possíveis estados do jogo, $A_{t\_ad}$ um conjunto de ações dos robôs
do time adversário, um sistema $f: {E_p} \leftarrow {A_{t\_ad}}$. Encontrar o sistema 
$F: {E_p} \leftarrow {A_{t\_ad}}$ que minimiza:

\begin{equation}
e_{total} = \sum_i E[f(x_i),F(x_i)]
\end{equation}

Onde $E: A_{t_ad} \leftarrow \Re^+$ é uma função erro que é proporcional a diferença entre as ações.

Como o sistema $F$ pode ser escrito de várias maneiras, é necessário assumir uma forma menos genérica
para $F$ de modo a possibilitar a análise e reduzir o espaço de busca. Uma vez que o conjunto das possíveis estruturas de $F$
escolhido afeta o valor mínimo $F_{min}$, tem-se que o mínimo encontrado pode minimizar menos o erro

\section{Estrutura do Trabalho}

\par Assim, inicialmente é apresentado a revisão bibliográfica do TDP(\textit{Team Description Paper}) do time Skuba
campeão das \textit{RoboCups} de 2011 e 2012. 
\par Em seguida são apresentados os métodos de modelagem de sistemas para modelar os
a time adversário com base nos \textit{logs} de jogos anteriores. Em seguida, são apresentados
métodos de otimização mais comum

