\chapter{Introdução}

\section{Contextualização}

Cooperação é uma relação de aujuda mútua entre indivíduos e/ou entidades, no sentido
de alcançar objetivos em comum, utilizando métodos mais ou menos consensuais. Projetar
robôs para trabalhar juntos não é uma tarefa trivial. A tarefa se torna mais difícil
ainda quando os robôs devem ser autônomos. Nos trabalhos em que o domínio da aplicação
é o futebol de robôs, é comum a idéia de se distribuir papéis dinamicamente entre os
membros que compoem a equipe. Este tipo de modelo em um ambiente cooperativo necessita
de que todos os membros tenham características em comum, não havendo especialização
entre eles. Adicionalmente, é comum a adoção de objetivos globais e locais em
trabalhos robóticos cooperativos.

A idéia de robôs jogando futebol foi mencionada pela primeira vez pelo professor
Alan Mackworth(\textit{Univercity of British Columbia}, Canadá) em um artigo intitulado
\textit{"On Seeing Robots"}, apresentado no Vision Interface 92 e posteriormente publicado em
um livro chamado \textit{Computer Vision: System, Theory and Applications}. Independentemente,
um grupo de pesquisadores japoneses organizou um \textit{Workshop} no \textit{Ground Challange in Artificial Inteligence}, em Outubro de 1992, Tókio, discutindo e propondo problemas que representavam grandes desafios. Esse Workshop os levou a sérias discussões sobre 
usar um jogo de futebol para promover ciência e tecnologia. Estudos foram feitos para
analisar a viabilidade prática dessa idéia. Os resultados desses estudos mostram que
a idéia era viável, desejável e englobal diversas aplicações práticas. Em 1993 um
grupo de pesquisadores incluindo Minoru Asada, Yasuo Kuniyoshi e Hiroaki Kitano,
lançaram uma competição robótica chamada de Robot \textit{J-League} (fazendo uma analogia à
\textit{J-League}, nome da Liga Japonesa de Futebol Profissional). Em um mês vários 
pesquisadores já se pronunciavam dizendo que a iniciativa deveria ser estendida ao 
âmbito internacional. Surgia então a \textit{Robot World Cup Initiative} (RoboCup).

RoboCup é uma competição destinada a desenvolver os estudos na área de robótica e
inteligência artificial (IA) através de um competição amigável. Além disso, ela tem
como objetivo até 2050, desenvolver uma equipe de robôs humanóides totalmente 
autônomos capazes de derrotar a equipe campeã mundial de futebol humano. a competição
possui várias modalidades. Para esta pesquisa o fator custo e o amplo número de opções
de desenvolvimento foram relevantes na escolha da Small Size Robot League, mais 
conhecida como F180. De acordo com as regras da RoboCup F180, as equipes devem ser
compostas por no máximo 6 robôs, sendo um deles o goleiro. O goleiro deve ser 
designado antes do início do jogo. Durante o jogo nenhuma interferência humana é
permitida com o sistema de controle dos robôs. As regras da RoboCup F180 permitem a
utilização de visão global e controle centralizado dos robôs. O sistema de controle
dos robôs geralmente é externo, recebe os dados da câmera localizada acima do campo,
processa os dados, determina qual comando deve ser executado em cada robô e envia este
comando através de ondas de rádio aos robôs. Embora a maioria das equipes utilize
visão e controle centralizado, algumas equipes têm utilizado visão e decisão locais.

\section{Objetivos}

A pesquisa tem por finalidade realizar um estudo sobre métodos para que agentes possam
interagir de maneira eficiente com outros agentes dos quais este não tem controle ou
conhecimento prévio. Devido a complexidade do problema, apesquisa se propoem a 
realizar esse aprendizado dos agentes externos estaticamente para posteriormente serem
incorporados ao sistema do futebol de robôs do Laboratório de Robótica, denominado
SSL-RoboIME. Se possível, deseja-se que essa modelagem/aprendizagem seja feita
dinamicamente.

\section{Justificativa}

O futebol de robôs, problema padrão de investigação internacional, reúne grande parte
dos desafios presentes em problemas do mundo real a serem resolvidos em tempo real.
As soluções encontradas para o futebol de robôs podem ser estendidas, possibilitando
o uso da robótica em locais de difícil acesso para humanos, ambientes insalubres e 
situações de risco de vida iminente, incluindo a exploração espacial.

Há diversas novas áreas de aplicação da robótica, tais como exploração espacial e submarina, navegação em ambientes inóspitos e perigosos, serviço de assistência médica
e cirúrgica, além do setor de entretenimento, podem beneficiar-se do uso de sistemas
múlti-robôs. Nestes domínios de aplicação, sistemas de multirobôs deparam-se sempre
com tarefas muito difíceis de serem efetuadas por um único robô. Um time de robôs pode
prover redundância e contribuir cooperativamente para resolver o problema em questão.
Com efeito, eles podem resolver o problema de maneira mais confiável, mais rápida e 
mais econômica, quando comparado com o desempenho de um único robô.

Devido a grande complexidade do problema de interação com humanos, faz-se necessário 
que os robôs sejam dotados de uma capacidade de aprendizado para facilitar a interação
destes com o mundo real. Isso é relevante tanto para aplicações industriais quanto para aplicações em resgates e militares. Isso diminue a necessidade de modelagem
exata dos hambientes em que os sistemas robóticos serão introduzidos e permite que
a adaptação a ambientes complexos seja realizada através da exposição destes sistemas
às possíveis situações de trabalho. Através da incorporação do sistema de
aprendizagem, cituações não consideradas podem ser incorporadas ao algoritmo de
controle dos robôs e permitir que esses reajam de maneira mais eficiente em futuras
cituações semelhantes.

\section{Metodologia}

Para atingir os objetivos propostos será seguida a seguinte metodologia.
Inicialmente o problema a ser investigado será definido formalmente, 
utilizando-se de definições e teoremas.

Posteriormente a bibliografia é revisada. São evidenciados os métodos comunmente
utilizados para a abordagem do problema mais geral de classificação. Também são
analisádos trabalhos aplicados especificamente á RoboCup F180.

A seguir são analisadas as heurísticas levantadas durante a
revisão da bibliografia. Cada uma delas é descrita. Após descrição sumária de
de cada heurística, são apresentados os respectivos exemplos de cada aplicação. Ao fim de cada exemplo são apresentados como as respectivas heurísticas se encaixam na
resolução do problema. 

A seguir são apresentadas abordagens envolvendo as heurísticas introduzidas anteriormente. Cada abordagem relaciona no mínimo uma dessas herísticas.
Ao final de cada seção objetiva-se apresentar uma metodologia a ser aplicada
na resolução do problema. Uma análise das características de cada abordagem
é feita ao final da definição cada abordagem.

Posteriormente, utilizando-se das análises de cada abordagem, uma das abordagens
é selecionada. Essa análise é estudada mais profundamente. São descritos possíveis 
algorítmos para implementar essa abordagem. Também são analisados os 
\textit{Frameworks} mais comumente utilizados, levando os pontos positivos e 
negativos. Também são desenvolvidos métricas para análise da abordagem selecionada.

Para validar os estudos desenvolvidos, o projeto de um software
que analise os dados dos \textit{Logs} disponíveis no site da SSL é apresentado.

\section{Estrutura do Trabalho}

No capítulo \ref{cap:def_problema} o problema a ser resolvido é definido formalmente.

No capítulo \ref{cap:rev_bibliografica} a bibliografia é revisada.

No capítulo \ref{cap:heuristicas} são apresentados tutoriais relacionados às 
heurísticas comumente utilizadas em problemas de classificação.

No capítulo \ref{cap:anal_abordagens} são descritas possíveis abordagens a serem
seguidas para resolução do problema.

No capítulo \ref{cap:cronograma} o cronograma de estudo é apresentado.

No capítulo \ref{cap:conclusao} são apresentadas as principais conclusões atingidas
até o momento.