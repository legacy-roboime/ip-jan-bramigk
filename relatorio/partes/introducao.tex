\chapter{Introdução}

\section{Contextualização}

\par O Laboratório de robótica da seção de computação do IME possui um time que
participa de uma competição de futebol de robôs oficializada pela Robocup.

\par Essa competição consiste em partidas de futebol autônomas entre robôs com formato
de cilindro de 18cm de diâmetro e 15cm de altura em um campo de 6m por 4m. Além disso
o sistema de visão computacional é padronizado e centralizado. O importante é notar
que a competição é autônoma e por isso há vários desafios envolvidos.

\par Um dos desafios é modelar o inimigo, de maneira geral modelar um time desconhecido
mas com o intuito de conseguir prever com certa confiança suas decisões para tomar
decisões melhores durante o jogo e conseguir um desempenho melhor.

\par As partidas são registradas e por isso podem ser feitas análises pré-jogo além
de em tempo real. A principal vantagem é a liberdade sobre quão eficiente é o método
de análise.

\par O objeto a ser estudado é a análise dos registros dessas partidas com foco principal
em modelar um time desconhecido.
