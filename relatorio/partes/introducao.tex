\chapter{Introdução}

% TODO: Seções necessárias na introdução:
% - Contextualização
% - Motivação
% - Objetivo
% - Justificativa
% - Metodologia
% - Estrutura

\section{Objetivo}

% TODO(???): revisar objetivo

\par O objeto a ser estudado é pesquisar técnicas para analisar registros de partidas do futebol de robôs,
denominados \textit{logs}, com foco principal em modelar um time desconhecido. O modelo
gerado será utilizado para prever os movimentos do time adversário. Isso deve tornar
as jogadas planejadas mais  eficazes, uma vez que já levarão em conta os movimentos
futuros do time adversário.

\section{Justificativa}

\par O Laboratório de robótica da seção de computação do IME possui um time que
participa de uma competição de futebol de robôs oficializada pela \textit{RoboCup}.
Essa competição consiste em partidas de futebol autônomas entre robôs com formato
de cilindro de 18cm de diâmetro e 15cm de altura em um campo de 6m por 4m. Além disso
o sistema de visão computacional é padronizado e centralizado. O importante é notar
que os times são autônomos e por isso há vários desafios envolvidos.
Um dos desafios é modelar o time adversário, de maneira geral modelar um time desconhecido
mas com o intuito de conseguir prever com certa confiança suas decisões para tomar
decisões melhores durante o jogo e conseguir um desempenho melhor.
As partidas são registradas e por isso podem ser feitas análises pré jogo além
de em tempo real. A principal vantagem é a liberdade sobre quão eficiente é o método
de análise.

\section{Problema}

% TODO[vbramigk]: tirar isso da introdução e explicar melhor.

O problema descrito anteriormente pode ser enunciado da seguinte maneira:

Dado um conjunto $E_p$ de possíveis estados do jogo, $A_{t\_ad}$ um conjunto de ações dos robôs
do time adversário, um sistema $f: {E_p} \rightarrow {A_{t\_ad}}$. Encontrar o sistema 
$F: {E_p} \rightarrow {A_{t\_ad}}$ que minimiza:

\begin{equation}
e_{total}(F) = \sum_i E[f(x_i),F(x_i)]
\end{equation}

Onde $E: A_{t_ad} \rightarrow \Re^+$ é uma função erro que é proporcional a diferença entre as ações e
${x_i} \in E_p$ é um conjunto limitado de \textit{logs}.

Como o sistema $F$ pode ser escrito de várias maneiras, é necessário assumir uma forma menos genérica
para $F$ de modo a possibilitar a análise e reduzir o espaço de busca. Uma vez que o conjunto das possíveis estruturas de $F$
escolhido afeta o valor mínimo $F_{min}$, tem-se que o erro $e_{total}(F_{min})$ esta atrelado a estrutura da função
$F$ escolhida. Logo, pretende-se aplicar um método de otimização que tem como parâmetros elementos que definem a estrutura
escolhida. Por exemplo, o número de camadas arquitetura da rede neural ou as regras utilizadas no sistema difuso.

\section{Estrutura do Trabalho}

% TODO(depois): Explicar o que cada capítulo tem.

\par Assim, inicialmente são apresentados os métodos candidatos a serem utilizados em um algorítimo de predição
automática do comportamento de um time de futebol de robôs, isto é, determinar a função $F$ especificada no problema anterior.
juntamente com heurísticas de otimização combinatorial para otimizar a estrutura desta função. Em seguida são propostas
abordagens envolvendo o tipo de estrutura da função $F$ e o método utilizado para procurar a estrutura pseudo ótima. Por fim é
apresentado as próximas etapas previstas para dar continuidade ao trabalho.
