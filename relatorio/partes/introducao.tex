\chapter{Introdução}
% =======================================
% TODO: falta a sec Motivação
% =======================================

Cooperação é uma relação de ajuda mútua entre indivíduos e/ou entidades, no sentido
de alcançar objetivos em comum, utilizando métodos mais ou menos consensuais. Projetar
robôs para trabalhar juntos não é uma tarefa trivial. A tarefa se torna mais difícil
ainda quando os robôs devem ser autônomos. Nos trabalhos em que o domínio da aplicação
é o futebol de robôs, é comum a idéia de se distribuir papéis dinamicamente entre os
membros que compõem a equipe. Este tipo de modelo em um ambiente cooperativo se torna
menos complexo quando todos os membros tem características em comum, não havendo especialização
entre eles. Adicionalmente, é comum a adoção de objetivos globais e locais ao sistema.

\section{\textit{RoboCup Small Size League}}

A idéia de robôs jogando futebol foi mencionada pela primeira vez pelo professor
Alan Mackworth (\textit{Univercity of British Columbia}, Canadá) em um artigo intitulado
\textit{"On Seeing Robots"}, apresentado no \textit{Vision Interface 92} e posteriormente publicado em
um livro chamado \textit{Computer Vision: System, Theory and Applications}. Independentemente,
um grupo de pesquisadores japoneses organizou um \textit{Workshop} no \textit{Ground Challange
in Artificial Inteligence}, em Outubro de 1992, Tóquio, discutindo e propondo problemas que
representavam grandes desafios. Esse \textit{Workshop} os levou a sérias discussões sobre
usar um jogo de futebol para promover ciência e tecnologia. Estudos foram feitos para
analisar a viabilidade dessa idéia. Os resultados desses estudos mostram que
a idéia era viável, desejável e engloba diversas aplicações práticas. Em 1993 um
grupo de pesquisadores incluindo Minoru Asada, Yasuo Kuniyoshi e Hiroaki Kitano,
lançaram uma competição robótica chamada de Robot \textit{J-League} (fazendo uma analogia à
\textit{J-League}, nome da Liga Japonesa de Futebol Profissional). Em um mês vários
pesquisadores já se pronunciavam dizendo que a iniciativa deveria ser estendida ao
âmbito internacional. Surgia então a \textit{Robot World Cup Initiative} (RoboCup).

RoboCup é uma competição destinada a desenvolver os estudos na área de robótica e
inteligência artificial (IA) através de um competição amigável. Além disso, ela tem
como objetivo até 2050, desenvolver uma equipe de robôs humanóides totalmente
autônomos capazes de derrotar a equipe campeã mundial de futebol humano. A competição
possui várias modalidades. Neste trabalho será analizada a \textit{Small Size Robot League} (SSL),
também conhecida como F180. De acordo com as regras da SSL, as equipes devem ser
compostas por 6 robôs, sendo um deles o goleiro. O goleiro deve ser
designado antes do início do jogo. Durante o jogo nenhuma interferência humana é
permitida com o sistema de controle dos robôs. É fornecido aos times um sistema de
visão global e esses controlam seus robôs com máquinas próprias. O sistema de controle
dos robôs geralmente é externo, recebe os dados de um conjunto de câmeras
localizada acima do campo, processa os dados, determina qual comando deve ser executado
em cada robô e envia este comando através de ondas de rádio aos robôs. Embora seja
permitido que as equipes utilizem sistemas próprios de visão, a maioria das equipes utiliza
a visão centralizada.

\section{Motivação}

O futebol de robôs, problema padrão de investigação internacional, reúne grande parte
dos desafios presentes em problemas do mundo real a serem resolvidos em tempo real.
As soluções encontradas para o futebol de robôs podem ser estendidas, possibilitando
o uso da robótica em locais de difícil acesso para humanos, ambientes insalubres e
situações de risco de vida iminente.

Há diversas novas áreas de aplicação da robótica, tais como exploração espacial e submarina,
navegação em ambientes inóspitos e perigosos, serviço de assistência médica
e cirúrgica, além do setor de entretenimento, podem beneficiar-se do uso de sistemas
multi-robôs. Nestes domínios de aplicação, sistemas de multirobôs deparam-se sempre
com tarefas muito difíceis de serem efetuadas por um único robô. Um time de robôs pode
prover redundância e contribuir cooperativamente para resolver o problema em questão.
Com efeito, eles podem resolver o problema de maneira mais confiável, mais rápida e
mais econômica, quando comparado com o desempenho de um único robô.

Devido a grande complexidade do problema de interação com humanos, faz-se necessário
que os robôs sejam dotados de uma capacidade de aprendizado para facilitar a interação
desses com o mundo real. Isso é relevante tanto para aplicações industriais quanto para
aplicações em resgates e militares. Isso diminui a necessidade de modelagem
exata dos ambientes em que os sistemas robóticos serão introduzidos e permite que
a adaptação a ambientes complexos seja realizada através da exposição destes sistemas
às possíveis situações de trabalho. Através da incorporação do sistema de
aprendizagem, situações não consideradas podem ser incorporadas ao algoritmo de
controle dos robôs e permitir que esses reajam de maneira mais eficiente em futuras
cituações semelhantes.

\section{Objetivos}

A pesquisa tem por finalidade realizar um estudo sobre métodos prever comportamentos.
Isso para que agentes controláveis
possam interagir de maneira eficiente com agentes do time inimigo, que não são controláveis.
A pesquisa se propoem a realizar esse aprendizado dos agentes inimigos baseado em gravações
coletadas da visão e juiz de partidas, também conhecidas como \textit{logs}, para posteriormente
serem incorporados ao sistema de inteligência da equipe de futebol de robôs do Laboratório
de Robótica, denominada RoboIME.
% NOTE: acho que não é o caso disso:
%Se possível, deseja-se que essa modelagem/aprendizagem seja feita dinamicamente durante a partida da SSL.

\section{Justificativa}

Um método concreto que possa prever o comportamento de agentes inteligêntes de um jogo de
futebol de robôs permite com que seja possível prever o comportamento de um time inimigo.
Com tal mecanismo é possível melhorar a Inteligência Artificial em uso pelo RoboIME
para tomar decisões que levem a resultados melhores e, por consequencia, ganhar mais partidas.
Outras equipes participantes da SSL usam mecanismo de predição do inimigo.
Portanto, é de grande importância desenvolver também tal mecanismo para acompanhar a evolução
das tecnologias envolvidas.

\section{Metodologia}

Para atingir os objetivos propostos será seguida a seguinte metodologia.
Inicialmente o problema a ser investigado será definido formalmente,
utilizando-se de definições e teoria apropriada.

Posteriormente a bibliografia é revisada. São evidenciados os métodos comumente
utilizados para a abordagem do problema mais geral de classificação. Também são
analisádos trabalhos aplicados especificamente à SSL.

A seguir são analisadas as heurísticas levantadas durante a
revisão da bibliografia. Cada uma delas é descrita. Após descrição sumária de
de cada heurística, são apresentados os respectivos exemplos de cada aplicação.
Ao fim de cada exemplo são apresentados como as respectivas heurísticas se encaixam na
resolução do problema.

A seguir são apresentadas abordagens envolvendo as heurísticas introduzidas
anteriormente. Cada abordagem relaciona no mínimo uma dessas heurísticas.
Ao final de cada seção é apresentado uma metodologia a ser aplicada
na resolução do problema. Uma análise das características de cada abordagem
é feita ao final da descrição de cada abordagem.

Posteriormente são descritos possíveis algorítmos para implementar essa abordagem. Também são
desenvolvidos métricas para análise de abordagem.

% NOTE: não é bem verdade isso:
%Para validar os estudos desenvolvidos, o projeto de um software
%que analise os dados dos \textit{logs} disponíveis no site da SSL é apresentado.

\section{Estrutura do Trabalho}

No capítulo \ref{cap:def_problema} o problema a ser resolvido é definido formalmente.

No capítulo \ref{cap:rev_bibliografica} a bibliografia é revisada.

No capítulo \ref{cap:heuristicas} são apresentados tutoriais relacionados às
heurísticas comumente utilizadas em problemas de classificação.

No capítulo \ref{cap:anal_abordagens} são descritas possíveis abordagens a serem
seguidas para resolução do problema.

No capítulo \ref{cap:cronograma} o cronograma de estudo é apresentado.

No capítulo \ref{cap:conclusao} são apresentadas as principais conclusões atingidas
até o momento.
