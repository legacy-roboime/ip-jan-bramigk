\chapter{Especificação dos Conceitos Envolvidos na SSL}\label{cap:def_problema}

\section{KDD}

De acordo com \cite{passos2005datamining}, a área responsável por estudar o processo
de extração de informação com base em um conjunto de dados é chamada de Descoberta
de Conhecimento em Bases de Dados, denominada KDD (\textit{Knowledge Discovery in
Databases}). As etapas operacionais da KDD são:

\begin{itemize}
  \item \textit{Pré-processamento}: etapa relacionada à captação,
        organização e ao tratamento dos dados, com o objetivo de
        preparar os dados para a etapa de \textit{Mineração de Dados};
  \item \textit{Mineração de Dados}: etapa relacionada à busca por
        conhecimentos úteis no conjunto de dados analisado;
  \item \textit{Pós-processamento}: etapa relacionada ao tratamento do
        conhecimento obtido na etapa de \textit{Mineração de Dados}, com
        o objetivo de viabilizar a utilização desse conhecimento
        (normalmente esta etapa é desnecessária).
\end{itemize}

% Este trabalho objetiva, através do processo de \textit{Mineração de Dados}, extrair
% informação dos $logs$ (definido a seguir) de um jogo do futebol de robôs. O
% processo de preprocessamento será estudado futuramente. Deseja-se, para facilitar
% a avaliação dos algoritmos a serem desenvolvidos, que o resultado do KDD em questão
% seja pós processado. Esse processo também será estudado futuramente. A seção a
% seguir descreve de maneira detalhada o domínio que será estudado.

\section{Domínio}

O problema do time de futebol de robôs que será modelado neste capítulo é baseado
no modelo de planejador apresentado em \cite{zickler}, na teoria de agentes descrita
em \cite{russellnorvig} e na arquitetura de sistemas multiagentes
STP (Skills, Tactics and Plays) descrita em \cite{bowling2003plays}.

\subsection{Arquitetura}

A arquitetura a ser considerada é baseada em \cite{felixnavarro}.
A \textit{RoboCup Small Size League} (SSL) envolve problemas de diversas áreas
da engenharia. Logo, com o objetivo de facilitar a compreensão do
problema, a arquitetura a ser considerada é apresentada na figura
\ref{arquitetura_ssl}. Essa arquitetura é composta pelos seguintes
sistemas:

\begin{itemize}
  \item Câmeras Visão: conjunto de câmeras \textit{firewire} que captura as imagens do
        campo e as envia para a SSL-Vision;
  \item Comunicação: módulo responsável por receber os parâmetros
        dos motores, drible, chute baixo e chute alto e enviar o comando via
        ondas de rádio para os robôs;
  \item Execução: módulo responsável por realizar a tomada de decisões
        em baixo nível de quais ações os robôs devem realizar a partir
        da decisão tomada pelo módulo de inteligência;
  \item Inteligência: módulo responsável por realizar a tomada de
        decisão em alto nível de quais ações os robôs devem realizar
        tendo auxílio de um módulo de Simulação;
  \item Mundo Real: campo de futebol real, onde os times 1 e 2 interagem
        através de seus respectivos robôs
  \item Referee-Box: \textit{software} padronizado pela Robocup para que as
        regras da competição sejam cumpridas sem que haja intervenção
        humana excessiva durante uma partida;
  \item Simulação: módulo do \textit{software} do time responsável por simular
        o ambiente da partida, tendo como entrada os parâmetros do mundo
        real;
  \item \textit{Software} Time 1/2: \textit{software} do time 1/2;
  \item SSL-Vision: \textit{software} padronizado pela Robocup que permite a
        integração com um conjunto de câmeras \textit{firewire} que capturam
        imagens do campo e as processa, extraindo informações sobre os objetos na
        imagem;
  \item Time 1/2: time de robôs que executa os comandos recebidos pelo
        sistema de transmissão do time 1/2;
  \item Transmissão 1/2: sistema de transmissão do time 1/2;
  \item World Model: módulo responsável por modelar o mundo e dar
        confiabilidade aos dados que serão enviados ao módulo de
        Inteligência e são oriundos do Referee-Box e SSL-Vision.
\end{itemize}

\begin{landscape}
  \begin{figure}[thpb]
    \centering
    \includegraphics[width=20cm]{imgs/arquitetura_ssl}
    \caption{Arquitetura básica da SSL}
    \label{arquitetura_ssl}
  \end{figure}
\end{landscape}

Apesar de modelar a maioria dos sistemas empregados atualmente na
SSL, existem alguns problemas na arquitetura descrita anteriormente. Entretanto,
são necessárias algumas definições apresentadas nas próximas secções para que eles possam ser discutidos.

\subsection{Definições}

\begin{defi}[Corpo Rígido]
  Um corpo rígido $r$ é definido por dois subconjuntos disjuntos
  de parâmetros $r= \langle \hat{r}, \bar{r} \rangle$ em que:
  \begin{itemize}
    \item $\hat{r} = \langle \alpha, \beta, \gamma, \omega \rangle$,
    que são os parâmetros de estado mutaveis, respectivamente:
    posição ($\mathbb{R} ^{3}$), orientação($\mathbb{R} ^{3}$),
    velocidade linear($\mathbb{R} ^{3}$), velocidade angular
    ($\mathbb{R} ^{3}$)

    \item $\bar{r} :$ parâmetros imutáveis corpo que descrevem sua
    natureza fixa e que permanece constante ao longo do curso de 
    planejamento.
  \end{itemize}

  Exemplos de parâmetros considerados nesta modelagem imutaveis são:
  coeficiente de atrito estático e dinâmico, descrição $3D$ do corpo
  (por exemplo, por meio de um conjunto de primitívas $3D$), centro de
  massa no referencial do corpo, coeficiente de restituição,
  coeficiente de amortecimento linear e angular. Note que a matrix de
  rotação $R\in\mathbb{R}^{3\times 3}$ gerada a partir da rotação
  em torno de um vetor unitário direção $d\in\mathbb{R}^{3}$  de
  $\theta \in \mathbb{R}$ radianos satisfaz a equação
  $R = exp\left( \beta \right)$ (em que $\beta = d. \theta \in \mathbb{R} ^{3}$)
  \cite{math2robotics}.
\end{defi}

\begin{defi}[Bola]\label{def:bola}
  Bola é um corpo rígido $\hat{b}$, no qual somente as componentes 
  $\langle x,y \rangle$ do o parâmetro $\hat{b}.\alpha$ são
  observáveis. De acordo com a
  aquitetura considerada para a SSL descrita na figura
  \ref{arquitetura_ssl}, tem-se que, a partir de uma sequênicia
  de quadros, é possível obter um valor estimado para o parâmetro
  $\hat{b}.\gamma$ a partir do intervalo entre os dados recebidos
  da \textit{SSL-Vision} e da equação $ \gamma \approxeq 
  \frac{\Delta \alpha}{\Delta t} $. Entretanto, uma vez que a componente
  $z$ de $\hat{b}.\alpha$ não é observável, $\hat{b}.\gamma.z$ 
  não pode ser estimada a partir do intervalo entre os dados recebidos
  da \textit{SSL-Vision}. Semelhantemente,  uma vez que o
  parâmetro $\hat{b}.\beta$ também não pode ser observado,
  não se pode estimar o valor de $\hat{b}.\omega$ com exatidão.
\end{defi}

\begin{defi}[Robô]
  Robô $rob$ é um conjunto de sistemas compostos de corpos rígidos,
  \textit{hardware} e \textit{firmware}. São eles:

  \begin{itemize}
    \item Drible: imprime um torque a bola;
    \item Chute baixo: imprime uma força à bola $\hat{b}$
          e, possivelmente, um torque, com o objetivo de
          alterar as componentes $\langle x,y \rangle$
          do parâmetro $\hat{b}.\gamma$;
    \item Chute alto: imprime uma força à bola $\hat{b}$
          e, possivelmente, um torque, com o objetivo de
          alterar as componentes $\langle x,y,z \rangle$
          do parâmetro $\hat{b}.\gamma$, com $\hat{b}.\gamma_z \neq 0$;
    \item Receptor: recebe comandos enviados pelo sistema de
          transmissão de seu respectivo time;
    \item Sistema de movimentação: imprime uma força e um torque
          ao centro de massa global do $rob$;
  \end{itemize}

  Por meio desses sistemas, cada robô $rob$ pode executar um conjunto de ações $A_{rob}$.

  Apesar de o modelo descrito acima abranger a mairia dos
  robôs utilizados atualmente por equipes da SSL, é importante
  ressaltar que o robô pode ter um conjunto de sensores que
  poderiam coletar informações adicionais às transmitidas pela
  \textit{SSL-Vision} juntamente com um sistema de transmissão
  para enviá-las ao software do seu respectivo time. Isso é
  interessante, pois, conforme observado na definição \ref{def:bola},
  o parâmetro $\hat{b}.\beta$ não é observável. Como o sitema de
  drible impõe um torque à bola, por meio de um sensor, é possível
  estimar o valor de $\hat{b}.\omega$. Sem esse sensor, não é possível
  prever com exatidão a tragetória da bola somente com informações de
  simulação ou da visão.

\end{defi}

\begin{defi}[Time]\label{def:time}
  Sejam os seguintes parâmetros:

  \begin{description}
    \item $Rob_c$ o conjunto dos robôs controlados;
    \item $Rob_i$ o conjunto dos robôs inimigos, isto é, não controlados;
    \item $X$ o espaço de estado de todos os corpos rígidos envolvidos na partida considerada;
    \item $x_{init} \in X$ o estado inicial;
    \item $X_{goal}\subset X$ o conjunto de estados objetivo;
    \item $x_{ob}^{i}$ os estados observados pelo módulo \textit{SSL-Vision} no instante $i$;
    \item $X_{ob}^{i} =  \lbrace{x_{ob}^{0} = x_{init},...,x_{ob}^{i}}\rbrace$;
    \item $Sk \subset A_{rob}$ um conjunto de \textit{skills};
    \item $prob: X \longrightarrow [0,1]$ uma distribuição de probabilidade, cujo argumento é
          $x \in X$;
    \item $tk = G(V \in Sk, E \in {prob} )$ o conjunto de todas as táticas possíveis
          formadas a partir de grafos orientados, em que os vértices são \textit{skills} $sk \in Sk$
          e as arestas são $prob$ associadas a possibilidade de ocorrerem as transições
          entre uma skill e outra;
    \item $A_c = A_{rob 1} \cup ... \cup A_{rob n_c}$ o conjunto das ações possíveis de $Rob_c$;
    \item $A_i = A_{rob 1} \cup ... \cup A_{rob n_i}$ o conjunto das ações possíveis de $Rob_i$;
    \item $A = A_c \cup A_i$ o conjunto das ações possíveis de $Rob_c \cup Rob_i$;
    \item $e$ a função de transição de estado que pode aplicar uma ação $a\in A$ em um estado particular
          $x \in X$ e computar o estado seguinte $x^{'} \in X$, i.e.:
          $e: \langle x,a \rangle \longrightarrow x^{'}$;
    \item $f_{U}: X \longrightarrow \mathbb{R^{+}} \cup\lbrace 0\rbrace$ uma função utilidade tal que
          $f_{U}(x)$ mede a utilidade do estado $x \in X$ um entre estados do mundo dado os estados;
    \item $r_i: \langle A_i, X_{ob}^{i}\rangle \longrightarrow a_i^{'}$ o modelo de reação dos robôs
          inimigos dado $X_{ob}^{i}$;
    \item $AB =\lbrace V \subset X, E \subset A\rbrace$ uma árvore de busca;
    \item $e_b: \langle X_{ob}^{i}, e, f_{U}, r_i, AB\rangle \longrightarrow AB^{'}$ uma estratégia de busca.

  \end{description}

  Então, um time $T$ é definido por:
  \[
    T: \langle A, X_{ob}^{i}, e, e_b, r_i \rangle \longrightarrow a_c^{i+1}
  \]
  Assim, utilizando-se de $e$, $T$ simula várias sequência de ações $a$ dado $X_{ob}^{i}$,
  gerando  a partir de $f_{U}$, $e_b$ e.
\end{defi}

\begin{defi}[Partida]
  Dado dois times $T_1$ e $T_2$. Uma partida $p$ é definida por:

  \[
   p = \lbrace T_1, T_2, \Delta t, \delta t, \langle Ref^{0}, X_{ob}^{0}, A_1^{0}, A_2^{0}\rangle, 
    ..., \langle Ref^{N}, X_{ob}^{N}, A_1^{N}, A_2^{N} \rangle \rbrace
 \]

  uma sequência , em que:
  \begin{description}
    \item $\Delta t$ é o tempo de duração da partida;
    \item $\delta t$ é o tempo médio entre cada frame enviado pela \textit{SSL-Vision} ao longo de $\Delta t$;
    \item $N \approxeq \frac{\Delta t}{\delta t}$ é número total de frames enviados pela \textit{SSL-Vision}
  ao longo de $\Delta t$;
    \item $Ref^{i}$ são os comandos enviados pelo módulo \textit{Referee-Box} no instante $i$;
    \item $X_{ob}^{i}$ são os dados enviados pelo módulo \textit{SSL-Vision} no instante $i$;
    \item $A_1^{i}$ são as ações executadas por $T_1$ no instante $i$;
    \item $A_2^{i}$ são as ações executadas por $T_2$ no instante $i$.
  \end{description}
\end{defi}

\begin{defi}[Logs]
  Dada uma partida $p$. O $log$ de $p$ é definidor por:

  \[
    log(p) = \lbrace p.\langle Ref^{0}, X_{ob}^{0}\rangle, ..., p.\langle Ref^{N}, X_{ob}^{N}\rangle \rbrace
  \]
\end{defi}

%\section{Enunciado do Problema}
\section{Escopo da Pesquisa}

A partir das definições apresentadas, pode-se enunciar o problema 
abordado por este trabalho. Deseja-se prever $x_{ob}^{i+1}$ dado
$X_{ob}^{i}$. Uma abordagem que será seguida é considerar
o problema como sendo uma tarefa de $classificação$ de KDD\@.
Resta então definir as classes e os atributos de acordo com as definições
apresentadas anteriormente. Assim, a seção a seguir descreve uma maneira
de se modelar essas classes e atributos.

\section{Modelagem por Classificação}

O problema a ser resolvido pode ser
encarado como um problema de classificação temporal. O sistema deve mapear uma
sequência temporal de observações para um conjunto de regras associadas a
subconjuntos de robôs. Essas regras definem o comportamento dos robôs. Assim,
para prever o próximo movimento do adversário, e com isso determinar
$x_{ob}^{i+1}$, pode-se descobrir quais são os robôs do time adversário que
estão bloqueando, quais estão marcando o atacante, quais estão esperando para
receber um passe dentre outras categorias comuns no futebol. A
tabela~\ref{regras} apresenta uma lista de regras comuns em futebol de robôs.

Outro problema relevante é definir o comportamento dos robôs a partir do estado
atual do jogo e das regras associadas aos robôs. Entretanto, devido ao tempo
destinado a esta pesquisa, será modelado somente uma solução para o mapeamento das
observações às regras. Também, o módulo de inteligência necessita dos dados
desse mapeamento para poder prever de maneira coerente estados futuros do jogo. Uma
solução ingênua é utilizar os modelos de táticas e \textit{skills} do próprio time.
Isso não representa a realidade, pois é pouco provável que dois times desenvolvam
todos os algoritmos que controlam os robôs de tal maneira que eles tenham o mesmo
resultado.

\subsection{Regras}

Em princípio cada equipe utiliza seu próprio conjunto de regra, que correspondem
às classes citadas anteriormente.
Na prática, entretanto, devido aos trabalhos publicados na comunidade científica
relacionados ao problema de futebol de robôs, muitas regras são compartilhadas
entre os times. Outra razão plausível é que muitas heurísticas e regras mimicam
as regras dos times de futebol jogados por humanos. Exemplos de regras típicas são
apresentados na tabela~\ref{regras}.

\begin{table}
  \begin{center}
    \begin{tabular}{|c|c|}
      \hline
      Regra                  & Descrição \\
      \hline
      \textit{Kick Off}      & Posicionamento antes do\\
                             & início do jogo\\
      \hline
      Marcar Oponente        & Defesa corpo-a-corpo\\
                             & contra um robô particular\\
      \hline
      Posicionar Para Passe  & Cria aberturas para passes\\
      \hline
      Receber Chute Alto     & Recebe um passe executado\\
                             & através do dispositivo de\\
      & chute alto\\
      \hline
      Posicionar             & Posicionamento no campo para\\
                             & receber um passe\\
      \hline
      \textit{Penalty Kick}  & Executa um chute a gol\\
      \hline
      Barreira               & Forma uma barreira com os \\
                             & jogadores do mesmo time \\
                             & para bloquear chutes\\
      \hline
      \textit{Set-Play-Kick} & Uma \textit{play} coordenada onde\\
                             & um robô passa a bola à outro que \\
                             & chuta de primeira em direção ao gol.\\
      \hline
      Atacante               & Regra de ataque primária\\
      \hline
      Defesa circular        & Defende o gol á um raio fixo\\
      \hline
    \end{tabular}
    \caption{Lista de regras comuns entre as equipes \cite{vail2008crf}}
  \label{regras}
  \end{center}
\end{table}

\subsection{Características}

Comumente informação é adicionada ao modelo através do pré-processamento dos
dados coletados nos sensores. Essas informações são chamadas de características.
Características são funções dos dados coletados nos sensores. Elas são utilizadas
para adicionar informação do domínio ao modelo.

As características tomam a forma:

\begin{gather}
f_i(t,Y,X)
\end{gather}

Os protótipos de características típicos do domínio do futebol de robôs,
conforme \cite{vail2008crf}, são:

\begin{gather}
f_i(t,Y,X)=I(y_t == regra)\\
f_i(t,Y,X)=I(y_{t-1} == regra_1)I(y_t == regra_2)\\
f_i(t,Y,X)=I(y_t == regra).g(t,X)
\end{gather}

Onde a função $I$ só retorna um valor não nulo se seu argumento for verdadeiro e
$g$ retorna um valor real.

A partir desse protótipos, associando cada valor aos robôs presentes
no campo, várias funções características podem ser instanciadas.
Tipicamente chega-se à ordem de 100 mil instâncias \cite{vail2008crf}. Faz-se necessário
selecionar quais funções efetivamente são relevantes para o problema.

\subsection{Seleção de Características}

Apesar de incorporar informação ao modelo, a inclusão de muitas características
pode fazer com que o modelo incorpore informações particulares dos dados de
treinamento. Isso é chamado de \textit{overfitting}. Esse fenômeno reduz a precisão
do modelo final por incorporar dados particulares do conjunto de dados de
treinamento. Devido a isso é necessário selecionar quais características devem
ser incorporadas ao modelo final.

A seleção de características reduz o tamanho do modelo, bem como o custo
computacional da classificação. Isso permite que as regras sejam reconhecidas
em tempo viável para serem utilizadas pelo módulo da inteligência durante o
jogo.

