\section{Definições}

\begin{defi}[Corpo Rígido]
  Um corpo rígido $r$ é definido por dois subconjuntos disjuntos
  de parâmetros $r= \langle \hat{r}, \bar{r} \rangle$ onde:
  \begin{itemize}
    \item $\hat{r} = \langle \alpha, \beta, \gamma, \omega \rangle$,
    que são os parâmetros de estado mutaveis, respectivamente:
    posição ($\mathbb{R} ^{3}$), orientação($\mathbb{R} ^{3}$),
    velocidade linear($\mathbb{R} ^{3}$), velocidade angular
    ($\mathbb{R} ^{3}$)
    
    \item $\bar{r} :$ parâmetros imutaveis corpo que descrevem sua
    natureza fixa e que permanece constante ao longo do curso de 
    planejamento.
  \end{itemize}
  
  Exemplos de parâmetros considerados nesta modelagem imutaveis são:
  coeficiente de atrito estático e dinâmico, descrição $3D$ do corpo
  (por exemplo, através de um conjunto de primitívas $3D$), centro de
  massa no referencial do corpo, coeficiente de restituição,
  coeficiente de amortecimento linear e angular. Note que a matrix de
  rotação $R\in\mathbb{R}^{3\times 3}$) gerada a partir da rotação
  em torno de um vetor unitário direção $d\in\mathbb{R}^{3}$  de
  $\theta \in \mathbb{R}$ radianos satisfaz a equação
  $R = exp\left( \beta \right)$ (onde $\beta = d. \theta \in \mathbb{R} ^{3}$)
  \cite{math2robotics}.
\end{defi}

\begin{defi}[Bola]\label{def:bola}
  Bola é um corpo rígido $\hat{b}$ no qual somente o parâmetro
  $\hat{b}.\alpha$ é observável. De acordo com acordo com a
  aquitetura considerada para a SSL descrita na figura
  \ref{arquitetura_ssl}, tem-se que a partir de uma sequênicia
  de quadros é possível obter um valor estimado para o parâmetro
  $\hat{b}.\gamma$ a partir do intervalo entre os dados recebidos
  da \textit{SSL-Vision} e da equação $ \gamma \approxeq 
  \frac{\delta \alpha}{\delta t} $. Entretanto, uma vez que o
  parâmetro $\hat{b}.\beta$, não se pode estimar o valor de
  $\hat{b}.\omega$.
\end{defi}

\begin{defi}[Robô]
  Robô $rob$ é um conjunto de sistemas compostos de corpos rígidos,
  \textit{hardware} e \textit{firmware}. São eles:
  
  \begin{itemize}
    \item Drible: imprime um torque a bola
    \item Chute baixo: imprime uma força a bola $\hat{b}$
          e possivelmente um torque, com o objetivo de
          alterar as componentes $\langle x,y \rangle$
          o parâmetro $\hat{b}.\gamma$ 
    \item Chute alto: Chute baixo: imprime uma força a bola $\hat{b}$
          e possivelmente um torque, com o objetivo de
          alterar as componentes $\langle x,y,z \rangle$
          o parâmetro $\hat{b}.\gamma$, com $\hat{b}.\gamma_z \neq 0$
    \item Receptor: rebe comandos enviados pelo sistema de
          transmissão de seu respectivo time
    \item Sistema de movimentação: imprime uma força e um torque
          o centro de massa global do $rob$
  \end{itemize}  
  
  Através desses sistemas $rob$ pode executar um conjunto de ações $A_{rob}$.
  
  Apesar de o modelo descrito acima abranger a mairia dos
  robôs utilizados atualmente por equipes da SSL, é importante
  ressaltar que o robô pode ter um conjunto de sensores que
  poderiam coletar informações adicionais ás transmitidas pela
  \textit{SSL-Vision} juntamente com um sistema de transmissão
  para envia-las ao software do seu respectivo time. Isso é
  interessante, pois, conforme observado na definição \ref{def:bola},
  o parâmetro $\hat{b}.\beta$ não é observavel. Como o sitema de
  drible impoem um torque a bola, através de um sensor é possível
  estimar o valor de $\hat{b}.\omega$. Sem esse sensor não é possível
  prever com exatidão a tragetória da bola através do módulo de
  simulação apenas com os dados da \textit{SSL-Vision}.
  
\end{defi}

\begin{defi}[Time]
  Seja $X$ o espaço de estado de todos os corpos rígidos envolvidos na partida,
  $x_{init} \in X$, seja $X_{goal}\subset X$ o conjunto de estados, e seja $A = A_{rob 1} 
  \cup ... \cup A_{rob n}$ o conjunto das ações possíveis. Seja $e$ a função de transição
  de estado que pode aplicar uma ação $a\in A$ em um estado particular
  $x \in X$ e compute o estado seguinte $x^{'} \in X$, i.e.: 
  $e: \langle x,a \rangle \longrightarrow x^{'}$.
  Um time $T$ é um sistema multi-agente composto pelos robôs 
  $\langle rob_1, ..., rob_n \rangle$ e baseado em uma função utilidade
  $f_{U}: X \longrightarrow \mathbb{R^{+}} \cup\lbrace 0\rbrace$.
  $f_{U}$ mede suas preferências entre estados do mundo. Assim, utilizando-se
  de $e$, $T$ simula várias seguencia de ações $a$ e escolhe a ação $a^{*}$ que
  leva á melhor utilidade avalida por $f_{U}$. 
\end{defi}