\section{Enunciado do Problema}

A partir das definições apresentadas, pode-se enunciar o problema 
abordado por este trabalho. Deseja-se prever $X_{ob}^{i+1}$ dado
$X_{ob}^{i}$. Uma abordagem que será seguida é considerar
o problema como sendo uma tarefa de $classificação$ de KDD. Conforme
\cite{passos2005datamining} tem-se que o problema de classificação:

\begin{description}
  \item \textit{"... Consistem em descobrir uma função que mapeie um conjunto de
  registros em um conjunto de rótulos categóricos predefinidos, denominados
  classes. Uma vez descoberta, tal função pode ser aplicada a novos registros
  de forma a prever a classe em que tais registros se enquadram. ...."}
\end{description}

De acordo com \cite{doringo2004ant}, tem-se uma definição mais formal seria:
\begin{description}
  \item \textit{"Dado conjunto de atributos $A = \lbrace a_1 ; . . . ; a_n\rbrace$ 
  (o domínio de cada atributo $a_i$ sendo o conjunto $V = \lbrace v_{i1} , . . ., v_{if_i}\rbrace$ de $f_i$ valores),
  um conjunto de $l$ classes $B = \lbrace b_1 , . . . , b_l \rbrace$, e um conjunto de treinamento 
  $TS = \lbrace ts_1 , . . . , ts_h \rbrace $, onde cada $ts_i$ é um caso,
  a tarefa é aprender um conjunto de regras \textbf{SE-ENTÃO}, cada regra tomando a forma
  \textbf{SE} $\langle term_1 \wedge term2 \wedge...\rangle $ \textbf{ENTÃO} $\langle b_i \rangle$. 
  A parte \textbf{SE} da regra é chamada antecedente, a parte \textbf{ENTÃO} é chamada de
  consequente e da a classe predita pela regra. Cada termo no antecedebte é uma tripla ordenada
  $(a, o, v)$ onde $v$ é um valor no domínio de atributos $a$, e $o$ um operador (que relaciona $a$ e $v$)."}
\end{description}

Resta então definir as classes e os atributos de acordo com as definições apresentadas anteriormente.