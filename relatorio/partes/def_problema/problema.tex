\section{Enunciado do Problema}

A partir das definições apresentadas, pode-se enunciar o problema 
abordado por este trabalho. Deseja-se prever $X_{ob}^{i+1}$ dado
$X_{ob}^{i}$. Uma abordagem que será seguida é considerar
o problema como sendo uma tarefa de $classificação$ de KDD. Conforme
\cite{passos2005datamining} tem-se que o problema de classificação:

\begin{description}
  \item \textit{"... Consistem em descobrir uma função que mapeie um conjunto de
  registros em um conjunto de rótulos categóricos predefinidos, denominados
  classes. Uma vez descoberta, tal função pode ser aplicada a novos registros
  de forma a prever a classe em que tais registros se enquadram. ...."}
\end{description}

Resta então definir as classes e os atributos de acordo com as definições
apresentadas anteriormente. Assim, a secção a seguir descreve uma maneira
de se modelar essas classes e atributos.

\section{Modelagem por Classificação}

O problema a ser resolvido
encarado como um problema de classificação temporal. O sistema deve mapear uma
sequência temporal de observações para um conjunto de regras associadas a
subconjuntos de robôs. Essas regras definem o comportamento dos robôs.

Outro problema relevante é definir o comportamento dos robôs a partir do estado
atual do jogo e das regras associadas aos robôs. Entretanto, devido ao tempo
destinado a esta pesquisa será modelado somente uma solução para o mapeamento das
observações às regras. Também, o módulo de inteligência necessita dos dados
desse mapeamento para poder prever de maneira coerente estados futuros do jogo. Uma
solução ingênua é utilizar os modelos de táticas e \textit{skills} do próprio time.
Isso não representa a realidade, pois é pouco provável que dois times desenvolvam
todos os algoritmos que controlam os robôs de tal maneira que eles tenham o mesmo
resultado.

\subsection{Regras}

Em princípio cada equipe utiliza seu próprio conjunto de regra, que correspondem
às classes citadas anteriormente.
Na prática, entretanto, devido aos trabalhos publicados na comunidade científica
relacionados ao problema de futebol de robôs, muitas regras são compartilhadas
entre os times. Outra razão plausível é que muitas heurísticas e regras mimicam
as regras dos times de futebol jogados por humanos. Exemplos de regras típicas são
apresentados na tabela~\ref{regras}.

\begin{table}
  \begin{center}
    \begin{tabular}{|c|c|}
      \hline
      Regra                  & Descrição \\
      \hline
      \textit{Kick Off}      & Posicionamento antes do\\
                             & início do jogo\\
      \hline
      Marcar Oponente        & Defesa corpo-a-corpo\\
                             & contra um robô particular\\
      \hline
      Posicionar Para Passe  & Cria aberturas para passes\\
      \hline
      Receber Chute Alto     & Recebe um passe executado\\
                             & através do dispositivo de\\
      & chute alto\\
      \hline
      Posicionar             & Posicionamento no campo para\\
                             & receber um passe\\
      \hline
      \textit{Penalty Kick}  & Executa um chute a gol\\
      \hline
      Barreira               & Forma uma barreira com os \\
                             & jogadores do mesmo time \\
                             & para bloquear chutes\\
      \hline
      \textit{Set-Play-Kick} & Uma \textit{play} coordenada onde\\
                             & um robô passa a bola à outro que \\
                             & chuta de primeira em direção ao gol.\\
      \hline
      Atacante               & Regra de ataque primária\\
      \hline
      Defesa circular        & Defende o gol á um raio fixo\\
      \hline
    \end{tabular}
  \caption{Lista de regras comuns entre as equipes}
  \label{regras}
  \end{center}
\end{table}

\subsection{Características}

Comumente informação é adicionada ao modelo através do pré-processamento dos
dados coletados nos sensores. Essas informações são chamadas de características.
Características são funções dos dados coletados nos sensores. Elas são utilizadas
para adicionar informação do domínio ao modelo.

As características tomam a forma:

\begin{centering}
\begin{equation}
f_i(t,Y,X)
\end{equation}
\end{centering}

Os protótipos de características típicos do domínio do futebol de robôs são:

\begin{centering}
\begin{eqnarray}
f_i(t,Y,X)=I(y_t == regra)\\
f_i(t,Y,X)=I(y_{t-1} == regra_1)I(y_t == regra_2)\\
f_i(t,Y,X)=I(y_t == regra).g(t,X)\\
\end{eqnarray}
\end{centering}

Onde a função $I$ só retorna um valor não nulo se seu argumento for verdadeiro e
$g$ retorna um valor real.

A partir desse protótipos, associando cada valor aos robôs presentes
no campo, várias funções características podem ser instanciadas.
Tipicamente chega-se à ordem de 100 mil instâncias. Faz-se necessário
selecionar quais funções efetivamente são relevantes para o problema.

\subsection{Seleção de Características}

Apesar de incorporar informação ao modelo, a inclusão de muitas características
pode fazer com que o modelo incorpore informações particulares dos dados de
treinamento. Isso é chamado de \textit{overfitting}. Esse fenômeno reduz a precisão
do modelo final por incorporar dados particulares do conjunto de dados de
treinamento. Devido a isso é necessário selecionar quais características devem
ser incorporadas ao modelo final.

A seleção de características reduz o tamanho do modelo, bem como o custo
computacional da classificação. Isso permite que as regras sejam reconhecidas
em tempo viável para serem utilizadas pelo módulo da inteligência durante o
jogo.

A seguir são apresentados os métodos inicialmente propostos
no início da pesquisa. Após isso, são discutidas abordagens envolvendo
esses métodos.
