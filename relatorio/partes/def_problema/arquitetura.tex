\subsection{Arquitetura}

A arquitetura a ser considerada é baseada em \cite{felixnavarro}.
A \textit{RoboCup Small Size League} (SSL) envolve problemas de diversas áreas
da engenharia. Logo, com o objetivo de facilitar a compreensão do
problema, a arquitetura a ser considerada é apresentada na figura
\ref{arquitetura_ssl}. Essa arquitetura é composta pelos seguintes
sistemas:

\begin{itemize}
  \item Câmeras Visão: conjunto de câmeras \textit{firewire} que captura as imagens do
        campo e as envia para a SSL-Vision;
  \item Comunicação: módulo responsável por receber os parâmetros
        dos motores, drible, chute baixo e chute alto e enviar o comando via
        ondas de rádio para os robôs;
  \item Execução: módulo responsável por realizar a tomada de decisões
        em baixo nível de quais ações os robôs devem realizar a partir
        da decisão tomada pelo módulo de inteligência;
  \item Inteligência: módulo responsável por realizar a tomada de
        decisão em alto nível de quais ações os robôs devem realizar
        tendo auxílio de um módulo de Simulação;
  \item Mundo Real: campo de futebol real, onde os times 1 e 2 interagem
        através de seus respectivos robôs
  \item Referee-Box: \textit{software} padronizado pela Robocup para que as
        regras da competição sejam cumpridas sem que haja intervenção
        humana excessiva durante uma partida;
  \item Simulação: módulo do \textit{software} do time responsável por simular
        o ambiente da partida, tendo como entrada os parâmetros do mundo
        real;
  \item \textit{Software} Time 1/2: \textit{software} do time 1/2;
  \item SSL-Vision: \textit{software} padronizado pela Robocup que permite a
        integração com um conjunto de câmeras \textit{firewire} que capturam
        imagens do campo e as processa, extraindo informações sobre os objetos na
        imagem;
  \item Time 1/2: time de robôs que executa os comandos recebidos pelo
        sistema de transmissão do time 1/2;
  \item Transmissão 1/2: sistema de transmissão do time 1/2;
  \item World Model: módulo responsável por modelar o mundo e dar
        confiabilidade aos dados que serão enviados ao módulo de
        Inteligência e são oriundos do Referee-Box e SSL-Vision.
\end{itemize}

\begin{landscape}
  \begin{figure}[thpb]
    \centering
    \includegraphics[width=20cm]{imgs/arquitetura_ssl}
    \caption{Arquitetura básica da SSL}
    \label{arquitetura_ssl}
  \end{figure}
\end{landscape}

Apesar de modelar a maioria dos sistemas empregados atualmente na
SSL, existem alguns problemas na arquitetura descrita anteriormente. Entretanto,
são necessárias algumas definições apresentadas nas próximas secções para que eles possam ser discutidos.
