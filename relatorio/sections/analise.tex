% vim: et sw=2 ts=2 sts=2

\section{Análise das Possíveis Abordagens}
\frame{
  \frametitle{Análise das Possíveis Abordagens}
  \begin{block}{}

    %\begin{enumerate}
    %  \item Lógica Fuzzy
    %  %\item CRF + ACO/SA
    %  \item Rede Neural
    %\end{enumerate}

 \begin{table}
   \begin{center}
     \begin{tabular}{|c|c|c|c|c|c|}
       \hline
       Critério          &  ACO & SA  & L. Fuzzy & ANN\\
       \hline
       Requer modelo     &  Sim & Sim &   Sim    & Não\\
       de classificação  &      &     &          &    \\
       \hline
       Estrutura de      &  Sim & Sim &   Sim    & Não\\
       fácil análise     &      &     &          &    \\
       \hline
       Degradação devido &  Sim & Não &   Não    & Não\\
       a discretização   &      &     &          &    \\
       \hline
       Fácil             &  Não & Não &   Não    & Sim\\
       pré-processamento &      &     &          &    \\
       \hline
       Fácil             &  Não & Não &   Não    & Sim\\
       pós-processamento &      &     &          &    \\
       \hline
       Característica    & Modelo   & Parte das & Parte das    & Somente    \\
       reutilizável      & treinado & regras    & regras       & topologia  \\
       \hline
     \end{tabular}
   \caption{Tabela comparativa dos métodos}
   \label{table:metodos}
   \end{center}
 \end{table}

  \end{block}
}
