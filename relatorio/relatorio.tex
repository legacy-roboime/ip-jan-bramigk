\documentclass[a4paper,12pt]{imethesis}
\usepackage[authoryear,round]{natbib}
\usepackage{indentfirst}
\usepackage[utf8]{inputenc}
%\usepackage[T1]{fontenc}                            % Fontes acentuados
\usepackage{amsmath}
\usepackage{amsfonts}
\usepackage{graphicx}
%\usepackage[table]{xcolor}
%\usepackage{breakcites}
%\usepackage{subfigure}
\usepackage{hyperref}
% pacote para escrever algoritmos
% http://www.cs.toronto.edu/~frank/Useful/algorithm2e.pdf
\usepackage[lined, boxed, portuguese, commentsnumbered]{algorithm2e}
\usepackage{cite}
\usepackage{nomencl} % Lista de Abreviaturas
\usepackage[usenames]{color}
\usepackage{float}
\usepackage{arydshln} %para fazer linhas pontilhadas na matriz

\setlength{\hoffset}{-1cm}
\setlength{\voffset}{-2cm}
\setlength{\textheight}{23cm}
\hyphenation{}

% Meta informações
% ================

\author{
  Bramigk, Victor\\
  \texttt{victorbramigk@gmail.com}
  \and
  Segre, Jan\\
  \texttt{jan@segre.in}
}
\title{Heurística Estática para Times Cooperativos de Robôs}
\date{2 de Outubro 2013}
\cityear{Rio de Janeiro}{2013}

%\abrevauthor{Wanous, c.}
\university{Instituto Militar de Engenharia}
\thesisapp{Iniciação à Pesquisa apresentada ao Curso de Graduação
em Engenharia de Computação do Instituto Militar de
Engenharia.}
%\patente{}
\cipcode{Z34r}

\chair{Paulo F. F. Rosa - Ph.D}{do IME}
\memberone{Maj Julio Cesar Duarte - D.Sc.}{do IME}
\membertwo{Ricardo Choren Noya - Ph.D}{do IME}
\numberofmembers{3}

%\renewcommand*\thesection{\arabic{section}}
%\linespread{1.5}

\begin{document}

% Elementos pré-textuais
% ======================
%
% (aka: de pouca utilidade prática)

\begin{titlepage}

\begin{center}
\large{MINISTÉRIO DA DEFESA} \\
\large{EXÉRCITO BRASILEIRO} \\
\large{DEPARTAMENTO DE CIÊNCIA E TECNOLOGIA}\\
\large{IME - INSTITUTO MILITAR DE ENGENHARIA} \\
\large{CURSO DE GRADUAÇÃO EM ENGENHARIA DA COMPUTAÇÃO} \\
\vspace{5 cm}
\large{CAMILA ANTONACCIO WANOUS} \\
\large{JOÃO LUIZ DO PRADO NETO} \\
\large{THIAGO DE PAULA VASCONCELOS} \\
\vspace{5 cm}
\large{CONECTIVIDADE ALGÉBRICA E SEUS AUTOVETORES NA CLASSE DAS ÁRVORES}\\
\vspace{5 cm}
Rio de Janeiro\\
2013
\end{center}

\end{titlepage}

%\begin{center}
%\textbf{MINISTÉRIO DA DEFESA}\\
%\textbf{EXÉRCITO BRASILEIRO}\\
%\textbf{DEPARTAMENTO DE CIÊNCIA E TECNOLOGIA}\\
%\textbf{INSTITUTO MILITAR DE ENGENHARIA}\\
%\textbf{Seção de Engenharia de Sistemas / SE 8}
%
%\vspace{2.5cm}
%
%\begin{large}
%\textbf{Proposta de Tema de Dissertação de Mestrado
%\\Curso: Mestrado em Sistemas e Computação}
%
%\vspace{1.5cm}
%
%\textbf{Sistema de Localização de Objetos para Apoio a um Assistente Robótico Móvel na Casa Inteligente}
%
%\vspace{1.5cm}
%
%\textbf{Aluno: Fulano}
%
%\vspace{1.5cm}
%
%\textbf{Orientador: Paulo F. F. Rosa, Ph.D}
%
%\end{large}
%
%\vspace{2cm}
%
%\begin{small}
%Data de Apresentação no SE/8:\\
%Rio de Janeiro, \today
%\end{small}
%
%\end{center}
%
%\pagebreak

\input{pre_textuais/folha_de_rosto}
\input{pre_textuais/ficha_catalografica}
\input{pre_textuais/nota_de_direitos_autorais} % deve ser o verso da folha de rosto de algum jeito
\input{pre_textuais/folha_de_aprovacao}

% begin opcionais
\input{pre_textuais/dedicatoria}
\input{pre_textuais/agradecimentos}
\input{pre_textuais/epigrafe}
% end opcionais

\section{Resumo}

O objetivo deste trabalho é prever como um time de futebol de robôs irá se comportar baseado
somente nas posições e orientações de um conjunto discreto. Para isso, foram estudados
os métodos da ACO(Ant Colony Optimization), SA(simulated Annealing), Algorítimo  Genético, Lógica Nebulosa
e Redes Neurais. A partir do estudo detalhado desse algoritmos definiu-se duas linhas principais de
ação para a solução do problema: uma baseada em Logica Nebulosa e a outra baseada em Redes Neurais. Após
um estudo mais aprofundado deseja-se implementar um processo de otimização em ambos os algorítimos para
que o resultado seja refinado.

\chapter*{Abstract}

The main objective of this work is predicting how a robot soccer team whill behave, based
on a set of positions and orientations of a discrete sample. For that, some heuristics were
studied: ACO (Ant Colony Optimization), SA (Simulated Annealing), GA (Genetic Algorithm),
and Neural Networks. Based on the detailed study of these algorithms two branches were defined
as candidate solutions: one based on the Neural Network heuristics, and the other based on
Fuzzy logic. After a deeper study on of these methods it's desirable to implement an algori-
thm which will refinethe result.


\input{pre_textuais/sumario}

% begin opcionais
\input{pre_textuais/listas}
% end opcionais

%\section{Resumo}

O objetivo deste trabalho é prever como um time de futebol de robôs irá se comportar baseado
somente nas posições e orientações de um conjunto discreto. Para isso, foram estudados
os métodos da ACO(Ant Colony Optimization), SA(simulated Annealing), Algorítimo  Genético, Lógica Nebulosa
e Redes Neurais. A partir do estudo detalhado desse algoritmos definiu-se duas linhas principais de
ação para a solução do problema: uma baseada em Logica Nebulosa e a outra baseada em Redes Neurais. Após
um estudo mais aprofundado deseja-se implementar um processo de otimização em ambos os algorítimos para
que o resultado seja refinado.

%\chapter*{Abstract}

The main objective of this work is predicting how a robot soccer team whill behave, based
on a set of positions and orientations of a discrete sample. For that, some heuristics were
studied: ACO (Ant Colony Optimization), SA (Simulated Annealing), GA (Genetic Algorithm),
and Neural Networks. Based on the detailed study of these algorithms two branches were defined
as candidate solutions: one based on the Neural Network heuristics, and the other based on
Fuzzy logic. After a deeper study on of these methods it's desirable to implement an algori-
thm which will refinethe result.


% Corpo do trabalho
% =================

% Introducao
% ----------

\chapter{Introdução}


As influências da robótica
são visíveis na indústria moderna. Ela tem viabilizado o desenvolvimento de peças
precisas, assim como um controle maior do processo. O uso de robôs confere aos
processos industriais precisão e repetibilidade maior que as adquiridas caso
fosse empregado um humano. Isso, além de reduzir o custo de retrabalhos, permite
que os desenvolvedores se preocupem mais com o processo em si. Assim, o emprego
de robôs confere um amadurecimento dos processos industrias, bem como indiretamente
permite que a sociedade concentre esforços em cargos intelectuais.

Com efeito, o emprego desses sistemas robóticos deixou evidente a necessidade do
desenvolvimento de teorias relacionadas as sistemas autônomos cooperativos. 

% Esses sistemas podem
% ser empregados para permitir que resgates sejam feitos de maneira eficiente. Isso
% evitaria que o pessoal altamente especializado empregado atualmente corra risco de vida.

Entretanto, projetar robôs autônomos para trabalharem juntos não é uma tarefa trivial. Essa
tarefa se complica quando um robô não tem um modelo bem definido dos outros robôs que atuarão em
conjunto. Um domínio de aplicação que envolve essa problemática é o futebol de robôs.
Nesse domínio é comum a distribuição de papéis dinamicamente entre os membros de cada
time. Entretanto, um time não tem conhecimento dos papéis atribuídos aos robôs
do time oponente. O planejamento de um time pode ser aprimorado através de um modelo
aproximado desse time oponente, pois levará em consideração a maneira como o
time reagirá às diversas ações possíveis.

\begin{figure}
  \includegraphics[width = \linewidth]{figuras/robocup2013}
  \caption{Imagem da SSL \textit{RoboCup} 2013 em Eindhoven, na Holanda}\label{fig:robocup2013}
\end{figure}

A ideia de robôs jogando futebol foi mencionada pela primeira vez pelo professor
Alan Mackworth (\textit{University of British Columbia}, Canadá) em um artigo intitulado
\textit{"On Seeing Robots"}, apresentado no \textit{Vision Interface 92} e posteriormente publicado em
um livro chamado \textit{Computer Vision: System, Theory and Applications}. Independentemente,
um grupo de pesquisadores japoneses organizou um \textit{Workshop} no \textit{Ground Challange
in Artificial Inteligence}, em Outubro de 1992, Tóquio, discutindo e propondo problemas que
representavam grandes desafios. Esse \textit{Workshop} os levou a sérias discussões sobre
usar um jogo de futebol para promover ciência e tecnologia. Estudos foram feitos para
analisar a viabilidade dessa ideia. Os resultados desses estudos mostram que
a ideia era viável, desejável e englobava diversas aplicações práticas. Em 1993, um
grupo de pesquisadores, incluindo Minoru Asada, Yasuo Kuniyoshi e Hiroaki Kitano,
lançaram uma competição de robótica chamada de Robot \textit{J-League} (fazendo uma analogia à
\textit{J-League}, nome da Liga Japonesa de Futebol Profissional). Em um mês, vários
pesquisadores já se pronunciavam dizendo que a iniciativa deveria ser estendida ao
âmbito internacional. Surgia então, a \textit{Robot World Cup Initiative} (RoboCup).

RoboCup é uma competição destinada a desenvolver os estudos na área de robótica e
Inteligência Artificial (IA) por meio de uma competição amigável. Além disso, ela tem
como objetivo, até 2050, desenvolver uma equipe de robôs humanoides totalmente
autônomos capazes de derrotar a equipe campeã mundial de futebol humano. A competição
possui várias modalidades. Neste trabalho, será analisada a \textit{Small Size Robot League} (SSL),
também conhecida como F180. De acordo com as regras da SSL, as equipes devem ser
compostas por 6 robôs, sendo um deles o goleiro, que deve ser
designado antes do início do jogo. Durante o jogo, nenhuma interferência humana é
permitida com o sistema de controle dos robôs. É fornecido aos times um sistema de
visão global e esses controlam seus robôs com máquinas próprias. O sistema de controle
dos robôs geralmente é externo e recebe os dados de um conjunto de duas câmeras
localizadas acima do campo. Esse sistema de controle processa os dados, determina qual comando deve ser executado por
cada robô e envia este comando através de ondas de rádio aos robôs. Embora seja
permitido que as equipes utilizem sistemas próprios de visão, a maioria das
equipes utiliza a visão centralizada. A figura~\ref{fig:robocup2013} mostra uma
imagem da SSL Robocup 2013, da qual a RoboIME (Equipe de Futebol de Robôs do
Laboratório de Robótica do IME) participou.

\section{Motivação}

O futebol de robôs, problema padrão de investigação internacional, reúne grande parte
dos desafios presentes em problemas do mundo real a serem resolvidos em tempo real.
As soluções encontradas para o futebol de robôs podem ser estendidas, possibilitando
o uso da robótica em locais de difícil acesso para humanos, ambientes insalubres e
situações de risco de vida iminente.

Há diversas novas áreas de aplicação da robótica, tais como exploração espacial e submarina,
navegação em ambientes inóspitos e perigosos, serviço de assistência médica
e cirúrgica, além do setor de entretenimento. Essas áreas podem ser beneficiadas com o
desenvolvimento de sistemas
multi robôs. Nestes domínios de aplicação, sistemas de multi robôs deparam-se sempre
com tarefas muito difíceis de serem efetuadas por um único robô. Um time de robôs pode
prover redundância e contribuir cooperativamente para resolver o problema em questão.
Com efeito, eles podem resolver o problema de maneira mais confiável, mais rápida e
mais econômica, quando comparado com o desempenho de um único robô.

% TODO Incluir imagem do robô
%\begin{figure}
%  \includegraphics[width = \pagewidth]{}
%\end{figure}

% Devido à grande complexidade do problema de interação com humanos, faz-se necessário
% que os robôs sejam dotados de uma capacidade de aprendizado para facilitar a interação
% desses com o mundo real. Isso é relevante tanto para aplicações industriais, quanto para
% aplicações em resgates e militares. Isso diminui a necessidade de modelagem
% exata dos ambientes em que os sistemas robóticos serão introduzidos e permite que
% a adaptação a ambientes complexos seja realizada através da exposição destes sistemas
% às possíveis situações de trabalho. Por meio da incorporação do sistema de
% aprendizagem, situações não consideradas podem ser incorporadas ao algoritmo de
% controle dos robôs dinamicamente. Isso permitiria que esses reagissem de maneira mais
% eficiente em futuras situações semelhantes.

Devido a alta complexidade de um jogo de futebol de robôs juntamente com a
necessidade de se refazer constantemente o planejamento das ações a serem tomadas
pelos robôs, é extremamente desejável evitar expandir nós da árvore de
planejamento que não representem o modelo de reação do adversário. Isso
aumentaria a eficiência no uso dos recursos de planejamento, permitindo
antecipar mais jogadas em um menor intervalo de tempo. Logo, faz-se necessário
pesquisar heurísticas que façam isso da maneira mais eficiente e controlada
possível.

\section{Objetivo}

Este trabalho objetiva estudar os algoritmos da Otimização da Colõnia de Formigas (ACO),
Recozimento Simulado (SA), Algoritmo Genético (GA), Rede Neural (ANN) assim como métodos
da logica fuzzy. Isso objetiva analisar se e como a inteligência
artificial de um time de futebol de robôs pode ser modelada utilizando esses
algoritmos e métodos com base nas informações contidas nos $logs$ (definido a seguir)
de um jogo da SSL da RoboCup. Isso tem o objetivo de permitir que agentes controláveis
possam reagir de maneira eficiente às ações dos agentes do time adversário, que
não são controláveis. A pesquisa se propõe a realizar esse aprendizado dos
agentes adversários baseado em gravações coletadas dos pacotes da
\textit{SSL-Vision} e do \textit{Referee-Box} durante jogos, também conhecidas
como \textit{logs}, para posteriormente serem incorporados ao sistema de inteligência da
equipe de futebol de robôs do Laboratório de Robótica, denominada RoboIME\@. Apesar
de existirem até a data deste trabalho vários artigos publicados relacionados a
SSL, os autores sentiram a necessidade de modelarem de uma maneria mais rigorosa
certos conceitos que surgiram ao longo da pesquisa. Isso com o objetivo de se
compreender melhor o problema a ser resolvido.
% NOTE: acho que não é o caso disso:
%Se possível, deseja-se que essa modelagem/aprendizagem seja feita dinamicamente durante a partida da SSL.

\section{Justificativa}% TODO Maj Duarte colocou um certo nessa encestação

Um método concreto que possa prever o comportamento de agentes inteligentes de um jogo de
futebol de robôs permite com que seja possível prever o comportamento de um time adversário.
Com tal mecanismo é possível melhorar a Inteligência Artificial em uso pela RoboIME
para tomar decisões que levem a resultados melhores e, por consequência, ganhar mais partidas.
Outras equipes participantes da SSL já utilizam mecanismos de predição do adversário.
Portanto, é de grande importância desenvolver também tal mecanismo para acompanhar a evolução
das tecnologias envolvidas.

\section{Metodologia}

Para atingir os objetivos propostos, será seguida a seguinte metodologia:
Inicialmente o problema a ser investigado será definido formalmente,
utilizando definições e teorias apropriadas.

% Posteriormente, a bibliografia é revisada e são evidenciados os métodos comumente
% utilizados para a abordagem do problema mais geral de classificação, bem como são
% analisados trabalhos aplicados especificamente à SSL.

A seguir, são analisadas as heurísticas levantadas durante a
revisão da bibliografia. Cada uma delas é descrita. Após descrição sumária
de cada heurística, são apresentados os respectivos exemplos de cada aplicação.

Então, são apresentadas abordagens envolvendo as heurísticas introduzidas
anteriormente. Cada abordagem relaciona, no mínimo, uma dessas heurísticas.
É apresentada uma tabela comparativa que evidencia as principais diferenças
entre cada abordagem.

% NOTE: não é bem verdade isso:
%Para validar os estudos desenvolvidos, o projeto de um software
%que analise os dados dos \textit{logs} disponíveis no site da SSL é apresentado.

\section{Estrutura do Trabalho}

No capítulo~\ref{cap:def_problema}, o problema a ser resolvido é definido.
No capítulo~\ref{cap:heuristicas}, são apresentadas as
heurísticas comumente utilizadas em problemas de classificação.
No capítulo~\ref{cap:anal_abordagens}, são descritas possíveis abordagens a serem
seguidas para resolução do problema.
No capítulo~\ref{cap:conclusao}, são apresentadas as principais conclusões
atingidas neste trabalho.


% Desenvolvimento
% ---------------

\chapter{Análise das Possíveis Abordagens}\label{cap:anal_abordagens}

Uma característica desejada no método a ser empregado para prever próximo
movimento do adversário é não requerer um modelo de classificação. Isso para
deixa a implementação muito complexa, o que dificulta a manutenibilidade do
software final.

Também é desejado que tenha uma estrutura de fácil análise. Essa característica
tem o objetivo de incorporar informação heurística no modelo de modo a facilitar
a modelagem. Apesar de ser desejáda, inicialmente essa característica é menos
importante que a anterior. Entretanto, no longo prazo ela é mais relevante.
Portanto, como o objetivo de se adquirir uma experiencia, essa característica é
menos importante que a anterior.

Um outro problema identificado na análise anterior é a degradação do modelo
devido a discretização. Isso reduz a presição do modelo, então é importante que
o modelo não seja degradado com a discretização.

O pré-processamento é uma tarefa importate do processo de KDD. É desejável que
essa etapa seja a mais fácil possível, para poder agilizar a automação dessa
etapa. No caso dos algorítimos que necessitam de modelo de classificação, fica
evidente que eles necessitam de de um pré-processamento maior, pois esses
modelos necessítam de um conjunto de dados classificados para treinamento.
Isso implica que mais informação que os \textit{logs} será nessária, tornado
esses algorítmos menos adequandos que a ANN.

A etapa de pós-processamento é o processamento que será necessário para que os
dados do algorítmos determinem $x_{ob}^{i+1}$. Essa também é uma etapa na qual
os algorítmos que necessitam de um modelo de classificação são inadequados, pois
é necessario desenvolver outra camada de software para processar a classificação
de cada robô de modo a determinar o próximo movinto dos robôs do time
adversário.

Também é desejável, mas não mandatório, que o modelo treinado possa ser
modificado e reutilizado para modelar diferentes IAs. Isso visa
agilizar a modelagem durante as partidas da SSL da Robocup, pois no intervalo
de um jogo é permitido aos times modificar suas respectivas IAs.

\subsection{Comparação dos Métodos}

Os algorítimos abordados se enquadram em duas classes: algorítmos que são
completos e que necessitam de uma estrutura de classificação que são o ACO, SA,
AG e a algorítmos que são não necessitam, que é o caso da ANN\@. O caso da
lógica fuzzy será discutido na próxima seção com mais detalhes. A
tabela~\ref{table:metodos} apresenta as principais diferenças dos métodos
apresentados anteriormente. Conforme pode ser visto nessa tabela, a rede neural
é a que se mais adequa às caraterísticas desejádas.


\section{Rede Neural}\label{cap:abordagem_rede_neural}

Essa aborgem consiste em usar uma Rede Neural que tem como entrada o estado do
jogo (todas as posições, orientações, velocidades e o comando do juiz) e como
saída o estado do time adversário (todas as posições, orientações e velocidades
dos robôs do time adversário), que visa prever as ações imediatas do adversário.
Essa rede deve ser treinada para prever um time específico usando os
\textit{logs} das partidas do torneio de 2013 da \textit{RoboCup}.

% vim: tw=80


 \begin{table}
   \begin{center}
     \begin{tabular}{|c|c|c|c|c|c|}
       \hline
                         &          &           &              &             &          \\
       Critério          &  ACO     &    SA     & Lógica Fuzzy & Rede Neural & Desejado \\
                         &          &           &              &             &          \\
       \hline                                                                           
                         &          &           &              &             &          \\
       Requer modelo     &   Sim    &    Sim    &     Sim      &    Não      &   Não    \\
       de classificação  &          &           &              &             &          \\
                         &          &           &              &             &          \\
       \hline                                                                           
                         &          &           &              &             &          \\
       Estrutura de      &   Sim    &    Sim    &     Sim      &    Não      &   Sim    \\
       fácil análise     &          &           &              &             &          \\
                         &          &           &              &             &          \\
       \hline                                                                           
                         &          &           &              &             &          \\
       Degradação devido &   Sim    &    Não    &     Não      &    Não      &   Não    \\
       a discretização   &          &           &              &             &          \\
                         &          &           &              &             &          \\
       \hline                                                                           
                         &          &           &              &             &          \\
       Fácil             &   Não    &    Não    &     Não      &    Sim      &   Sim    \\
       pré-processamento &          &           &              &             &          \\
                         &          &           &              &             &          \\
       \hline                                                                           
                         &          &           &              &             &          \\
       Fácil             &   Não    &    Não    &     Não      &    Sim      &   Sim    \\
       pós-processamento &          &           &              &             &          \\
                         &          &           &              &             &          \\
       \hline                                                                           
                         &          &           &              &             &          \\
       Característica    & Modelo   & Parte das & Parte das    & Somente     & Modelo   \\
       reutilizável      & treinado & regras    & regras       & topologia   & treinado \\
                         &          &           &              &             &          \\
       \hline
     \end{tabular}
   \caption{Tabela comparativa dos métodos}
   \label{table:metodos}
   \end{center}
 \end{table}

\section{Lógica Fuzzy}

O método da Lógica Fuzzy, conforme exposto anteriormente, necessita que um
conjunto de regras seja definido. Um exemplo dessas regras foi apresentado na
seção \ref{sec_regras}. Essas regras poder ser geradas a partir de uma
análise mais detalhada do problema, mas também podem ser obtidas utilizando
algum algorítimo de aprendizagem de regras. Para aplicar esse método ao problema
analisado neste trabalho é necessário definir as regras e as distribuições das
variáveis.

A vantagem dos conjuntos difusos é que eles tornam o modelo mais robusto. A
lógica fuzzy tenta melhorar a classificação e os sistemas de decisão.

A principal desvantagem deste método é a modelagem necessária para encaixar os
conceitos descritos acima. Isso, pois o conceito de conjuntos nebulosos ainda
estão em desenvolvimento para o problema abordado neste trabalho. Essa modelagem
não é imediata, pois o problema é de classificação temporal. Não basta que as
características do ambiente sejam associadas aos conjuntos nebulosos de
características. É necessário que regras sejam especificadas estática ou
dinamicamente. No caso estático, elas seriam incorporadas ao modelo através de
especialistas. No caso dinâmico, uma solução é utilizar um classificador para
deduzir as regras.

% XXX: Adicionar classificação temporal e viabilidade
% \begin{table}
%   \begin{center}
%     \begin{tabular}{|c|c|c|c|c|c|}
%       \hline
%                         &      &           &          &              &            \\
%       Critério          & CRF  & CRF + ACO & CRF + SA & Lógica Fuzzy & Rede Neural\\
%                         &      &           &          &              &            \\
%       \hline
%                         &      &           &          &              &            \\
%       Requer modelo     &  X   &     X     &    X     &      X       &      -     \\
%       de classificação  &      &           &          &              &            \\
%                         &      &           &          &              &            \\
%       \hline
%                         &      &           &          &              &            \\
%       Estrutura de      &  X   &     X     &    X     &      X       &      -     \\
%       fácil análise     &      &           &          &              &            \\
%                         &      &           &          &              &            \\
%       \hline
%                         &      &           &          &              &            \\
%       Degradação devido &  -   &     X     &    -     &      -       &      -     \\
%       a discretização   &      &           &          &              &            \\
%                         &      &           &          &              &            \\
%       \hline
%                         &      &           &          &              &            \\
%       Fácil             &  -   &     -     &    -     &      -       &      X     \\
%       pré-processamento &      &           &          &              &            \\
%                         &      &           &          &              &            \\
%       \hline
%                         &      &           &          &              &            \\
%       Fácil             &  -   &     -     &    -     &      -       &      X     \\
%       pós-processamento &      &           &          &              &            \\
%                         &      &           &          &              &            \\
%       \hline
%                         &          &          &           &           &          \\
%       Característica    & Modelo   & Modelo   & Parte das & Parte das & Somente  \\
%       reutilizável      & treinado & treinado & regras    & regras    & topologia\\
%                         &          &          &           &           &          \\
%       \hline
%     \end{tabular}
%   \caption{Tabela comparativa dos métodos}
%   \label{regras}
%   \end{center}
% \end{table}

\section{Logica Nebulosa}

\subsection{Introdução}

Sistemas nebulosos aproximam funções. Eles são aproximadores universais se usarem regras suficientes. 
Neste sentido sistemas difusos podem modelar qualquer função ou sistema contínuos. Aqueles sistemas 
podem vir tanto da física quanto da sociologia, bem como da teoria do controle ou do 
processamento de sinais.

A qualidade da aproximação difusa depende da qualidade das regras. Na prática especialistas sugerem regras
difusas ou aprendem-nas através de esquemas neurais através de dados e ajustam as regras com novos dados.
Os resultados sempre aproximam alguma função não linear desconhecida que pode mudar com o tempo. Melhores 
cérebros e melhores redes neurais resultam em melhores aproximações \cite{kosko1997fuzzy}.

\subsection{Modelo Aditivo Padrão(SAM)}

O sistema difuso $F:\Re^n \rightarrow \Re^p$ é em si uma árvore de regras rasa e extensa. É um aproximador
por antecipação. Existem $m$ regras da forma "Se $X$ é conjunto difuso $A$ então $Y$ é conjunto difuso $B$".
A partir desse nível o sistema depende cada vez menos em palavras. 

Cada entrada $x$ aciona parcialmente todas as regras em paralelo. Então o sistema age como um processador 
associativo a medida que calcula a saída
$F(x)$. 
%Definir a_j(x) e b_j(y)

Essas regras relacionam os conjuntos $A_j$ e $B_j$, gerando o caminho difuso $A_j x B_j$. Na prática,
é utilizado o produto para definir $ a_j x b_j (x,y) = a_j(x).b_j(y)$. Esta é a parte "padrão" no SAM.
A parte "aditiva" se refere ao fato de a entrada $x$ acionar a $j$-ésima regra em um grau $a_j(x)$ e o sistema 
soma os acionamentos ou partes escaladas dos conjuntos escalados $a_j(x)B_j$, conforme \cite{kosko1997fuzzy}:

\begin{eqnarray}
F(x) = \frac{\sum w_i.a_i(x).V_i.c_i}{\sum w_j.a_j(x).V_j}
\end{eqnarray}

Com o volume/área $V_j$ e o centroide $c_j$ são dados por:

\begin{eqnarray}
V_j = \int{b_j(y_1,...,y_p)}_{\Re^{p}}.dy_1...dy_p > 0\\
c_j = \frac{\int{y.b_j(y_1,...,y_p)}_{\Re^{p}}.dy_1...dy_p}{V_j}
\end{eqnarray}

\section{Otimização da Colônia de Formigas}

Na busca por alimento, as formigas utilizam de feromônios para encontrar o melhor caminho.
Isso acontece da seguinte maneira: cada formiga deposita feromônio ao se deslocar. A partir
da avaliação da quantidade de feromônio depositada por formigas que já passaram pelo local,
formigas subsequentes tem mais probabilidade de se mover em rotas que tem mais feromônios. Ao
decorrer do tempo os feromônios vão evaporando, apagando rastros que não foram reforçados. 
Com isso, caminhos que são percorridos por mais formigas tem mais chance de serem 
percorridos por outras formigas do que aqueles que foram percorridos por menos formigas e 
caminhos que foram percorridos á pouco tempo tem mais chance de serem percorridos que caminhos
percorridos a muito tempo. A quantidade de feromônio depositado é mais intensa no trajeto de volta,
quando a comida foi encontrada. Outro fator que é levado em consideração é a qualidade da comida
encontrada, de maneira que mais feromônio é depositado quanto melhor for a fonte de alimento encontrada.
A medida que mais formigas exploram o local e encontram alimento, esse procedimento tende a otimizar o
trajeto entre a fonte de alimento e a colônia.

Apesar dessa heurística utilizada pelas formigas ser interessante para se resolver problemas combinatórios 
do tipo NP(i.e., com complexidade não polinomial), são necessários algumas adaptações na construção
de um algoritmo computacional.

A seguir é apresentado a meta-heurística do ACO(\textit{Ant Colony Optimization}) algoritmo, juntamente com observações relacionadas as diferenças
entre a heurística do ACO e o comportamento natural das formigas descrito anteriormente.

\subsection{Pseudo código da meta-heurística do ACO}
%Algoritmo
\begin{algorithm}[H]
%Macros
\SetKwBlock{AgendarAtividade}{AgendarAtividade}{fim}
\SetKwBlock{Procedimento}{Procedimento}{fim}

\Procedimento{
  \Enqto{$n < N_{MAX\_IT}$}{
    %\tcp*[f]{$N_{MAX\_IT}$ é o número máximo de iterações\\}
    \AgendarAtividade{
      ConstruirSolucoesFormigas\\
      AtualizarFeromonios\\
      %\tcp*[f]{opcional}\\
      \tcp{opcional:}
      AcoesGlobais
    }
  }
}
%}

\caption{Pseudo código da meta-heurística do ACO}\label{meta-heuristica_aco}
\end{algorithm}
%\\
%\\

A meta-heurística do ACO pode ser subdividida em três partes, 
conforme proposto por (Dorigo, Marco; 2004): \textit{ConstruirSolucoesFormigas}, 
\textit{AtualizarFeromonios} e \textit{AcoesGlobais}.

\textit{ConstruirSolucoesFormigas} gerencia a movimentação de uma colônia de formigas
em torno dos nós vizinhos. A escolha do próximo nó é feita através de uma decisão 
estocástica que é função da quantidade de feromônio no nós vizinhos e informação heurística.
Quando uma formiga encontra uma solução, ou enquanto a solução é construída, esta avalia a
qualidade da solução(completa ou parcial) que será utilizada pelo procedimento
\textit{AtualizarFeromonios} para decidir a quantidade de feromônio que será depositada.
Outro procedimento relevante na construção da solução é a eliminação de possíveis ciclos, utilizado
por exemplo, no problema do caixeiro viajante.

\textit{AtualizarFeromonios} é o processo que atualiza os traços de feromônio depositados pelas
formigas no espaço de busca. Os traços de feromônio podem aumentar, caso uma formiga tenha visitado
o nó/conexão em questão, ou diminuir, devido ao processo de evaporação do feromônio. Esse procedimento faz com 
que nós/conexões que foram visitados por muitas formigas ou por uma formiga e que tenha levado em
uma solução boa aumentem a probabilidade de serem visitados por futuras formigas. Semelhantemente, reduz 
a probabilidade de que nós que não foram visitados por novas formigas por muitas iterações sejam visitados
novamente. Logo, este procedimento evita a convergência a caminhos sub ótimos, favorecendo também a exploração
de novas regiões do espaço de busca.

Por fim, o procedimento \textit{AcoesGlobais} é utilizado para centralizar ações que não podem ser executadas
pelas formigas individualmente. Um exemplo de ações desse tipo é a filtragem de soluções ou o favorecimento de
regiões por meio de informações globais.

O procedimento \textit{AgendarAtividade} não necessariamente é uma instrução sequencial. Pode-se, portanto,
implementá-lo de maneira sequencial ou paralela, síncrona ou assincronamente. O tipo de abordagem que será
utilizada depende das características do problema que se deseja resolver.

\section{Recozimento Simulado}

No processo de recozimento de um metal, a quantidade de energia interna livre
esta intrinsecamente relacionada ao processo de resfriamento em que o metal é
submetido. Quanto mais rápido se resfriam um metal mais energia é armazenada
internamente. Isso pode ser explicado considerando que o tempo que a estrutura
leva para atingir o estado de menor energia é maior que o disponível devido
a redução da mobilidade dos átomos com o decaimento da temperatura. Com efeito,
quanto maior a taxa de resfriamento maior o número de defeitos na estrutura do
sólido e menor o tamanho médio dos grãos. Quando se reduz a taxa de resfriamento, 
há uma maior chance de se atingir configurações mais estáveis.
Como resultado, a energia interna é reduzida. De acordo com
\cite{bertsimas1993simulated}, pode-se modelar a probabilidade $p_{ij}$ de uma
configuração atômica $\{r_i\}$ com energia $E\{r_i\}$ passar para a configuração $\{r_j\}$ com energia $E\{r_j\}$ na temperatura $T$ como:

\begin{equation}
\mbox{$p_{ij}$}=\left\{
	\begin{array}{rl}
	1 & \mbox{se $E\{r_j\} \le E\{r_i\}$} \\
	exp\left\{-\frac{(E\{r_j\}-E\{r_i\})}{k_B.T}\right\} & \mbox{se $E\{r_j\} > E\{r_i\}$}
\end{array} \right.
\end{equation}

Onde $k_B$ é a constante de Boltzmann. Para se reduzir a energia livre, é necessário que uma
rotina de resfriamento seja escolhida de acordo com o tipo de material a ser resfriado.

Conforme proposto por Kirikpartrick, Gellett e Vechin (1983) e Cerny (1985), pode-se desenvolver
uma heurística probabilística para se encontrar o mínimo global de uma função custo que possua
vários mínimos locais fazendo-se uma analogia com o fenômeno físico descrito acima. A meta-heurística
induzida por este processo é chamada de meta-heurística \textit{Simulated Annealing} (Recozimento
Simulado), ou SA, apresentada a seguir.

\subsection{Meta-heurística do SA}

De acordo com \cite{bertsimas1993simulated}, os elementos básicos da meta-heurística do SA
para a resolução de um problema combinatório são:

\begin{enumerate}
 \item Um conjunto finito $S$.
 \item Um função custo $J$ de imagem real definida em $S$. Seja $S^* \subset S$ o conjunto de todos os mínimos globais da
 função $J$, suposto subconjunto próprio.
 \item Para cada $i \in S$ um conjunto $S(i) \subset S - \{i\}$, chamado de conjunto dos vizinhos de $i$.
 \item Para cada $i$, uma coleção de coeficientes positivos $q_{ij}$, $j \in S(i)$, tal que $\sum_{j \in S(i)} q_{ij} = 1$.
 \item Uma função não crescente $T: \textbf{N} \rightarrow (0,\infty)$, chamada de rotina de resfriamento. Aqui \textbf{N}
 representa o conjunto de inteiros positivos, e $T(t)$ é chamada de \textit{temperatura} no tempo $t$.
 \item Um estado inicial $x(0) \in S$.
\end{enumerate}

Com base nas definições acima, tem-se o pseudo código para a meta-heurística do
SA apresentado no algoritmo~\ref{lst:meta-heuristica_sa}.

%Algoritmo
\begin{algorithm}
%Macros
\SetKwBlock{Procedimento}{Procedimento}{fim}
\SetKwBlock{EscolherVizinho}{EscolherVizinho}{fim}
\SetKwBlock{CalcTransicao}{CalcTransicao}{fim}

\Procedimento{
  SetarValoresInicias\;
  \Para{$n = 1$ até $N_{MAX\_IT}$ ou $J(x^*) \le TOL$ }{
    \Para{$k = 1$ até $N_{MAX\_IT}$ ou a solução convergir}{
      \EscolherVizinho{
        selecionar algum $j \in S(i)$\;
      }

      \CalcTransicao{
        $\Delta J \leftarrow J(j)-J(i)$\;
        \Se{$Delta J \le 0$}{
          $x(t+1) \leftarrow j$\;
          $x^* \leftarrow j$\;
        }
        \Senao{
          %$q_{ij} \leftarrow exp^{\left\{-\frac{\Delta J}{T(t)} \rigth\}}$\;\\
          $q_{ij} \leftarrow exp^{ -\frac{\Delta J}{T(t)} } $\;
          \lSe{$random() < q_{ij}$}{$x(t+1) \leftarrow j$}
          \lSenao{$x(t+1) \leftarrow i$}
        }
      }
    }
    AtualizarTemperetura\;
  }
}

\caption{Pseudo código da meta-heurística do SA}
\label{lst:meta-heuristica_sa}
\end{algorithm}

No algoritmo~\ref{lst:meta-heuristica_sa}, o procedimento \textit{AtualizarTemperetura} executa a
rotina de resfriamento através da função $T(t)$ definida anteriormente. Já o procedimento
\textit{EscolherVisinho} escolhe aleatoriamente um dos elementos da vizinhança do vértice atual $i$.

\subsection{Exemplo de Aplicação}

O método SA pode ser aplicado para otimizar a busca em árvores de decisão.
Com efeito, devido a complexidade de determinados jogos, não é possível
avaliar todos os possíveis resultados de cada jogada em tempo hábil. A
tabela~\ref{table:games} apresenta a complexidade de alguns jogos de tabuleiro.
Na tabela, \textbf{espaço-de-estado} denota o número de posições legais
atingíveis a partir da posição inicial e \textbf{árvore-do-jogo} denota o
número total de jogos que podem ser jogados, i.e., o número de folhas na árvore
de jogo cuja raiz é a posição inicial do jogo.


\begin{table}
  \begin{center}
    \begin{tabular}{|c|c|c|}
      \hline
                        &                              &                      \\
 \textbf{Jogo} & log(\textbf{espaço-de-estado}) & log(\textbf{árvore-do-jogo})\\
                        &                              &                      \\
      \hline
        Jogo da velha   &             3               &          5           \\
      \hline
        Trilha          &             10              &          50          \\
      \hline
        Oware           &             16              &          32          \\
      \hline
        Pentaminó       &             12              &          18          \\
      \hline
        Lig 4           &             14              &          21          \\
      \hline
        Damas           &             33              &          50          \\
      \hline
        Lines of Action &             24              &          56          \\
      \hline
        Reversi         &             28              &          58          \\
      \hline
        Gamão           &             20              &          144         \\
      \hline
        Quoridor        &             42              &          162         \\
      \hline
        Xadrez          &             46              &          123         \\
      \hline
        Xadrez Chinês   &             52              &          150         \\
      \hline
        Arimaa          &             42              &          190         \\
      \hline
        Shogi           &             71              &          226         \\
      \hline
        Connect6        &             172             &          140         \\
      \hline
        Go              &             171             &          360         \\
      \hline
    \end{tabular}
    \caption{Complexidades do espaço de estado e da árvore do jogo de alguns
             jogos \cite{mertens2006quoridor}}
  \label{table:games}
  \end{center}
\end{table}


%%%%Applying Genetic Algorithms to
%%%Quoridor Game Search Trees for Next-Move Selection
Como o SA é uma otimização de busca local guloso, assumi-se que o adversário
selecionará a melhor jogada de acordo com uma função de avaliação comum aos dois
jogadores. Então a seleção de nós é feita utilizando-se o SA tradicional.

Isso conceitualmente traduz no jogador SA ter olhado para baixo três níveis de
uma determinada posição na árvore de jogo para avaliar a posição atual. A mesma
função heurística de avaliação linear é usada para a avaliação da própria
posição de jogo. No caso do jogo quoridor essa heurística leva em consideração
tanto o número de paredes restantes para cada jogador quanto a distância de cada
jogador ao lado objetivo.

A função objetivo é responsável por determinar a probabilidade de movimentos
piores serem tomados a partir de uma determinada posição. Uma possível função
objetivo é da forma:

\begin{equation}
  exp\left\{\frac{h(atual)- h(vizinho)}{tempo}\right\}
\end{equation}

em que $h$ é uma função de avaliação heurística que é tanto menor quanto melhor
for o estado do jogo. O parâmetro de controle aleatoriedade contra a busca
gananciosa pura é o tempo, e não um constante. Esse é o aspecto do SA que evita
ficar preso em ótimos locais.

Quando se atinge estágios posteriores na árvore de busca jogo onde o jogador
esta perto de seu estado objetivo, a diferença $h(atual)- h(vizinho)$
será relativamente grande (sendo relativamente pequena no início do jogo). Isto
sugere que o uso de um parâmetro constante no lugar do tempo resultaria em um
jogador que seleciona movimentos ruins mais frequentemente perto do fim.
Como o tempo também seria um valor relativamente grande perto do final do jogo,
isso iria compensar essa grande diferença e levar ao estado objetivo mais
rápido \cite{mcdermid2003gaquoridor}.


\section{Algorítimo Genético}

Um \emph{algorítimo genético} é uma heurística de busca que procura
imitar a seleção natural que ocorre no processo evolucionário dos
organismos vivos.

Nessa heurística, uma população de soluções (também chamadas de
indivíduos ou fenótipos) para problemas de otimização é evoluída para
conseguir soluções melhores. Cada solução possui um conjunto de
propriedades (cromossomos ou genótipos) que podem ser mutados ou
alterados.

Os requerimentos são, tipicamente:

\begin{itemize}
\item
  uma representação genética da solução
\item
  uma função de aptidão para avaliação da solução
\end{itemize}

\subsection{O processo}

O processo é iniciado com uma população com propriedades geradas
aleatoriamente.

A iteração da heurística se da em 3 etapas:

\begin{itemize}
\item
  procriação: indivíduos são pareados e é aplicada a operação de
  cruzamento (\emph{crossover})
\item
  mutação: alguns indivíduos são selecionados e é aplicada a operação de
  mutação (\emph{mutation})
\item
  seleção: é usada a função de aptidão para descartar os indivíduos
  menos aptos restando as soluções que de fato trouxeram alguma melhora.
\end{itemize}

As condições mais comuns para terminação do processo são as seguintes:

\begin{itemize}
\item
  encontrada uma solução que atende os requisitos mínimos
\item
  número fixo de gerações alcançado
\item
  recursos alocados (tempo ou dinheiro) alcançados
\item
  a melhor solução alcançou um patamar estável em que mais iterações não
  produzem soluções melhores
\item
  inspeção manual
\end{itemize}

\subsection{Limitações}

As limitações mais comuns no emprego de um algorítimo genético são:

\begin{itemize}
\item
  Funções de avaliação computacionalmente caras tornam essa heurística
  ineficiente.
\item
  Não escala bem com a complexidade, isto é, quando o número de
  elementos expostos a mutação é grande o espaço de busca cresce
  exponencialmente. Por isso, na prática algorítimos genéticos são
  usados para, por exemplo, projetar uma hélice e não um motor.
\item
  A melhor solução é relativa às outras soluções, por isso o critério de
  parada não é muito claro em alguns problemas.
\item
  Em muitos problemas os algorítimos genéticos tendem a convergir para
  um ótimo local ou as vezes pontos arbitrários em vez do ótimo global.
\item
  É difícil aplicar algorítimos genéticos para conjunto de dados
  dinâmicos. Pois as soluções podem começar a convergir para um conjunto
  de dados que já não é mais válido.
\item
  Algorítimos genéticos não conseguem resolver eficientemente problemas
  em que a avaliação é binária (certo/errado), como em problemas de
  decisão. Nesse caso buscas aleatórias convergem tão rápido quanto essa
  heurística.
\item
  Para problemas mais específicos existem outras heurísticas que
  encontram a solução mais rapidamente.
\end{itemize}

\subsection{Pseudo código de um Algorítimo Genético}

%\begin{lstlisting}
\begin{algorithm}[H]
\SetKwBlock{Procedimento}{Procedimento}{fim}

%Algorithm: GA(n, \ki, \mu)
\Procedimento{
  %// Initialise generation 0:
  $k \leftarrow 0$\;
  $P_k \leftarrow $ população de n indivíduos escolhidos aleatoriamente\;
  %// EvaluatePk:
  %\Para{cada $i$ em $P_k$}
  %Compute fitness(i) for each i ∈ Pk;
  %Computar a $avaliacao(i)$ para cada $i$ em $P_k$\;
  %while fitness of fittest individual in Pk is not high enough;
  \Enqto{a $avaliacao(i)$ de cada $i$ em $P_k$ não for boa o suficiente}{
    %// Create generation k + 1:
    %// 1. Copy:
    %Select (1−χ)×n members ofPk and insert into Pk+1;
    Selecionar os $(1 - \chi) \times n$ membros com maior $avaliacao(i)$ de $P_k$ e inserir em $P_{k+1}$\;
    %// 2. Crossover:
    %Select χ×n members of Pk; pair them up; produce offspring; insert the offspring into Pk+1;
    Selecionar $\chi \times n$ membros de $P_k$, pareá-los e inserir a cria em $P_{k+1}$\;
    %// 3. Mutate:
    %Select µ×n members of Pk+1; invert a randomly-selected bit in each;
    Selecionar os $\mu \times n$ membros de $P_{k+1}$ com maior $avaliacao(i)$ e inverter um bit aleatório de cada membro\;
    %// Evaluate Pk+1:
    %Compute fitness(i) for each i ∈ Pk;
    %Computar a $avaliacao(i)$ para cada $i$ em $P_{k+1}$\;
    %// Increment:
    %k := k + 1;
    $k \leftarrow k + 1$\;
  }
%return the fittest individual from Pk;ut your code here.
  $melhor \leftarrow$ o membro $i$ em $P_k$ com maior $avaliacao(i)$\;
  \Retorna{$melhor$}
}
\end{algorithm}
%\end{lstlisting}

%\subsection{Referências}
%
%\begin{itemize}
%\item
%  \href{http://en.wikipedia.org/wiki/Genetic\_algorithm}{Genetic algorithm}
%\item
%  \href{http://www.cs.ucc.ie/~dgb/courses/tai/notes/handout12.pdf}{Genetic Algorithms - Derek Bridge}
%\end{itemize}

\subsection{Exemplo de Aplicação}

Suponha uma fábrica com duas máquinas, e chegam quatro pedidos, em ordem, para serem produzidos.
Cada pedido consiste irá levar um tempo fixo para ser produzido.
O problema a ser resolvido é determinar qual máquina processa cada pedido, de tal modo que
o tempo total (até o último pedido ser atendido) seja mínimo.
Concretizando o exemplo supõe-se os seguintes pedidos: $A$: 1h, $B$: 5h, $C$: 2h, $D$: 3h.

Primeiro deve ser modelado a representação genética, ou fenótipo.
Uma proposta simples é uma string de 4 bits, em que cada bit representa qual das duas máquinas produz
o pedido, e a ordem dos bits é a ordem dos pedidos.

Segundo uma função de aptidão, que nesse caso é determinística: o maior entre a soma dos tempos dos pedidos
de bits 0 e a soma dos de bit 1. Isto é para uma sequencia $0110$ teríamos o maior entre $1h + 3h$ e $5h + 2h$,
ou seja $7h$.

Terceiro uma regra de cruzamento, que nesse caso pode simplesmente ser a troca aleatória de alguns bits.
Por exemplo, o cruzamento de $1100$ e $0011$ pode trocar apenas o primeiro bit, resultando em $0100$ e $1011$.

Por último uma regra de mutação, nesse caso é sufiente a inversão de um bit.

Com esses quatro requisitos é possível aplicar a iteração genética sobre uma população inicial.

Esse exemplo demonstra a modelagem de uma otimização usando um algorítimo genético e foi baseado em \cite{vieira2002algogeneticos}.


\subsection{Aplicação ao problema}

O problema descrito neste trabalho pode se beneficiar de algorítimos genéticos para otimizar conjuntos de
parâmetros de outras heurísticas.

\section{Rede Neural}

O termo mais apropriado é rede neural aritificial, já que apenas rede
neural pode se referir ao sistema biológico de nervos, no entando dado o
contexto desse texto e o uso consagrado do termo ``rede neural'', esse
será usado no lugar da versão mais explícita ``rede neural artificial''.

Uma rede neural é um sistema inspirado no sistema nervoso central (em
especial o cérebro) encontrado em muitos animais. A ideia básica é ter
um grafo em que cada nó abstrai um neurônio e é representado como uma
função, alguns desses nós são responsáveis pela observação e outros pela
saída e os nós de entrada alimentam os próximos nós até chegar nos nós
de saída. \cite{haykin2001redes}

\begin{figure}[H]
  \centering
  \includegraphics[width=10cm]{figuras/rede_neural_grafo}
  \caption{Rede neural com três camadas.}\label{fig:rede_neural_grafo}
\end{figure}

A figura \ref{fig:rede_neural_grafo} exemplifica uma rede neural \emph{feedforward},
que é baseada num grafo direcionado acíclico, em que podem ser vistas 3 camadas
a primeira é chamada de camada de entrada, a última, de saída e as intermediárias,
de escondidas. \cite{shiffman2012nature}

Um dos diferencias da rede neural é a capacidade de aprender, essa heurística
forma um sistem adaptativo. Existem três tipos de aprendizados:

\begin{itemize}
\item
  Aprendizado supervisionado: alimentar a rede com um problema cuja a solução é conhecida
  e depois fornecer a resposta certa para que a rede possa se ajustar.
\item
  Aprendizado não supervisionado: consiste em buscar padrões não conhecidos, não se conhece
  a resposta certa ou se uma resposta é certa ou não.
\item
  Aprendizado por reforço: alimentar a rede com um problema cuja a solução pode ser avaliada
  em boa ou má. Esse tipo de aprendizado é comum em robótica onde o robô caminha por um ambiente
  e tem o reforço negativo ou positivo de colodir ou encontrar o objetivo.
\end{itemize}

\subsection{O Neurônio}

O bloco de construção básico de uma rede neural são os neurônios.

\begin{figure}[H]
  \centering
  \includegraphics[width=10cm]{figuras/rede_neural_perceptron}
  \caption{Perceptron de duas entradas e uma saída.}\label{fig:rede_neural_perceptron}
\end{figure}

A figura~\ref{fig:rede_neural_perceptron} mostra um perceptron.

%\ldots{}

%\begin{itemize}
%\itemsep1pt\parskip0pt\parsep0pt
%\item
%  http://en.wikipedia.org/wiki/Neural\_network
%\item
%  http://en.wikipedia.org/wiki/Artificial\_neural\_network
%\item
%  HAYKIN, S. Redes neurais princípios e prática.
%\end{itemize}


% modelagem é tipo uma introducao
%\input{partes/modelagem}

%\chapter{Cronograma}
% TODO(depois): fazer cronograma


% Conclusão
% ---------

\input{partes/proximas_etapas}
\chapter{Conclusão}\label{cap:conclusao}

Este trabalho descreve uma modelagem pesquisada para o jogo de futebol de robôs da
categoria SSL da competição RoboCup e introduz métodos que podem
contribuir para a confecção da solução final. Concluiu-se que nem todas as variáveis
desejadas durante  processo de predição são observáveis, tornando o problema mais
complexo do que realmente se imaginava. A Rede Neural, ACO, SA, \textit{Lógica Fuzzy}
e o AG são abordagens apropriadas para o problema. Para a Rede Neural e \textit{Lógica Fuzzy}
não é necessário supor uma modelagem, uma vez que ela incorpora conhecimento
implicitamente em seus parâmetros. Esse já não é o caso dos métodos restantes, e por isso
foi necessário definir alguns parâmetros inicialmente. As heurísticas de otimização
serão de grande importância para a próxima etapa. Faz-se necessário continuar a
pesquisa para desenvolver os algoritmos que serão efetivamente utilizados.


% Elementos pós-textuais
% ======================

\section{Referências}
\frame{
\frametitle{Referências}
\begin{itemize}
\item BERTSIMAS, D.; TSITSIKLIS, J. Simulated Annealing. Statistical Science, JSTOR, p.
10–15, 1993;
\item DORIGO, M.; STÜTZLE, T. Ant Colony Optimization. [S.l.]: Bradford Book, 2004.
ISBN 0262042193;
\item HAYKIN, S. Redes Neurais. [S.l.]: Grupo A, 2001;
\item KOSKO, B. Fuzzy engineering. [S.l.]: Prentice-Hall, Inc., 1997;
\item SHIFFMAN, D.; FRY, S.; MARSH, Z. The Nature of Code. [S.l.]: D. Shiffman, 2012.
\end{itemize} 
}


% begin opcionais
\input{pos_textuais/apendices}
\input{pos_textuais/anexos}
\input{pos_textuais/indice}
\input{pos_textuais/glossario}
% end opcionais

\input{pos_textuais/contracapa}

\bibliography{bibliografia}
% FIM!!!!
% =======
\end{document}
