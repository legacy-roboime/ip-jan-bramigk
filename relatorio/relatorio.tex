%% abtex2-modelo-trabalho-academico.tex, v-1.8 laurocesar
%% Copyright 2012-2013 by abnTeX2 group at http://abntex2.googlecode.com/ 
%%
%% This work may be distributed and/or modified under the
%% conditions of the LaTeX Project Public License, either version 1.3
%% of this license or (at your option) any later version.
%% The latest version of this license is in
%%   http://www.latex-project.org/lppl.txt
%% and version 1.3 or later is part of all distributions of LaTeX
%% version 2005/12/01 or later.
%%
%% This work has the LPPL maintenance status `maintained'.
%% 
%% The Current Maintainer of this work is the abnTeX2 team, led
%% by Lauro César Araujo. Further information are available on 
%% http://abntex2.googlecode.com/
%%
%% This work consists of the files abntex2-modelo-trabalho-academico.tex,
%% abntex2-modelo-include-comandos and abntex2-modelo-references.bib
%%

% ------------------------------------------------------------------------
% ------------------------------------------------------------------------
% abnTeX2: Modelo de Trabalho Academico (tese de doutorado, dissertacao de
% mestrado e trabalhos monograficos em geral) em conformidade com 
% ABNT NBR 14724:2011: Informacao e documentacao - Trabalhos academicos -
% Apresentacao
% ------------------------------------------------------------------------
% ------------------------------------------------------------------------

\documentclass[
	% -- opções da classe memoir --
	12pt,				% tamanho da fonte
	%openright,			% capítulos começam em pág ímpar (insere página vazia caso preciso)
	%twoside,			% para impressão em verso e anverso. Oposto a oneside
	oneside,			% para impressão em verso e anverso. Oposto a oneside
	a4paper,			% tamanho do papel. 
	% -- opções da classe abntex2 --
	%chapter=TITLE,		% títulos de capítulos convertidos em letras maiúsculas
	%section=TITLE,		% títulos de seções convertidos em letras maiúsculas
	%subsection=TITLE,	% títulos de subseções convertidos em letras maiúsculas
	%subsubsection=TITLE,% títulos de subsubseções convertidos em letras maiúsculas
	% -- opções do pacote babel --
	english,			% idioma adicional para hifenização
	brazil,				% o último idioma é o principal do documento
	% -- opções da classe ime-abntex2 --
	%brasao,
	]{ime-abntex2}


% ---
% PACOTES
% ---

% ---
% Pacotes fundamentais 
% ---
\usepackage[utf8]{inputenc}		% Codificacao do documento (conversão automática dos acentos)
\usepackage{cmap}			% Mapear caracteres especiais no PDF
\usepackage{lmodern}			% Usa a fonte Latin Modern
%\usepackage{amsmath}			% Para usar fontes ams
%\usepackage{amsfonts}			% Para usar fontes ams
%\usepackage{times}			% Usa a fonte Times
\usepackage[T1]{fontenc}		% Selecao de codigos de fonte.
%\usepackage{lastpage}			% Usado pela Ficha catalográfica
\usepackage{indentfirst}		% Indenta o primeiro parágrafo de cada seção.
\usepackage{color}			% Controle das cores
\usepackage{graphicx}			% Inclusão de gráficos
\usepackage{algorithm2e}		% Para escrever algorítimos
\usepackage{lscape}			% Para usar landscape
\usepackage{amsthm}			% Para usar Definições e lemas
\usepackage{amssymb}			% \mathbb{•}
\usepackage{caption}			% dependencia de subcaption
%\usepackage{subcaption}			% Para usar \begin{subfigure} e colocar figuras (a) e (b) lado a lado
\usepackage{listings}			% Para usar \lstinputlisting e incluir código
\newtheorem{teo}{Teorema}[section]	%Renomear o comando
\newtheorem{defi}[teo]{Definição}
% o seguinte é necessário para corrigir um bug no algorithm2e
% http://tex.stackexchange.com/questions/113325/problem-with-algorithm2e-and-portuguese-option
\SetKwFor{Para}{para}{fa\c{c}a}{fim para}
% não era pra precisar do seguinte, mas precisa para traduzir a captions dos algoritmos
\renewcommand{\algorithmcfname}{Algoritmo}%
% ---

% ---
% Pacotes de citacoes
% ---
\usepackage[brazilian,hyperpageref]{backref}	 % Paginas com as citações na bibl
\usepackage[alf]{abntex2cite}	% Citações padrão ABNT

% --- 
% CONFIGURAÇÕES DE PACOTES
% --- 

% ---
% Configurações do pacote backref
% Usado sem a opção hyperpageref de backref
\renewcommand{\backrefpagesname}{Citado na(s) página(s):~}
% Texto padrão antes do número das páginas
\renewcommand{\backref}{}
% Define os textos da citação
\renewcommand*{\backrefalt}[4]{
	\ifcase #1 %
		Nenhuma citação no texto.%
	\or
		Citado na página #2.%
	\else
		Citado #1 vezes nas páginas #2.%
	\fi}%
% ---


% ---
% Informações de dados para CAPA e FOLHA DE ROSTO
% ---
\titulo{Heurística Estática para Times\\Cooperativos de Robôs}
\autor{
  Jan Segre\\
  Victor Bramigk
}
\local{Rio de Janeiro}
\data{Março de 2014}
\orientador{Paulo F. F. Rosa - Ph.D}
%\coorientador{Equipe \abnTeX}
\instituicao{%
  Instituto Militar de Engenharia
  \par
  Seção de Computação
  \par
  Graduação em Engenharia de Computação}
\tipotrabalho{Iniciação à Pesquisa}
% O preambulo deve conter o tipo do trabalho, o objetivo, 
% o nome da instituição e a área de concentração 
\preambulo{Iniciação à Pesquisa apresentada ao Curso de Graduação
em Engenharia de Computação do Instituto Militar de
Engenharia.}
% ---


% ---
% Configurações de aparência do PDF final

% alterando o aspecto da cor azul
%\definecolor{blue}{RGB}{41,5,195}
\definecolor{blue}{RGB}{0,0,0}

% informações do PDF
\makeatletter
\hypersetup{
	%pagebackref=true,
	pdftitle={\@title},
	pdfauthor={\@author},
	pdfsubject={\imprimirpreambulo},
	pdfcreator={LaTeX with abnTeX2},
	pdfkeywords={ime}{robocup}{analise}{logs}{rede neural},
	colorlinks=true,		% false: boxed links; true: colored links
	linkcolor=blue,			% color of internal links
	citecolor=blue,			% color of links to bibliography
	filecolor=magenta,		% color of file links
	urlcolor=blue,
	bookmarksdepth=4
}
\makeatother
% ---

% ---
% Espaçamentos entre linhas e parágrafos 
% ---

% O tamanho do parágrafo é dado por:
%\setlength{\parindent}{1.3cm}

% Controle do espaçamento entre um parágrafo e outro:
%\setlength{\parskip}{0.2cm}  % tente também \onelineskip
\setlength{\parskip}{\onelineskip}

% ---
% compila o indice
% ---
\makeindex
% ---

% ----
% Início do documento
% ----
\begin{document}

% Retira espaço extra obsoleto entre as frases.
%\frenchspacing

% ----------------------------------------------------------
% ELEMENTOS PRÉ-TEXTUAIS
% ----------------------------------------------------------
% \pretextual

% ---
% Capa
% ---
\imprimircapa
%\begin{titlepage}

\begin{center}
\large{MINISTÉRIO DA DEFESA} \\
\large{EXÉRCITO BRASILEIRO} \\
\large{DEPARTAMENTO DE CIÊNCIA E TECNOLOGIA}\\
\large{IME - INSTITUTO MILITAR DE ENGENHARIA} \\
\large{CURSO DE GRADUAÇÃO EM ENGENHARIA DA COMPUTAÇÃO} \\
\vspace{5 cm}
\large{CAMILA ANTONACCIO WANOUS} \\
\large{JOÃO LUIZ DO PRADO NETO} \\
\large{THIAGO DE PAULA VASCONCELOS} \\
\vspace{5 cm}
\large{CONECTIVIDADE ALGÉBRICA E SEUS AUTOVETORES NA CLASSE DAS ÁRVORES}\\
\vspace{5 cm}
Rio de Janeiro\\
2013
\end{center}

\end{titlepage}

%\begin{center}
%\textbf{MINISTÉRIO DA DEFESA}\\
%\textbf{EXÉRCITO BRASILEIRO}\\
%\textbf{DEPARTAMENTO DE CIÊNCIA E TECNOLOGIA}\\
%\textbf{INSTITUTO MILITAR DE ENGENHARIA}\\
%\textbf{Seção de Engenharia de Sistemas / SE 8}
%
%\vspace{2.5cm}
%
%\begin{large}
%\textbf{Proposta de Tema de Dissertação de Mestrado
%\\Curso: Mestrado em Sistemas e Computação}
%
%\vspace{1.5cm}
%
%\textbf{Sistema de Localização de Objetos para Apoio a um Assistente Robótico Móvel na Casa Inteligente}
%
%\vspace{1.5cm}
%
%\textbf{Aluno: Fulano}
%
%\vspace{1.5cm}
%
%\textbf{Orientador: Paulo F. F. Rosa, Ph.D}
%
%\end{large}
%
%\vspace{2cm}
%
%\begin{small}
%Data de Apresentação no SE/8:\\
%Rio de Janeiro, \today
%\end{small}
%
%\end{center}
%
%\pagebreak

% ---

% ---
% Folha de rosto
% (o * indica que haverá a ficha bibliográfica)
% ---
\imprimirfolhaderosto*
%\input{pre_textuais/folha_de_rosto}
% ---

% ---
% Inserir a ficha bibliografica
% ---

% Isto é um exemplo de Ficha Catalográfica, ou ``Dados internacionais de
% catalogação-na-publicação''. Você pode utilizar este modelo como referência. 
% Porém, provavelmente a biblioteca da sua universidade lhe fornecerá um PDF
% com a ficha catalográfica definitiva após a defesa do trabalho. Quando estiver
% com o documento, salve-o como PDF no diretório do seu projeto e substitua todo
% o conteúdo de implementação deste arquivo pelo comando abaixo:
%
% \begin{fichacatalografica}
%     \includepdf{fig_ficha_catalografica.pdf}
% \end{fichacatalografica}
\imprimirfichacatalografica
{xxxx}
{Segre, J., Bramigk, V.}
{1. Engenharia da computação.
 2. Redes Neurais.
 3. Conditional Random Fields
 4. Lógica Nebulosa.
 5. Otimização da Colônia de Formigas. Instituto Militar de Engenharia.
}
% ---

% ---
% Inserir errata
% ---
%\begin{errata}
%Elemento opcional da \citeonline[4.2.1.2]{NBR14724:2011}. Exemplo:
%
%\vspace{\onelineskip}
%
%FERRIGNO, C. R. A. \textbf{Tratamento de neoplasias ósseas apendiculares com
%reimplantação de enxerto ósseo autólogo autoclavado associado ao plasma
%rico em plaquetas}: estudo crítico na cirurgia de preservação de membro em
%cães. 2011. 128 f. Tese (Livre-Docência) - Faculdade de Medicina Veterinária e
%Zootecnia, Universidade de São Paulo, São Paulo, 2011.
%
%\begin{table}[htb]
%\center
%\footnotesize
%\begin{tabular}{|p{1.4cm}|p{1cm}|p{3cm}|p{3cm}|}
%  \hline
%   \textbf{Folha} & \textbf{Linha}  & \textbf{Onde se lê}  & \textbf{Leia-se}  \\
%    \hline
%    1 & 10 & auto-conclavo & autoconclavo\\
%   \hline
%\end{tabular}
%\end{table}
%
%\end{errata}
% ---

% ---
% Inserir folha de aprovação
% ---
%
% Isto é um exemplo de Folha de aprovação, elemento obrigatório da NBR
% 14724/2011 (seção 4.2.1.3). Você pode utilizar este modelo até a aprovação
% do trabalho. Após isso, substitua todo o conteúdo deste arquivo por uma
% imagem da página assinada pela banca com o comando abaixo:
%
% \includepdf{folhadeaprovacao_final.pdf}
%
\convidadoum{Julio Cesar Duarte - D\.Sc\. do IME}{}
\convidadodois{Ricardo Choren Noya - Ph.D do IME}{}
\imprimirfolhadeaprovacao{27 de março de 2014}
% ---

% ---
% Dedicatória
% ---
%\begin{dedicatoria}
%   \vspace*{\fill}
%   \centering
%   \noindent
%   \textit{ Este trabalho é dedicado às crianças adultas que,\\
%   quando pequenas, sonharam em se tornar cientistas.} \vspace*{\fill}
%\end{dedicatoria}
% ---

% ---
% Agradecimentos
% ---
%\begin{agradecimentos}
%Os agradecimentos principais são direcionados à Gerald Weber, Miguel Frasson,
%Leslie H. Watter, Bruno Parente Lima, Flávio de Vasconcellos Corrêa, Otavio Real
%Salvador, Renato Machnievscz\footnote{Os nomes dos integrantes do primeiro
%projeto abn\TeX\ foram extraídos de
%\url{http://codigolivre.org.br/projects/abntex/}} e todos aqueles que
%contribuíram para que a produção de trabalhos acadêmicos conforme
%as normas ABNT com \LaTeX\ fosse possível.
%
%Agradecimentos especiais são direcionados ao Centro de Pesquisa em Arquitetura
%da Informação\footnote{\url{http://www.cpai.unb.br/}} da Universidade de
%Brasília (CPAI), ao grupo de usuários
%\emph{latex-br}\footnote{\url{http://groups.google.com/group/latex-br}} e aos
%novos voluntários do grupo
%\emph{\abnTeX}\footnote{\url{http://groups.google.com/group/abntex2} e
%\url{http://abntex2.googlecode.com/}}~que contribuíram e que ainda
%contribuirão para a evolução do \abnTeX.
%
%\end{agradecimentos}
% ---

% ---
% Epígrafe
% ---
%\begin{epigrafe}
%    \vspace*{\fill}
%	\begin{flushright}
%		\textit{``Não vos amoldeis às estruturas deste mundo, \\
%		mas transformai-vos pela renovação da mente, \\
%		a fim de distinguir qual é a vontade de Deus: \\
%		o que é bom, o que Lhe é agradável, o que é perfeito.\\
%		(Bíblia Sagrada, Romanos 12, 2)}
%	\end{flushright}
%\end{epigrafe}
% ---

% ---
% RESUMOS
% ---

% resumo em português
\setlength{\absparsep}{18pt} % ajusta o espaçamento dos parágrafos do resumo
\begin{resumo}
% TODO Citar KDD
% TODO Contextualizar o tema
% TODO Apresentar objetivos
% TODO Listar contribuições


Neste trabalho são abordadas maneiras de analisar \textit{logs} de
partidas de futebol de robôs da categoria SSL da competição RoboCup
com a finalidade de obter melhores resultados na Inteligência
Artificial do time RoboIME (do Laboratório de Robótica do IME) que
participa anualmente dessa competição.  O objetivo desejado é prever
como um time de futebol de robôs irá se comportar baseado somente nas
informações coletadas de partidas, chamados \textit{logs}.  Foi
elaborado uma modelagem de alto nível que permite utilizar rotinas já
existentes tanto para testes quanto predições.  Foram estudados os
métodos da ACO (\textit{Ant Colony Optimization}), SA
(\textit{Simulated Annealing}), AG (Algoritimo  Genético), lógica
nebulosa, redes neurais e CRFs (\textit{Conditional Random Fields}). A
partir do estudo detalhado foram definidos métodos de abordagem do
problema. Foi necessário especificar o modelo de time nas abordagens
envolvendo ACO e SA, que tem como base os CRFs. A abordagem envolvendo
lógica nebulosa se mostrou muito complexo para ser desenvolvido neste
trabalho.  Após um estudo mais aprofundado deseja-se futuramente
implementar um processo de otimização em ambos os algorítimos para que
o resultado seja refinado e para avaliar o desempenho dos métodos
propostos.  Para futuros trabalhos os métodos propostos são ideais
para simular o adversário num algoritmo \textit{MiniMax}.

% Segundo a \citeonline[3.1-3.2]{NBR6028:2003}, o resumo deve ressaltar o
% objetivo, o método, os resultados e as conclusões do documento. A ordem e a extensão
% destes itens dependem do tipo de resumo (informativo ou indicativo) e do
% tratamento que cada item recebe no documento original. O resumo deve ser
% precedido da referência do documento, com exceção do resumo inserido no
% próprio documento. (\ldots) As palavras-chave devem figurar logo abaixo do
% resumo, antecedidas da expressão Palavras-chave:, separadas entre si por
% ponto e finalizadas também por ponto.
%
% \textbf{Palavras-chaves}: latex. abntex. editoração de texto.
\end{resumo}

% resumo em inglês
% TODO Atualizar
\begin{resumo}[Abstract]
\begin{otherlanguage*}{english}

This paper covers ways to analyze logs of robot soccer matches from
RoboCup's SSL category in order to obtain better results on the
RoboIME's artificial intelligence (from IME's robotics lab) which
participates yearly on that competition.  The desired goal is to
predict the behaviour of a robot team based solely on data collected
on previous matches, which are called logs.  A high level modeling was
developed such that existing routines could be used both for tests and
to actual predictions.  The following methods were researched: ACO
(Ant Colony Optimization), SA (Simulated Annealing), GA (Genetic
Algorithm), fuzzy logic, neural networks and CRFs (Conditional Random
Fields).  From the detailed research made, some approaches were
defined.  It was necessary to specify the team model for the
approaches involving the ACO and SA, which use CRFs as a base.  The
approach involving fuzzy logic showed itself too complex to be
developed in this paper.  After a thrill study it is desired to
implement an optimization process for both algorithms, such that the
result can be refined and the proposed methods benchmarked.  For
future works the proposed methods are ideal adversary simulators for a
MiniMax algorithm.

%\vspace{\onelineskip}

%\noindent
%\textbf{Key-words}: latex. abntex. text editoration.
\end{otherlanguage*}
\end{resumo}

% resumo em francês 
%\begin{resumo}[Résumé]
% \begin{otherlanguage*}{french}
%    Il s'agit d'un résumé en français.
% 
%   \textbf{Mots-clés}: latex. abntex. publication de textes.
% \end{otherlanguage*}
%\end{resumo}

% resumo em espanhol
%\begin{resumo}[Resumen]
% \begin{otherlanguage*}{spanish}
%   Este es el resumen en español.
%  
%   \textbf{Palabras clave}: latex. abntex. publicación de textos.
% \end{otherlanguage*}
%\end{resumo}
% ---

% ---
% inserir o sumario
% ---
\pdfbookmark[0]{\contentsname}{toc}
\tableofcontents*
\cleardoublepage
% ---

% ---
% inserir lista de ilustrações
% ---
\pdfbookmark[0]{\listfigurename}{lof}
\listoffigures*
\cleardoublepage
% ---

% ---
% inserir lista de tabelas
% ---
%\pdfbookmark[0]{\listtablename}{lot}
%\listoftables*
%\cleardoublepage
% ---

% ---
% inserir lista de abreviaturas e siglas
% ---
\begin{siglas}
	\item[ACO] \textit{Ant Colony Optimization}
	\item[AG] Algoritmo genético
	\item[IA] Inteligência artificial
	\item[KDD] \textit{Knowledge Discovery in Databases}
	\item[RoboCup] \textit{Robot World Cup Initiative}
	\item[SA] \textit{Simulated Annealing}
	\item[SAM] \textit{Standard Addition Method}
	\item[SSL] \textit{Small Size League}
	\item[STP] \textit{Skill Tactic Play}
  \item[CRF] \textit{Conditional Random Fields}
\end{siglas}
% ---

% ---
% inserir lista de símbolos
% ---
%\begin{simbolos}
%  \item[$ \Gamma $] Letra grega Gama
%  \item[$ \Lambda $] Lambda
%  \item[$ \zeta $] Letra grega minúscula zeta
%  \item[$ \in $] Pertence
%\end{simbolos}
% ---



% ----------------------------------------------------------
% ELEMENTOS TEXTUAIS
% ----------------------------------------------------------
\textual

% ----------------------------------------------------------
% Introdução
% ----------------------------------------------------------
\chapter{Introdução}


As influências da robótica
são visíveis na indústria moderna. Ela tem viabilizado o desenvolvimento de peças
precisas, assim como um controle maior do processo. O uso de robôs confere aos
processos industriais precisão e repetibilidade maior que as adquiridas caso
fosse empregado um humano. Isso, além de reduzir o custo de retrabalhos, permite
que os desenvolvedores se preocupem mais com o processo em si. Assim, o emprego
de robôs confere um amadurecimento dos processos industrias, bem como indiretamente
permite que a sociedade concentre esforços em cargos intelectuais.

Com efeito, o emprego desses sistemas robóticos deixou evidente a necessidade do
desenvolvimento de teorias relacionadas as sistemas autônomos cooperativos. 

% Esses sistemas podem
% ser empregados para permitir que resgates sejam feitos de maneira eficiente. Isso
% evitaria que o pessoal altamente especializado empregado atualmente corra risco de vida.

Entretanto, projetar robôs autônomos para trabalharem juntos não é uma tarefa trivial. Essa
tarefa se complica quando um robô não tem um modelo bem definido dos outros robôs que atuarão em
conjunto. Um domínio de aplicação que envolve essa problemática é o futebol de robôs.
Nesse domínio é comum a distribuição de papéis dinamicamente entre os membros de cada
time. Entretanto, um time não tem conhecimento dos papéis atribuídos aos robôs
do time oponente. O planejamento de um time pode ser aprimorado através de um modelo
aproximado desse time oponente, pois levará em consideração a maneira como o
time reagirá às diversas ações possíveis.

\begin{figure}
  \includegraphics[width = \linewidth]{figuras/robocup2013}
  \caption{Imagem da SSL \textit{RoboCup} 2013 em Eindhoven, na Holanda}\label{fig:robocup2013}
\end{figure}

A ideia de robôs jogando futebol foi mencionada pela primeira vez pelo professor
Alan Mackworth (\textit{University of British Columbia}, Canadá) em um artigo intitulado
\textit{"On Seeing Robots"}, apresentado no \textit{Vision Interface 92} e posteriormente publicado em
um livro chamado \textit{Computer Vision: System, Theory and Applications}. Independentemente,
um grupo de pesquisadores japoneses organizou um \textit{Workshop} no \textit{Ground Challange
in Artificial Inteligence}, em Outubro de 1992, Tóquio, discutindo e propondo problemas que
representavam grandes desafios. Esse \textit{Workshop} os levou a sérias discussões sobre
usar um jogo de futebol para promover ciência e tecnologia. Estudos foram feitos para
analisar a viabilidade dessa ideia. Os resultados desses estudos mostram que
a ideia era viável, desejável e englobava diversas aplicações práticas. Em 1993, um
grupo de pesquisadores, incluindo Minoru Asada, Yasuo Kuniyoshi e Hiroaki Kitano,
lançaram uma competição de robótica chamada de Robot \textit{J-League} (fazendo uma analogia à
\textit{J-League}, nome da Liga Japonesa de Futebol Profissional). Em um mês, vários
pesquisadores já se pronunciavam dizendo que a iniciativa deveria ser estendida ao
âmbito internacional. Surgia então, a \textit{Robot World Cup Initiative} (RoboCup).

RoboCup é uma competição destinada a desenvolver os estudos na área de robótica e
Inteligência Artificial (IA) por meio de uma competição amigável. Além disso, ela tem
como objetivo, até 2050, desenvolver uma equipe de robôs humanoides totalmente
autônomos capazes de derrotar a equipe campeã mundial de futebol humano. A competição
possui várias modalidades. Neste trabalho, será analisada a \textit{Small Size Robot League} (SSL),
também conhecida como F180. De acordo com as regras da SSL, as equipes devem ser
compostas por 6 robôs, sendo um deles o goleiro, que deve ser
designado antes do início do jogo. Durante o jogo, nenhuma interferência humana é
permitida com o sistema de controle dos robôs. É fornecido aos times um sistema de
visão global e esses controlam seus robôs com máquinas próprias. O sistema de controle
dos robôs geralmente é externo e recebe os dados de um conjunto de duas câmeras
localizadas acima do campo. Esse sistema de controle processa os dados, determina qual comando deve ser executado por
cada robô e envia este comando através de ondas de rádio aos robôs. Embora seja
permitido que as equipes utilizem sistemas próprios de visão, a maioria das
equipes utiliza a visão centralizada. A figura~\ref{fig:robocup2013} mostra uma
imagem da SSL Robocup 2013, da qual a RoboIME (Equipe de Futebol de Robôs do
Laboratório de Robótica do IME) participou.

\section{Motivação}

O futebol de robôs, problema padrão de investigação internacional, reúne grande parte
dos desafios presentes em problemas do mundo real a serem resolvidos em tempo real.
As soluções encontradas para o futebol de robôs podem ser estendidas, possibilitando
o uso da robótica em locais de difícil acesso para humanos, ambientes insalubres e
situações de risco de vida iminente.

Há diversas novas áreas de aplicação da robótica, tais como exploração espacial e submarina,
navegação em ambientes inóspitos e perigosos, serviço de assistência médica
e cirúrgica, além do setor de entretenimento. Essas áreas podem ser beneficiadas com o
desenvolvimento de sistemas
multi robôs. Nestes domínios de aplicação, sistemas de multi robôs deparam-se sempre
com tarefas muito difíceis de serem efetuadas por um único robô. Um time de robôs pode
prover redundância e contribuir cooperativamente para resolver o problema em questão.
Com efeito, eles podem resolver o problema de maneira mais confiável, mais rápida e
mais econômica, quando comparado com o desempenho de um único robô.

% TODO Incluir imagem do robô
%\begin{figure}
%  \includegraphics[width = \pagewidth]{}
%\end{figure}

% Devido à grande complexidade do problema de interação com humanos, faz-se necessário
% que os robôs sejam dotados de uma capacidade de aprendizado para facilitar a interação
% desses com o mundo real. Isso é relevante tanto para aplicações industriais, quanto para
% aplicações em resgates e militares. Isso diminui a necessidade de modelagem
% exata dos ambientes em que os sistemas robóticos serão introduzidos e permite que
% a adaptação a ambientes complexos seja realizada através da exposição destes sistemas
% às possíveis situações de trabalho. Por meio da incorporação do sistema de
% aprendizagem, situações não consideradas podem ser incorporadas ao algoritmo de
% controle dos robôs dinamicamente. Isso permitiria que esses reagissem de maneira mais
% eficiente em futuras situações semelhantes.

Devido a alta complexidade de um jogo de futebol de robôs juntamente com a
necessidade de se refazer constantemente o planejamento das ações a serem tomadas
pelos robôs, é extremamente desejável evitar expandir nós da árvore de
planejamento que não representem o modelo de reação do adversário. Isso
aumentaria a eficiência no uso dos recursos de planejamento, permitindo
antecipar mais jogadas em um menor intervalo de tempo. Logo, faz-se necessário
pesquisar heurísticas que façam isso da maneira mais eficiente e controlada
possível.

\section{Objetivo}

Este trabalho objetiva estudar os algoritmos da Otimização da Colõnia de Formigas (ACO),
Recozimento Simulado (SA), Algoritmo Genético (GA), Rede Neural (ANN) assim como métodos
da logica fuzzy. Isso objetiva analisar se e como a inteligência
artificial de um time de futebol de robôs pode ser modelada utilizando esses
algoritmos e métodos com base nas informações contidas nos $logs$ (definido a seguir)
de um jogo da SSL da RoboCup. Isso tem o objetivo de permitir que agentes controláveis
possam reagir de maneira eficiente às ações dos agentes do time adversário, que
não são controláveis. A pesquisa se propõe a realizar esse aprendizado dos
agentes adversários baseado em gravações coletadas dos pacotes da
\textit{SSL-Vision} e do \textit{Referee-Box} durante jogos, também conhecidas
como \textit{logs}, para posteriormente serem incorporados ao sistema de inteligência da
equipe de futebol de robôs do Laboratório de Robótica, denominada RoboIME\@. Apesar
de existirem até a data deste trabalho vários artigos publicados relacionados a
SSL, os autores sentiram a necessidade de modelarem de uma maneria mais rigorosa
certos conceitos que surgiram ao longo da pesquisa. Isso com o objetivo de se
compreender melhor o problema a ser resolvido.
% NOTE: acho que não é o caso disso:
%Se possível, deseja-se que essa modelagem/aprendizagem seja feita dinamicamente durante a partida da SSL.

\section{Justificativa}% TODO Maj Duarte colocou um certo nessa encestação

Um método concreto que possa prever o comportamento de agentes inteligentes de um jogo de
futebol de robôs permite com que seja possível prever o comportamento de um time adversário.
Com tal mecanismo é possível melhorar a Inteligência Artificial em uso pela RoboIME
para tomar decisões que levem a resultados melhores e, por consequência, ganhar mais partidas.
Outras equipes participantes da SSL já utilizam mecanismos de predição do adversário.
Portanto, é de grande importância desenvolver também tal mecanismo para acompanhar a evolução
das tecnologias envolvidas.

\section{Metodologia}

Para atingir os objetivos propostos, será seguida a seguinte metodologia:
Inicialmente o problema a ser investigado será definido formalmente,
utilizando definições e teorias apropriadas.

% Posteriormente, a bibliografia é revisada e são evidenciados os métodos comumente
% utilizados para a abordagem do problema mais geral de classificação, bem como são
% analisados trabalhos aplicados especificamente à SSL.

A seguir, são analisadas as heurísticas levantadas durante a
revisão da bibliografia. Cada uma delas é descrita. Após descrição sumária
de cada heurística, são apresentados os respectivos exemplos de cada aplicação.

Então, são apresentadas abordagens envolvendo as heurísticas introduzidas
anteriormente. Cada abordagem relaciona, no mínimo, uma dessas heurísticas.
É apresentada uma tabela comparativa que evidencia as principais diferenças
entre cada abordagem.

% NOTE: não é bem verdade isso:
%Para validar os estudos desenvolvidos, o projeto de um software
%que analise os dados dos \textit{logs} disponíveis no site da SSL é apresentado.

\section{Estrutura do Trabalho}

No capítulo~\ref{cap:def_problema}, o problema a ser resolvido é definido.
No capítulo~\ref{cap:heuristicas}, são apresentadas as
heurísticas comumente utilizadas em problemas de classificação.
No capítulo~\ref{cap:anal_abordagens}, são descritas possíveis abordagens a serem
seguidas para resolução do problema.
No capítulo~\ref{cap:conclusao}, são apresentadas as principais conclusões
atingidas neste trabalho.


% Desenvolvimento
% ---------------

% TODO[vbramigk]: definição do problema (problema+tema)
\chapter{Especificação dos Conceitos Envolvidos na SSL}\label{cap:def_problema}

\section{KDD}

De acordo com \cite{passos2005datamining}, a área responsável por estudar o processo
de extração de informação com base em um conjunto de dados é chamada de Descoberta
de Conhecimento em Bases de Dados, denominada KDD (\textit{Knowledge Discovery in
Databases}). As etapas operacionais da KDD são:

\begin{itemize}
  \item \textit{Pré-processamento}: etapa relacionada à captação,
        organização e ao tratamento dos dados, com o objetivo de
        preparar os dados para a etapa de \textit{Mineração de Dados};
  \item \textit{Mineração de Dados}: etapa relacionada à busca por
        conhecimentos úteis no conjunto de dados analisado;
  \item \textit{Pós-processamento}: etapa relacionada ao tratamento do
        conhecimento obtido na etapa de \textit{Mineração de Dados}, com
        o objetivo de viabilizar a utilização desse conhecimento
        (normalmente esta etapa é desnecessária).
\end{itemize}

% Este trabalho objetiva, através do processo de \textit{Mineração de Dados}, extrair
% informação dos $logs$ (definido a seguir) de um jogo do futebol de robôs. O
% processo de preprocessamento será estudado futuramente. Deseja-se, para facilitar
% a avaliação dos algoritmos a serem desenvolvidos, que o resultado do KDD em questão
% seja pós processado. Esse processo também será estudado futuramente. A seção a
% seguir descreve de maneira detalhada o domínio que será estudado.

\section{Domínio}

O problema do time de futebol de robôs que será modelado neste capítulo é baseado
no modelo de planejador apresentado em \cite{zickler}, na teoria de agentes descrita
em \cite{russellnorvig} e na arquitetura de sistemas multiagentes
STP (Skills, Tactics and Plays) descrita em \cite{bowling2003plays}.

\subsection{Arquitetura}

A arquitetura a ser considerada é baseada em \cite{felixnavarro}.
A \textit{RoboCup Small Size League} (SSL) envolve problemas de diversas áreas
da engenharia. Logo, com o objetivo de facilitar a compreensão do
problema, a arquitetura a ser considerada é apresentada na figura
\ref{arquitetura_ssl}. Essa arquitetura é composta pelos seguintes
sistemas:

\begin{itemize}
  \item Câmeras Visão: conjunto de câmeras \textit{firewire} que captura as imagens do
        campo e as envia para a SSL-Vision;
  \item Comunicação: módulo responsável por receber os parâmetros
        dos motores, drible, chute baixo e chute alto e enviar o comando via
        ondas de rádio para os robôs;
  \item Execução: módulo responsável por realizar a tomada de decisões
        em baixo nível de quais ações os robôs devem realizar a partir
        da decisão tomada pelo módulo de inteligência;
  \item Inteligência: módulo responsável por realizar a tomada de
        decisão em alto nível de quais ações os robôs devem realizar
        tendo auxílio de um módulo de Simulação;
  \item Mundo Real: campo de futebol real, onde os times 1 e 2 interagem
        através de seus respectivos robôs
  \item Referee-Box: \textit{software} padronizado pela Robocup para que as
        regras da competição sejam cumpridas sem que haja intervenção
        humana excessiva durante uma partida;
  \item Simulação: módulo do \textit{software} do time responsável por simular
        o ambiente da partida, tendo como entrada os parâmetros do mundo
        real;
  \item \textit{Software} Time 1/2: \textit{software} do time 1/2;
  \item SSL-Vision: \textit{software} padronizado pela Robocup que permite a
        integração com um conjunto de câmeras \textit{firewire} que capturam
        imagens do campo e as processa, extraindo informações sobre os objetos na
        imagem;
  \item Time 1/2: time de robôs que executa os comandos recebidos pelo
        sistema de transmissão do time 1/2;
  \item Transmissão 1/2: sistema de transmissão do time 1/2;
  \item World Model: módulo responsável por modelar o mundo e dar
        confiabilidade aos dados que serão enviados ao módulo de
        Inteligência e são oriundos do Referee-Box e SSL-Vision.
\end{itemize}

\begin{landscape}
  \begin{figure}[thpb]
    \centering
    \includegraphics[width=20cm]{imgs/arquitetura_ssl}
    \caption{Arquitetura básica da SSL}
    \label{arquitetura_ssl}
  \end{figure}
\end{landscape}

Apesar de modelar a maioria dos sistemas empregados atualmente na
SSL, existem alguns problemas na arquitetura descrita anteriormente. Entretanto,
são necessárias algumas definições apresentadas nas próximas secções para que eles possam ser discutidos.

\subsection{Definições}

\begin{defi}[Corpo Rígido]
  Um corpo rígido $r$ é definido por dois subconjuntos disjuntos
  de parâmetros $r= \langle \hat{r}, \bar{r} \rangle$ em que:
  \begin{itemize}
    \item $\hat{r} = \langle \alpha, \beta, \gamma, \omega \rangle$,
    que são os parâmetros de estado mutaveis, respectivamente:
    posição ($\mathbb{R} ^{3}$), orientação($\mathbb{R} ^{3}$),
    velocidade linear($\mathbb{R} ^{3}$), velocidade angular
    ($\mathbb{R} ^{3}$)

    \item $\bar{r} :$ parâmetros imutáveis corpo que descrevem sua
    natureza fixa e que permanece constante ao longo do curso de 
    planejamento.
  \end{itemize}

  Exemplos de parâmetros considerados nesta modelagem imutaveis são:
  coeficiente de atrito estático e dinâmico, descrição $3D$ do corpo
  (por exemplo, por meio de um conjunto de primitívas $3D$), centro de
  massa no referencial do corpo, coeficiente de restituição,
  coeficiente de amortecimento linear e angular. Note que a matrix de
  rotação $R\in\mathbb{R}^{3\times 3}$ gerada a partir da rotação
  em torno de um vetor unitário direção $d\in\mathbb{R}^{3}$  de
  $\theta \in \mathbb{R}$ radianos satisfaz a equação
  $R = exp\left( \beta \right)$ (em que $\beta = d. \theta \in \mathbb{R} ^{3}$)
  \cite{math2robotics}.
\end{defi}

\begin{defi}[Bola]\label{def:bola}
  Bola é um corpo rígido $\hat{b}$, no qual somente as componentes 
  $\langle x,y \rangle$ do o parâmetro $\hat{b}.\alpha$ são
  observáveis. De acordo com a
  aquitetura considerada para a SSL descrita na figura
  \ref{arquitetura_ssl}, tem-se que, a partir de uma sequênicia
  de quadros, é possível obter um valor estimado para o parâmetro
  $\hat{b}.\gamma$ a partir do intervalo entre os dados recebidos
  da \textit{SSL-Vision} e da equação $ \gamma \approxeq 
  \frac{\Delta \alpha}{\Delta t} $. Entretanto, uma vez que a componente
  $z$ de $\hat{b}.\alpha$ não é observável, $\hat{b}.\gamma.z$ 
  não pode ser estimada a partir do intervalo entre os dados recebidos
  da \textit{SSL-Vision}. Semelhantemente,  uma vez que o
  parâmetro $\hat{b}.\beta$ também não pode ser observado,
  não se pode estimar o valor de $\hat{b}.\omega$ com exatidão.
\end{defi}

\begin{defi}[Robô]
  Robô $rob$ é um conjunto de sistemas compostos de corpos rígidos,
  \textit{hardware} e \textit{firmware}. São eles:

  \begin{itemize}
    \item Drible: imprime um torque a bola;
    \item Chute baixo: imprime uma força à bola $\hat{b}$
          e, possivelmente, um torque, com o objetivo de
          alterar as componentes $\langle x,y \rangle$
          do parâmetro $\hat{b}.\gamma$;
    \item Chute alto: imprime uma força à bola $\hat{b}$
          e, possivelmente, um torque, com o objetivo de
          alterar as componentes $\langle x,y,z \rangle$
          do parâmetro $\hat{b}.\gamma$, com $\hat{b}.\gamma_z \neq 0$;
    \item Receptor: recebe comandos enviados pelo sistema de
          transmissão de seu respectivo time;
    \item Sistema de movimentação: imprime uma força e um torque
          ao centro de massa global do $rob$;
  \end{itemize}

  Por meio desses sistemas, cada robô $rob$ pode executar um conjunto de ações $A_{rob}$.

  Apesar de o modelo descrito acima abranger a mairia dos
  robôs utilizados atualmente por equipes da SSL, é importante
  ressaltar que o robô pode ter um conjunto de sensores que
  poderiam coletar informações adicionais às transmitidas pela
  \textit{SSL-Vision} juntamente com um sistema de transmissão
  para enviá-las ao software do seu respectivo time. Isso é
  interessante, pois, conforme observado na definição \ref{def:bola},
  o parâmetro $\hat{b}.\beta$ não é observável. Como o sitema de
  drible impõe um torque à bola, por meio de um sensor, é possível
  estimar o valor de $\hat{b}.\omega$. Sem esse sensor, não é possível
  prever com exatidão a tragetória da bola somente com informações de
  simulação ou da visão.

\end{defi}

\begin{defi}[Time]\label{def:time}
  Sejam os seguintes parâmetros:

  \begin{description}
    \item $Rob_c$ o conjunto dos robôs controlados;
    \item $Rob_i$ o conjunto dos robôs inimigos, isto é, não controlados;
    \item $X$ o espaço de estado de todos os corpos rígidos envolvidos na partida considerada;
    \item $x_{init} \in X$ o estado inicial;
    \item $X_{goal}\subset X$ o conjunto de estados objetivo;
    \item $x_{ob}^{i}$ os estados observados pelo módulo \textit{SSL-Vision} no instante $i$;
    \item $X_{ob}^{i} =  \lbrace{x_{ob}^{0} = x_{init},...,x_{ob}^{i}}\rbrace$;
    \item $Sk \subset A_{rob}$ um conjunto de \textit{skills};
    \item $prob: X \longrightarrow [0,1]$ uma distribuição de probabilidade, cujo argumento é
          $x \in X$;
    \item $tk = G(V \in Sk, E \in {prob} )$ o conjunto de todas as táticas possíveis
          formadas a partir de grafos orientados, em que os vértices são \textit{skills} $sk \in Sk$
          e as arestas são $prob$ associadas a possibilidade de ocorrerem as transições
          entre uma skill e outra;
    \item $A_c = A_{rob 1} \cup ... \cup A_{rob n_c}$ o conjunto das ações possíveis de $Rob_c$;
    \item $A_i = A_{rob 1} \cup ... \cup A_{rob n_i}$ o conjunto das ações possíveis de $Rob_i$;
    \item $A = A_c \cup A_i$ o conjunto das ações possíveis de $Rob_c \cup Rob_i$;
    \item $e$ a função de transição de estado que pode aplicar uma ação $a\in A$ em um estado particular
          $x \in X$ e computar o estado seguinte $x^{'} \in X$, i.e.:
          $e: \langle x,a \rangle \longrightarrow x^{'}$;
    \item $f_{U}: X \longrightarrow \mathbb{R^{+}} \cup\lbrace 0\rbrace$ uma função utilidade tal que
          $f_{U}(x)$ mede a utilidade do estado $x \in X$ um entre estados do mundo dado os estados;
    \item $r_i: \langle A_i, X_{ob}^{i}\rangle \longrightarrow a_i^{'}$ o modelo de reação dos robôs
          inimigos dado $X_{ob}^{i}$;
    \item $AB =\lbrace V \subset X, E \subset A\rbrace$ uma árvore de busca;
    \item $e_b: \langle X_{ob}^{i}, e, f_{U}, r_i, AB\rangle \longrightarrow AB^{'}$ uma estratégia de busca.

  \end{description}

  Então, um time $T$ é definido por:
  \[
    T: \langle A, X_{ob}^{i}, e, e_b, r_i \rangle \longrightarrow a_c^{i+1}
  \]
  Assim, utilizando-se de $e$, $T$ simula várias sequência de ações $a$ dado $X_{ob}^{i}$,
  gerando  a partir de $f_{U}$, $e_b$ e.
\end{defi}

\begin{defi}[Partida]
  Dado dois times $T_1$ e $T_2$. Uma partida $p$ é definida por:

  \[
   p = \lbrace T_1, T_2, \Delta t, \delta t, \langle Ref^{0}, X_{ob}^{0}, A_1^{0}, A_2^{0}\rangle, 
    ..., \langle Ref^{N}, X_{ob}^{N}, A_1^{N}, A_2^{N} \rangle \rbrace
 \]

  uma sequência , em que:
  \begin{description}
    \item $\Delta t$ é o tempo de duração da partida;
    \item $\delta t$ é o tempo médio entre cada frame enviado pela \textit{SSL-Vision} ao longo de $\Delta t$;
    \item $N \approxeq \frac{\Delta t}{\delta t}$ é número total de frames enviados pela \textit{SSL-Vision}
  ao longo de $\Delta t$;
    \item $Ref^{i}$ são os comandos enviados pelo módulo \textit{Referee-Box} no instante $i$;
    \item $X_{ob}^{i}$ são os dados enviados pelo módulo \textit{SSL-Vision} no instante $i$;
    \item $A_1^{i}$ são as ações executadas por $T_1$ no instante $i$;
    \item $A_2^{i}$ são as ações executadas por $T_2$ no instante $i$.
  \end{description}
\end{defi}

\begin{defi}[Logs]
  Dada uma partida $p$. O $log$ de $p$ é definidor por:

  \[
    log(p) = \lbrace p.\langle Ref^{0}, X_{ob}^{0}\rangle, ..., p.\langle Ref^{N}, X_{ob}^{N}\rangle \rbrace
  \]
\end{defi}

%\section{Enunciado do Problema}
\section{Escopo da Pesquisa}

A partir das definições apresentadas, pode-se enunciar o problema 
abordado por este trabalho. Deseja-se prever $x_{ob}^{i+1}$ dado
$X_{ob}^{i}$. Uma abordagem que será seguida é considerar
o problema como sendo uma tarefa de $classificação$ de KDD\@.
Resta então definir as classes e os atributos de acordo com as definições
apresentadas anteriormente. Assim, a seção a seguir descreve uma maneira
de se modelar essas classes e atributos.

\section{Modelagem por Classificação}

O problema a ser resolvido pode ser
encarado como um problema de classificação temporal. O sistema deve mapear uma
sequência temporal de observações para um conjunto de regras associadas a
subconjuntos de robôs. Essas regras definem o comportamento dos robôs. Assim,
para prever o próximo movimento do adversário, e com isso determinar
$x_{ob}^{i+1}$, pode-se descobrir quais são os robôs do time adversário que
estão bloqueando, quais estão marcando o atacante, quais estão esperando para
receber um passe dentre outras categorias comuns no futebol. A
tabela~\ref{regras} apresenta uma lista de regras comuns em futebol de robôs.

Outro problema relevante é definir o comportamento dos robôs a partir do estado
atual do jogo e das regras associadas aos robôs. Entretanto, devido ao tempo
destinado a esta pesquisa, será modelado somente uma solução para o mapeamento das
observações às regras. Também, o módulo de inteligência necessita dos dados
desse mapeamento para poder prever de maneira coerente estados futuros do jogo. Uma
solução ingênua é utilizar os modelos de táticas e \textit{skills} do próprio time.
Isso não representa a realidade, pois é pouco provável que dois times desenvolvam
todos os algoritmos que controlam os robôs de tal maneira que eles tenham o mesmo
resultado.

\subsection{Regras}

Em princípio cada equipe utiliza seu próprio conjunto de regra, que correspondem
às classes citadas anteriormente.
Na prática, entretanto, devido aos trabalhos publicados na comunidade científica
relacionados ao problema de futebol de robôs, muitas regras são compartilhadas
entre os times. Outra razão plausível é que muitas heurísticas e regras mimicam
as regras dos times de futebol jogados por humanos. Exemplos de regras típicas são
apresentados na tabela~\ref{regras}.

\begin{table}
  \begin{center}
    \begin{tabular}{|c|c|}
      \hline
      Regra                  & Descrição \\
      \hline
      \textit{Kick Off}      & Posicionamento antes do\\
                             & início do jogo\\
      \hline
      Marcar Oponente        & Defesa corpo-a-corpo\\
                             & contra um robô particular\\
      \hline
      Posicionar Para Passe  & Cria aberturas para passes\\
      \hline
      Receber Chute Alto     & Recebe um passe executado\\
                             & através do dispositivo de\\
      & chute alto\\
      \hline
      Posicionar             & Posicionamento no campo para\\
                             & receber um passe\\
      \hline
      \textit{Penalty Kick}  & Executa um chute a gol\\
      \hline
      Barreira               & Forma uma barreira com os \\
                             & jogadores do mesmo time \\
                             & para bloquear chutes\\
      \hline
      \textit{Set-Play-Kick} & Uma \textit{play} coordenada onde\\
                             & um robô passa a bola à outro que \\
                             & chuta de primeira em direção ao gol.\\
      \hline
      Atacante               & Regra de ataque primária\\
      \hline
      Defesa circular        & Defende o gol á um raio fixo\\
      \hline
    \end{tabular}
    \caption{Lista de regras comuns entre as equipes \cite{vail2008crf}}
  \label{regras}
  \end{center}
\end{table}

\subsection{Características}

Comumente informação é adicionada ao modelo através do pré-processamento dos
dados coletados nos sensores. Essas informações são chamadas de características.
Características são funções dos dados coletados nos sensores. Elas são utilizadas
para adicionar informação do domínio ao modelo.

As características tomam a forma:

\begin{gather}
f_i(t,Y,X)
\end{gather}

Os protótipos de características típicos do domínio do futebol de robôs,
conforme \cite{vail2008crf}, são:

\begin{gather}
f_i(t,Y,X)=I(y_t == regra)\\
f_i(t,Y,X)=I(y_{t-1} == regra_1)I(y_t == regra_2)\\
f_i(t,Y,X)=I(y_t == regra).g(t,X)
\end{gather}

Onde a função $I$ só retorna um valor não nulo se seu argumento for verdadeiro e
$g$ retorna um valor real.

A partir desse protótipos, associando cada valor aos robôs presentes
no campo, várias funções características podem ser instanciadas.
Tipicamente chega-se à ordem de 100 mil instâncias \cite{vail2008crf}. Faz-se necessário
selecionar quais funções efetivamente são relevantes para o problema.

\subsection{Seleção de Características}

Apesar de incorporar informação ao modelo, a inclusão de muitas características
pode fazer com que o modelo incorpore informações particulares dos dados de
treinamento. Isso é chamado de \textit{overfitting}. Esse fenômeno reduz a precisão
do modelo final por incorporar dados particulares do conjunto de dados de
treinamento. Devido a isso é necessário selecionar quais características devem
ser incorporadas ao modelo final.

A seleção de características reduz o tamanho do modelo, bem como o custo
computacional da classificação. Isso permite que as regras sejam reconhecidas
em tempo viável para serem utilizadas pelo módulo da inteligência durante o
jogo.


\chapter{Revisão Bibliográfica}\label{cap:rev_bibliografica}


Em \cite{russellnorvig}, \cite{haykin2001redes}, \cite{kosko1997fuzzy}, \cite{passos2005datamining},
\cite{doringo2004ant} e \cite{bertsimas1993simulated} são descritas as heurísticas
mais comumente aplicadas a problemas envolvendo Inteligência Artificial (IA).

Em \cite{yoneyama2004ia} é apresentado uma abordagem de problemas de controle, utilizando de IA. Destaca-se
a abordagem utilizando algoritmos de otimização.

Em \cite{felixnavarro} é apresentada arquitetura para o ambiente da SSL/F180. Essa arquitetura foi base
para a arquitetura descrito no capítulo \ref{cap:def_problema}.

Em \cite{zickler} é apresentado uma modelagem de um planejador robótico baseado em física para ambientes dinâmicos. 
Essa modelagem foi base para a enunciação do problema descrita no capítulo \ref{cap:def_problema}, juntamente com 
arquitetura orientada a \textit{Skills, Tactics and Plays} descrita em \cite{bowling2003plays} e com a teoria dos 
agentes apresentada em \cite{russellnorvig}.

Em \cite{vail2008crf} é apresentada uma modelagem do problema considerado neste trabalho
utilizando-se \textit{Conditional Random Fields}. É interesante resaltar que essa modelagem
foi realizada utilizando-se um time $T_1$ (definição \ref{def:time}) conhecido no qual os dados
de treinamento que foram coletados da partida $p$ sabendo-se $p.A_1^{i}$, e não somente $X_{ob}^{i}$.

Em \cite{sheng2005motionprediction} é apresentada uma modelagem considerando apenas $X_{ob}^{i}$
de um time $T$. Essa modelagem utiliza redes neurais de três camadas. Foi utilizada
uma função de ativação sigmoidal para a camada oculta e linear para a função de 
ativação da saída. O algoritmo de treinamento utilizado na rede neural foi o \textit{standard back-propagation
algorithm}, descrito em \cite{haykin2001redes}.


\chapter{Heurísticas para Solução do Problema}\label{cap:heuristicas}

\section{Introdução}

Conforme descrito no capítulo anterior, o problema a ser resolvido será
encarado como um problema de classificação temporal. O sistema deve mapeiar uma
sequência temporal de observações para um conjunto de regras associadas a
subconjuntos de robôs. Essas regras definem o comportamento dos robôs.

Outro problema relevante é definir o comportamento dos robôs a apartir do estado
atual do jogo e das regras associadas aos robôs. Entretanto, devido ao tempo
destinado a esta pesquisa será modelado somente uma solução para o mapeamento das
observações às regras. Entretanto, o módulo de inteligência necessita dos desse
mapeamento para poder de maneira coerente estados futuros do jogo. Uma solução
ingênua é utilizar os modelos de tática e \textit{skill}.

\subsection{Regras}

Em princípio cada
equipe utiliza seu próprio conjunto de regras. Na prática, entretanto, devido
aos trabalhos publicados na comunidade científica relacionados ao problema de
futebol de robôs, muitas regras são compartilhadas entre os times. Exemplos
de regras típicas são:


\begin{table}
  \begin{center}
    \begin{tabular}{|c|c|}
      \hline
      Regra                  & Descrição \\
      \hline
      \textit{Kick Off}      & Posicionamento antes do\\
                             & início do jogo\\
      \hline
      Marcar Oponente        & Defesa corpo-a-corpo\\
                             & contra um robô particular\\
      \hline
      Posicionar Para Passe  & Cria aberturas para passes\\
      \hline
      Receber Chute Alto     & Recebe um passe executado\\
                             & através do dispositivo de\\
      & chute alto\\
      \hline
      Posicionar             & Posicionamento no campo para\\
                             & receber um passe\\
      \hline
      \textit{Penalty Kick}  & Take a penalty kick\\
      \hline
      Barreira               & Forma uma barreira com os \\
                             & jogadores do mesmo time \\
                             & para bloquear chutes\\
      \hline
      \textit{Set-Play-Kick} & Uma \textit{play} coordenada onde\\
                             & um robô passa a bola à outro que \\
                             & chuta de primeira em direção ao gol.\\
      \hline
      Atacante               & Regra de defesa primária\\
      \hline
      Defesa circular        & Defende o gol á um raio fixo\\
      \hline
    \end{tabular}
  \caption{Lista de regras comuns entre as equipes}
  \label{regras}
  \end{center}
\end{table}


\subsection{Características}

Comumente informação é adicionada ao modelo através do pré-processamento dos
dados coletados nos sensores. Essas informações são chamadas de características.
Características são funções dos dados coletados nos sensores. Elas são utilizadas
para adicionar informação do domínio ao modelo.

As características tomam a forma:

\begin{centering}
\begin{equation}
f_i(t,Y,X)
\end{equation}
\end{centering}

Os protótipos de características típicos do domínio do futebol de robôs são:

\begin{centering}
\begin{eqnarray}
f_i(t,Y,X)=I(y_t == regra)\\
f_i(t,Y,X)=I(y_{t-1} == regra_1)I(y_t == regra_2)\\
f_i(t,Y,X)=I(y_t == regra).g(t,X)\\
\end{eqnarray}
\end{centering}

Onde a função $I$ só retorna um falor não nulo se seu argumento for verdadeiro e
$g$ retorna um valor real.

A partir desse protótipos, associando cada valor aos robôs presentes
no campo, várias funções características podem ser instânciadas.
Tipicamente chega-se à ordem de 100 mil instâncias. Faz-se necessário
selecionar quais funções efetivamente são relevantes para o problema.

\subsection{Seleção de Características}

A pesar de incorporar informção ao modelo, a inclusão de muitas características
pode fazer com que o modelo incorpore informações particulares dos dados de
treinamento. Isso é chamado de \textit{overfitting}. Esse fenômeno reduz a precisão
do modelo final por incorporar dados particulares do conjunto de dados de
treinamento. Devido a isso é necessário selecionar quais características devem
ser incorporadas ao modelo final.

A seleção de características reduz o tamanho do modelo, bem como o custo
computacional da classificação. Isso permite que as regras sejam reconhecidas
em tempo viável para serem utilizadas pelo módulo da inteligência durante o
jogo.

\subsection{Modelagens}

Várias estruturas são utilizadas em problemas de classificação. Dentre elas:
Hidden Markov Models (HMM), i\textit{Conditional Random Fields} (CRF), Redes Neurais Artificiais (RNA), Support-Vector Machines.

As modelagens estudadas nesse trabalho serão as RNA e CRF.

\section{Conditional Random Fields}

\textit{Conditional Random Fields}, ou CRFs, são modelos de grafos não direcionados
utilizados para classificação estruturada. Os vértices são compostos por
observações, $X$, ou classes, $Y$.

A representação um dos aspectos fundamentais no problema de classificação de
dados sequenciais. Claramente, um modelo que suporta inferência tratável é
necessário, mas um modelo que representa os dados sem fazer suposições de
dependências não garantidas também é desejável. Um jeito de satisfazer ambos
os critérios é usar um modelo que define uma probabilidade condicional
$p(Y|x)$

Um caso particular dos CRF é quando as classes formam uma cadeia linear.
Isso pressupõem que a classificação atual só depende imediatamente da
anterior.

%
% Tradução do artigo: "Conditional Random Fields: An Introduction"
%

\subsection{Rotulamento de dados sequenciais}

A tarefa de rotular sequências para um conjunto de sequências de observações
surge em vários campos, incluindo bioinformática, linguística computacional
e reconhecimento de fala.

Um dos métodos mais comuns para executar tal tarefa de rotulamento e segmentação
é o emprego de \textit{Hidden Markov Models} (HMMs) ou autómato de estado finito
probabilístico para identificar a sequência de rótulos mais provável para qualquer
sequência de palavras dada. HMMs são uma forma de modelos generativa, que define
a distribuição de probabilidade conjunta $p(X,Y)$, onde $X$ e $Y$ são variáveis
aleatórias variando ao longo das sequências de observações e as correspondentes
sequências de rótulos. Com o objetivo de definir uma probabilidade conjunta dessa
natureza, os modelos generativos precisam enumerar todas as possíveis sequências de
observações - uma tarefa que, para alguns domínios, é intratável ao menos que os
elementos da observação forem representados por unidades isoladas, independente dos
outros elementos em uma sequencia de observações. Mais precisamente, um elemento
de observação em um dado instante no tempo só pode depender diretamente do estado,
ou rótulo, naquele tempo. Essa é uma suposição apropriada para poucos conjuntos de
dados simples, entretanto a maioria das sequencias de observação do mundo real são
bem representadas em termos de \textit{features} interativos múltiplos e dependências
de longo prazo elementos de observação.

Essa questão de representação é um dos problemas mais fundamentais do rotulamento de
dados sequenciais. Claramente, um modelo que suporte \textit{tractable inference} é
necessário, entretanto um moledo que represente os dados sem fazer suposições não
garantidas de independências também é desejável. Uma maneira de satisfazer ambos os
critérios é usar um modelo que define a probabilidade condicional $p(Y|x)$ ao longo
das sequências rotuladas dada uma sequência de observações particular $x$, ao invés de
um a distribuição conjunta ao longo de ambos as sequencias de rótulos e observações.
Modelos condicionais são usados para rotular uma sequência nova de observações $x_*$
selecionando uma sequência de rótulos $y_*$ que maximize a probabilidade condicional
$p(y_*|x_*)$. A natureza condicional de tais modelos significa que nenhum esforço é
perdido na modelagem das observações, e não é necessário fazer suposições não
garantidas de independência sobre o modelo. Atributos arbitrários dos dados observados
podem ser capturados, sem a necessidade do modelador ter que se preocupar como tais
atributos estão relacionados.

Campos aleatórios condicionais (CRFs) são \textit{frameworks} probabilísticas para
rotular e segmentar dados sequencias, baseado na abordagem condicional descrita no
parágrafo anterior. Um CRF é uma forma de modelo de grafo não direcionado que define
uma única distribuição \textit{log-linear} sobre sequências de rótulos dada uma
sequência de observações. A vantagem primária dos CRFs sobre os HMMs é a natureza
condicional, resultando em um \textbf{relachamento} das suposições de independência
necessárias pelos HMMs para garantir \textit{tractable inference}. Adicionalmente,
CRFs evitam o \textit{bias} de rotulamento, um problema dos modelos de máxima entropia
de Markov (MEMMs) e outros modelos de Markov condicionais baseados em modelos de grafos
direcionados. CRFs supera ambos os MEMMs e HMMs no número de tarefas de rotulamento de
sequências.

\subsection{Modelos Gráficos Não Direcionados}

Um CRF pode ser visto como um modelo gráfico não direcionado, ou um campo aleatório
de Markov, condicionado globalmente em \textbf{X}, a variável aleatória representando
as sequências de observações. Formalmente, define-se $G=(V,E)$ como sendo um
grafo não direcionado tal que exista um nó $v \in V$ correspondente a cada uma das
variáveis aleatórias representando um elemento $Y_v \in Y$. Se cada uma das variáveis
aleatórias $Y_v$ obedece a propriedade de Markov com relação a $G$, então $(Y,X)$ é
um CRF\@. Na teoria a estrutura do grafo $G$ pode ser arbitrária, desde que ele represente
as independências condicionais da sequência de rótulos sendo modelada. Entretanto,
ao se modelar as sequências, o grafo mais comum e simples encontrado é aquele cujos nós
correspondentes aos elementos de $Y$ formam uma cadeia linear de primeira ordem, conforme
ilustrado na figura~\ref{grafo_crf}.

<<incluir imagem>>

caption: Estrutura gráfica de CRFs de com estruturas em cadeia para sequências.
As variáveis correspondentes aos nós em braco não são geradas pelo modelo.

\subsection{Funções Potencias}

A estrutura gráfica de um CRF pode ser utilizada para fatorar a distribuição conjunta
ao longo dos elementos $Y_v$ de $Y$ em um produto normalizado de funções potencias de
valor real derivadas da noção de independência condicional. Cada função potencial
opera em um subconjunto das variáveis aleatórias representadas pelo vértices em $G$.
De acordo com a definição de independência condicional para um modelos gráficos não
direcionados, a ausência de uma aresta entre dois vértices em $G$ implica que as variáveis
aleatórias representadas por esses vértices são condicionalmente independentes dado todos
as outras variáveis aleatórias no modelo. A função potencial precisa garantir que é
possível fatorar a probabilidade conjunta de tal maneira que variáveis aleatórias
independentes não apareçam na mesma função potencial. A maneira mais fácil de satisfazer
esse requisito é impor à cada função potencial operar em um conjunto de variáveis aleatórias
cujos vértices correspondentes formam uma clique maximal em $G$. Isso garante que nenhuma
função potencial refira a qualquer par de variáveis aleatórias cujos vértices não são
diretamente conectados e, se dois vértices aparecem juntos em um clique essa relação é
explicitada. No caso de um CRF de com estruturas em cadeia, igual ao representado na
figura~\ref{grafo_crf}, cada função potencial irá operar em pares de variáveis rótulo adjacentes
$Y_i$ e $Y_{i+1}$.

É interessante notar que uma função potencial isolada não tem uma interpretação probabilística
isolada, entretanto representa restrições nas configurações das variáveis aleatórias nas quais
a função esta definida. Isso afeta a probabilidade de configurações globais - uma configuração
global com uma probabilidade maior tem mais chance de satisfazer mais restrições que uma
configuração global com baixa probabilidade.

\subsection{Modelagem Matemática}

Lafferty \textit{et al.} \cite{Lafferty} define a probabilidade de uma sequência particular
de rótulos $y$ dado a sequência de observações $x$ como sendo o produto de funções potenciais,
cada uma da forma:

\begin{eqnarray}
  exp(\sum_j \lambda_j t_j(y_{i-1},y_i,x,i) + \sum_k \mu_k s_k(y_i,x,i)),\label{crf_eq}
\end{eqnarray}

onde $t_j(y_{i-1},y_i,x,i)$ é uma \textit{feature transition function} do label em posições
$i$ e $i-1$ na sequência de rótulos; $s_k(y_i,x,i)$ é uma \textit{state feature function} do
rótulo na posição $i$ e da sequência de observação; e $\lambda_j$ e $\mu_k$ são parâmetros a
serem estimados dos dados de treinamento.

Quando se define as \textit{features function}, constrói-se um conjunto de funções com
contradomínio $R$ da forma $b(x,i)$ que extraem as \textit{features} das observações para
expressar alguma característica da distribuição empírica dos dados de treinamento que deveria
valer também para a distribuição do modelo. Um exemplo de tal \textit{feature} é:

\begin{eqnarray}
  b(x,i)=
  \left\{
    \begin{array}{rc}
      1, & \mbox{se a bola esta mais perto do time adversário}\\
      0, & \mbox{caso contrário}.
    \end{array}
  \right.
\end{eqnarray}

Cada \textit{feature function} pega um valor de uma das funções $b(x,i)$ se o estado atual
(no caso da função de estado) ou estado anterior e atual (no caso da função de transição)
assumam valores particulares. Todos as \textit{feature functions} são portanto de contradomínio
real. Por exemplo, considere as seguintes funções de transição:

\begin{eqnarray}
  t_j(y_{i-1},y_i,x,i)=
  \left\{
    \begin{array}{rc}
      b(x,i), & \mbox{se $y_{i-1}=IN$ e $y_i=NP$}\\
      0,      & \mbox{caso contrário}.
    \end{array}
  \right.
\end{eqnarray}

A seguinte notação simplificada será utilizada:

\begin{eqnarray}
  s(y_i,x,i)=s(y_{i-1},y_i,x,i)
\end{eqnarray}
e
\begin{eqnarray}
  F_j(y,x) = \sum_{i=1}^n f_j(y_{i-1},y_i,x,i),
\end{eqnarray}

Onde cada $f_j(y_{i-1},y_i,x,i)$ e ou uma função de estado $s(y_{i-1},y_i,x,i)$
ou uma função de transição $t(y_{i-1},y_i,x,i)$. Isso permite que a probabilidade
do rotulamento de uma sequência $y$ dado uma sequência de observações $x$ ser escrita
como:

\begin{eqnarray}
  p(y|x,\lambda) = \frac{1}{Z(x)} exp(\sum_j \lambda_j F_j(y,x)),
\end{eqnarray}

Onde $Z(x)$ é um fator de normalização.

\subsection{Máxima Entropia}

A forma de um CRF,mostrada em~\ref{crf_eq}, é fortemente motivada pelo princípio da
máxima entropia - um \textit{framework} para estimar distribuições de probabilidade
de um conjunto de dados de treinamento. A entropia de um conjunto de probabilidades
é uma medida de incerteza e é utilizada quando a distribuição em questão é a mais
uniforme possível. O princípio da máxima entropia afirma que a única distribuição
de probabilidade que pode justificavelmente ser construída de informação incompleta,
tal como um conjunto de dados finitos, é aquela que tem entropia máxima sujeita a
um conjunto de restrições que representam a informação disponível. Qualquer outra
distribuição vai envolver suposições não garantidas.

\subsection{Parâmetro de Inferência}

Assumindo que os dados de treinamento $\lbrace (x^{(k)},y^{(k)}) \rbrace$
são independentemente e identicamente distribuídos, o produto de~\ref{crf_eq}
sobre todas as sequências de treinamento, como uma função de parâmetro $\lambda$,
é conhecido como \textit{likelihood}, denotado por $p(\lbrace y^{(k)}\rbrace,\lbrace
x^{(k)}\rbrace, \lambda)$. Treinamentos que maximizam o \textit{likelihood} escolhem
valores de parâmetros tais que o logaritmo do \textit{likelihood}, também chamada de
\textit{log-likelihood}, seja máximo. Para um CRF, o \textit{log-likelihood} é dada por:

\begin{eqnarray}
  \mathcal{L}(\lambda) = \sum_k \left[ log\frac{1}{Z(x^{(k)})}+
    \sum_j \lambda_j F_j(y^{(k)},x^{(k)}) \right]
\end{eqnarray}

Não é possível determinar analiticamente a os valores dos parâmetros que maximizam
o \textit{log-likelihood}. Logo, para resolver essa equação dois métodos foram estudados
neste trabalho: \textit{ant colony optimization} e \textit{simulated annealing}.
Esse métodos são descritos nas próximas seções.

\section{Otimização da Colônia de Formigas}

Na busca por alimento, as formigas utilizam de feromônios para encontrar o melhor caminho.
Isso acontece da seguinte maneira: cada formiga deposita feromônio ao se deslocar. A partir
da avaliação da quantidade de feromônio depositada por formigas que já passaram pelo local,
formigas subsequentes tem mais probabilidade de se mover em rotas que tem mais feromônios. Ao
decorrer do tempo os feromônios vão evaporando, apagando rastros que não foram reforçados. 
Com isso, caminhos que são percorridos por mais formigas tem mais chance de serem 
percorridos por outras formigas do que aqueles que foram percorridos por menos formigas e 
caminhos que foram percorridos á pouco tempo tem mais chance de serem percorridos que caminhos
percorridos a muito tempo. A quantidade de feromônio depositado é mais intensa no trajeto de volta,
quando a comida foi encontrada. Outro fator que é levado em consideração é a qualidade da comida
encontrada, de maneira que mais feromônio é depositado quanto melhor for a fonte de alimento encontrada.
A medida que mais formigas exploram o local e encontram alimento, esse procedimento tende a otimizar o
trajeto entre a fonte de alimento e a colônia.

Apesar dessa heurística utilizada pelas formigas ser interessante para se resolver problemas combinatórios 
do tipo NP(i.e., com complexidade não polinomial), são necessários algumas adaptações na construção
de um algoritmo computacional.

A seguir é apresentado a meta-heurística do ACO(\textit{Ant Colony Optimization}) algoritmo, juntamente com observações relacionadas as diferenças
entre a heurística do ACO e o comportamento natural das formigas descrito anteriormente.

\subsection{Pseudo código da meta-heurística do ACO}
%Algoritmo
\begin{algorithm}[H]
%Macros
\SetKwBlock{AgendarAtividade}{AgendarAtividade}{fim}
\SetKwBlock{Procedimento}{Procedimento}{fim}

\Procedimento{
  \Enqto{$n < N_{MAX\_IT}$}{
    %\tcp*[f]{$N_{MAX\_IT}$ é o número máximo de iterações\\}
    \AgendarAtividade{
      ConstruirSolucoesFormigas\\
      AtualizarFeromonios\\
      %\tcp*[f]{opcional}\\
      \tcp{opcional:}
      AcoesGlobais
    }
  }
}
%}

\caption{Pseudo código da meta-heurística do ACO}\label{meta-heuristica_aco}
\end{algorithm}
%\\
%\\

A meta-heurística do ACO pode ser subdividida em três partes, 
conforme proposto por (Dorigo, Marco; 2004): \textit{ConstruirSolucoesFormigas}, 
\textit{AtualizarFeromonios} e \textit{AcoesGlobais}.

\textit{ConstruirSolucoesFormigas} gerencia a movimentação de uma colônia de formigas
em torno dos nós vizinhos. A escolha do próximo nó é feita através de uma decisão 
estocástica que é função da quantidade de feromônio no nós vizinhos e informação heurística.
Quando uma formiga encontra uma solução, ou enquanto a solução é construída, esta avalia a
qualidade da solução(completa ou parcial) que será utilizada pelo procedimento
\textit{AtualizarFeromonios} para decidir a quantidade de feromônio que será depositada.
Outro procedimento relevante na construção da solução é a eliminação de possíveis ciclos, utilizado
por exemplo, no problema do caixeiro viajante.

\textit{AtualizarFeromonios} é o processo que atualiza os traços de feromônio depositados pelas
formigas no espaço de busca. Os traços de feromônio podem aumentar, caso uma formiga tenha visitado
o nó/conexão em questão, ou diminuir, devido ao processo de evaporação do feromônio. Esse procedimento faz com 
que nós/conexões que foram visitados por muitas formigas ou por uma formiga e que tenha levado em
uma solução boa aumentem a probabilidade de serem visitados por futuras formigas. Semelhantemente, reduz 
a probabilidade de que nós que não foram visitados por novas formigas por muitas iterações sejam visitados
novamente. Logo, este procedimento evita a convergência a caminhos sub ótimos, favorecendo também a exploração
de novas regiões do espaço de busca.

Por fim, o procedimento \textit{AcoesGlobais} é utilizado para centralizar ações que não podem ser executadas
pelas formigas individualmente. Um exemplo de ações desse tipo é a filtragem de soluções ou o favorecimento de
regiões por meio de informações globais.

O procedimento \textit{AgendarAtividade} não necessariamente é uma instrução sequencial. Pode-se, portanto,
implementá-lo de maneira sequencial ou paralela, síncrona ou assincronamente. O tipo de abordagem que será
utilizada depende das características do problema que se deseja resolver.

\section{Recozimento Simulado}

No processo de recozimento de um metal, a quantidade de energia interna livre
esta intrinsecamente relacionada ao processo de resfriamento em que o metal é
submetido. Quanto mais rápido se resfriam um metal mais energia é armazenada
internamente. Isso pode ser explicado considerando que o tempo que a estrutura
leva para atingir o estado de menor energia é maior que o disponível devido
a redução da mobilidade dos átomos com o decaimento da temperatura. Com efeito,
quanto maior a taxa de resfriamento maior o número de defeitos na estrutura do
sólido e menor o tamanho médio dos grãos. Quando se reduz a taxa de resfriamento, 
há uma maior chance de se atingir configurações mais estáveis.
Como resultado, a energia interna é reduzida. De acordo com
\cite{bertsimas1993simulated}, pode-se modelar a probabilidade $p_{ij}$ de uma
configuração atômica $\{r_i\}$ com energia $E\{r_i\}$ passar para a configuração $\{r_j\}$ com energia $E\{r_j\}$ na temperatura $T$ como:

\begin{equation}
\mbox{$p_{ij}$}=\left\{
	\begin{array}{rl}
	1 & \mbox{se $E\{r_j\} \le E\{r_i\}$} \\
	exp\left\{-\frac{(E\{r_j\}-E\{r_i\})}{k_B.T}\right\} & \mbox{se $E\{r_j\} > E\{r_i\}$}
\end{array} \right.
\end{equation}

Onde $k_B$ é a constante de Boltzmann. Para se reduzir a energia livre, é necessário que uma
rotina de resfriamento seja escolhida de acordo com o tipo de material a ser resfriado.

Conforme proposto por Kirikpartrick, Gellett e Vechin (1983) e Cerny (1985), pode-se desenvolver
uma heurística probabilística para se encontrar o mínimo global de uma função custo que possua
vários mínimos locais fazendo-se uma analogia com o fenômeno físico descrito acima. A meta-heurística
induzida por este processo é chamada de meta-heurística \textit{Simulated Annealing} (Recozimento
Simulado), ou SA, apresentada a seguir.

\subsection{Meta-heurística do SA}

De acordo com \cite{bertsimas1993simulated}, os elementos básicos da meta-heurística do SA
para a resolução de um problema combinatório são:

\begin{enumerate}
 \item Um conjunto finito $S$.
 \item Um função custo $J$ de imagem real definida em $S$. Seja $S^* \subset S$ o conjunto de todos os mínimos globais da
 função $J$, suposto subconjunto próprio.
 \item Para cada $i \in S$ um conjunto $S(i) \subset S - \{i\}$, chamado de conjunto dos vizinhos de $i$.
 \item Para cada $i$, uma coleção de coeficientes positivos $q_{ij}$, $j \in S(i)$, tal que $\sum_{j \in S(i)} q_{ij} = 1$.
 \item Uma função não crescente $T: \textbf{N} \rightarrow (0,\infty)$, chamada de rotina de resfriamento. Aqui \textbf{N}
 representa o conjunto de inteiros positivos, e $T(t)$ é chamada de \textit{temperatura} no tempo $t$.
 \item Um estado inicial $x(0) \in S$.
\end{enumerate}

Com base nas definições acima, tem-se o pseudo código para a meta-heurística do
SA apresentado no algoritmo~\ref{lst:meta-heuristica_sa}.

%Algoritmo
\begin{algorithm}
%Macros
\SetKwBlock{Procedimento}{Procedimento}{fim}
\SetKwBlock{EscolherVizinho}{EscolherVizinho}{fim}
\SetKwBlock{CalcTransicao}{CalcTransicao}{fim}

\Procedimento{
  SetarValoresInicias\;
  \Para{$n = 1$ até $N_{MAX\_IT}$ ou $J(x^*) \le TOL$ }{
    \Para{$k = 1$ até $N_{MAX\_IT}$ ou a solução convergir}{
      \EscolherVizinho{
        selecionar algum $j \in S(i)$\;
      }

      \CalcTransicao{
        $\Delta J \leftarrow J(j)-J(i)$\;
        \Se{$Delta J \le 0$}{
          $x(t+1) \leftarrow j$\;
          $x^* \leftarrow j$\;
        }
        \Senao{
          %$q_{ij} \leftarrow exp^{\left\{-\frac{\Delta J}{T(t)} \rigth\}}$\;\\
          $q_{ij} \leftarrow exp^{ -\frac{\Delta J}{T(t)} } $\;
          \lSe{$random() < q_{ij}$}{$x(t+1) \leftarrow j$}
          \lSenao{$x(t+1) \leftarrow i$}
        }
      }
    }
    AtualizarTemperetura\;
  }
}

\caption{Pseudo código da meta-heurística do SA}
\label{lst:meta-heuristica_sa}
\end{algorithm}

No algoritmo~\ref{lst:meta-heuristica_sa}, o procedimento \textit{AtualizarTemperetura} executa a
rotina de resfriamento através da função $T(t)$ definida anteriormente. Já o procedimento
\textit{EscolherVisinho} escolhe aleatoriamente um dos elementos da vizinhança do vértice atual $i$.

\subsection{Exemplo de Aplicação}

O método SA pode ser aplicado para otimizar a busca em árvores de decisão.
Com efeito, devido a complexidade de determinados jogos, não é possível
avaliar todos os possíveis resultados de cada jogada em tempo hábil. A
tabela~\ref{table:games} apresenta a complexidade de alguns jogos de tabuleiro.
Na tabela, \textbf{espaço-de-estado} denota o número de posições legais
atingíveis a partir da posição inicial e \textbf{árvore-do-jogo} denota o
número total de jogos que podem ser jogados, i.e., o número de folhas na árvore
de jogo cuja raiz é a posição inicial do jogo.


\begin{table}
  \begin{center}
    \begin{tabular}{|c|c|c|}
      \hline
                        &                              &                      \\
 \textbf{Jogo} & log(\textbf{espaço-de-estado}) & log(\textbf{árvore-do-jogo})\\
                        &                              &                      \\
      \hline
        Jogo da velha   &             3               &          5           \\
      \hline
        Trilha          &             10              &          50          \\
      \hline
        Oware           &             16              &          32          \\
      \hline
        Pentaminó       &             12              &          18          \\
      \hline
        Lig 4           &             14              &          21          \\
      \hline
        Damas           &             33              &          50          \\
      \hline
        Lines of Action &             24              &          56          \\
      \hline
        Reversi         &             28              &          58          \\
      \hline
        Gamão           &             20              &          144         \\
      \hline
        Quoridor        &             42              &          162         \\
      \hline
        Xadrez          &             46              &          123         \\
      \hline
        Xadrez Chinês   &             52              &          150         \\
      \hline
        Arimaa          &             42              &          190         \\
      \hline
        Shogi           &             71              &          226         \\
      \hline
        Connect6        &             172             &          140         \\
      \hline
        Go              &             171             &          360         \\
      \hline
    \end{tabular}
    \caption{Complexidades do espaço de estado e da árvore do jogo de alguns
             jogos \cite{mertens2006quoridor}}
  \label{table:games}
  \end{center}
\end{table}


%%%%Applying Genetic Algorithms to
%%%Quoridor Game Search Trees for Next-Move Selection
Como o SA é uma otimização de busca local guloso, assumi-se que o adversário
selecionará a melhor jogada de acordo com uma função de avaliação comum aos dois
jogadores. Então a seleção de nós é feita utilizando-se o SA tradicional.

Isso conceitualmente traduz no jogador SA ter olhado para baixo três níveis de
uma determinada posição na árvore de jogo para avaliar a posição atual. A mesma
função heurística de avaliação linear é usada para a avaliação da própria
posição de jogo. No caso do jogo quoridor essa heurística leva em consideração
tanto o número de paredes restantes para cada jogador quanto a distância de cada
jogador ao lado objetivo.

A função objetivo é responsável por determinar a probabilidade de movimentos
piores serem tomados a partir de uma determinada posição. Uma possível função
objetivo é da forma:

\begin{equation}
  exp\left\{\frac{h(atual)- h(vizinho)}{tempo}\right\}
\end{equation}

em que $h$ é uma função de avaliação heurística que é tanto menor quanto melhor
for o estado do jogo. O parâmetro de controle aleatoriedade contra a busca
gananciosa pura é o tempo, e não um constante. Esse é o aspecto do SA que evita
ficar preso em ótimos locais.

Quando se atinge estágios posteriores na árvore de busca jogo onde o jogador
esta perto de seu estado objetivo, a diferença $h(atual)- h(vizinho)$
será relativamente grande (sendo relativamente pequena no início do jogo). Isto
sugere que o uso de um parâmetro constante no lugar do tempo resultaria em um
jogador que seleciona movimentos ruins mais frequentemente perto do fim.
Como o tempo também seria um valor relativamente grande perto do final do jogo,
isso iria compensar essa grande diferença e levar ao estado objetivo mais
rápido \cite{mcdermid2003gaquoridor}.


\section{Logica Nebulosa}

\subsection{Introdução}

Sistemas nebulosos aproximam funções. Eles são aproximadores universais se usarem regras suficientes. 
Neste sentido sistemas difusos podem modelar qualquer função ou sistema contínuos. Aqueles sistemas 
podem vir tanto da física quanto da sociologia, bem como da teoria do controle ou do 
processamento de sinais.

A qualidade da aproximação difusa depende da qualidade das regras. Na prática especialistas sugerem regras
difusas ou aprendem-nas através de esquemas neurais através de dados e ajustam as regras com novos dados.
Os resultados sempre aproximam alguma função não linear desconhecida que pode mudar com o tempo. Melhores 
cérebros e melhores redes neurais resultam em melhores aproximações \cite{kosko1997fuzzy}.

\subsection{Modelo Aditivo Padrão(SAM)}

O sistema difuso $F:\Re^n \rightarrow \Re^p$ é em si uma árvore de regras rasa e extensa. É um aproximador
por antecipação. Existem $m$ regras da forma "Se $X$ é conjunto difuso $A$ então $Y$ é conjunto difuso $B$".
A partir desse nível o sistema depende cada vez menos em palavras. 

Cada entrada $x$ aciona parcialmente todas as regras em paralelo. Então o sistema age como um processador 
associativo a medida que calcula a saída
$F(x)$. 
%Definir a_j(x) e b_j(y)

Essas regras relacionam os conjuntos $A_j$ e $B_j$, gerando o caminho difuso $A_j x B_j$. Na prática,
é utilizado o produto para definir $ a_j x b_j (x,y) = a_j(x).b_j(y)$. Esta é a parte "padrão" no SAM.
A parte "aditiva" se refere ao fato de a entrada $x$ acionar a $j$-ésima regra em um grau $a_j(x)$ e o sistema 
soma os acionamentos ou partes escaladas dos conjuntos escalados $a_j(x)B_j$, conforme \cite{kosko1997fuzzy}:

\begin{eqnarray}
F(x) = \frac{\sum w_i.a_i(x).V_i.c_i}{\sum w_j.a_j(x).V_j}
\end{eqnarray}

Com o volume/área $V_j$ e o centroide $c_j$ são dados por:

\begin{eqnarray}
V_j = \int{b_j(y_1,...,y_p)}_{\Re^{p}}.dy_1...dy_p > 0\\
c_j = \frac{\int{y.b_j(y_1,...,y_p)}_{\Re^{p}}.dy_1...dy_p}{V_j}
\end{eqnarray}

\input{partes/algoritmo_genetico}
\section{Rede Neural}

O termo mais apropriado é rede neural aritificial, já que apenas rede
neural pode se referir ao sistema biológico de nervos, no entando dado o
contexto desse texto e o uso consagrado do termo ``rede neural'', esse
será usado no lugar da versão mais explícita ``rede neural artificial''.

Uma rede neural é um sistema inspirado no sistema nervoso central (em
especial o cérebro) encontrado em muitos animais. A ideia básica é ter
um grafo em que cada nó abstrai um neurônio e é representado como uma
função, alguns desses nós são responsáveis pela observação e outros pela
saída e os nós de entrada alimentam os próximos nós até chegar nos nós
de saída. \cite{haykin2001redes}

\begin{figure}[H]
  \centering
  \includegraphics[width=10cm]{figuras/rede_neural_grafo}
  \caption{Rede neural com três camadas.}\label{fig:rede_neural_grafo}
\end{figure}

A figura \ref{fig:rede_neural_grafo} exemplifica uma rede neural \emph{feedforward},
que é baseada num grafo direcionado acíclico, em que podem ser vistas 3 camadas
a primeira é chamada de camada de entrada, a última, de saída e as intermediárias,
de escondidas. \cite{shiffman2012nature}

Um dos diferencias da rede neural é a capacidade de aprender, essa heurística
forma um sistem adaptativo. Existem três tipos de aprendizados:

\begin{itemize}
\item
  Aprendizado supervisionado: alimentar a rede com um problema cuja a solução é conhecida
  e depois fornecer a resposta certa para que a rede possa se ajustar.
\item
  Aprendizado não supervisionado: consiste em buscar padrões não conhecidos, não se conhece
  a resposta certa ou se uma resposta é certa ou não.
\item
  Aprendizado por reforço: alimentar a rede com um problema cuja a solução pode ser avaliada
  em boa ou má. Esse tipo de aprendizado é comum em robótica onde o robô caminha por um ambiente
  e tem o reforço negativo ou positivo de colodir ou encontrar o objetivo.
\end{itemize}

\subsection{O Neurônio}

O bloco de construção básico de uma rede neural são os neurônios.

\begin{figure}[H]
  \centering
  \includegraphics[width=10cm]{figuras/rede_neural_perceptron}
  \caption{Perceptron de duas entradas e uma saída.}\label{fig:rede_neural_perceptron}
\end{figure}

A figura~\ref{fig:rede_neural_perceptron} mostra um perceptron.

%\ldots{}

%\begin{itemize}
%\itemsep1pt\parskip0pt\parsep0pt
%\item
%  http://en.wikipedia.org/wiki/Neural\_network
%\item
%  http://en.wikipedia.org/wiki/Artificial\_neural\_network
%\item
%  HAYKIN, S. Redes neurais princípios e prática.
%\end{itemize}


\chapter{Análise das Possíveis Abordagens}\label{cap:anal_abordagens}

Uma característica desejada no método a ser empregado para prever próximo
movimento do adversário é não requerer um modelo de classificação. Isso para
deixa a implementação muito complexa, o que dificulta a manutenibilidade do
software final.

Também é desejado que tenha uma estrutura de fácil análise. Essa característica
tem o objetivo de incorporar informação heurística no modelo de modo a facilitar
a modelagem. Apesar de ser desejáda, inicialmente essa característica é menos
importante que a anterior. Entretanto, no longo prazo ela é mais relevante.
Portanto, como o objetivo de se adquirir uma experiencia, essa característica é
menos importante que a anterior.

Um outro problema identificado na análise anterior é a degradação do modelo
devido a discretização. Isso reduz a presição do modelo, então é importante que
o modelo não seja degradado com a discretização.

O pré-processamento é uma tarefa importate do processo de KDD. É desejável que
essa etapa seja a mais fácil possível, para poder agilizar a automação dessa
etapa. No caso dos algorítimos que necessitam de modelo de classificação, fica
evidente que eles necessitam de de um pré-processamento maior, pois esses
modelos necessítam de um conjunto de dados classificados para treinamento.
Isso implica que mais informação que os \textit{logs} será nessária, tornado
esses algorítmos menos adequandos que a ANN.

A etapa de pós-processamento é o processamento que será necessário para que os
dados do algorítmos determinem $x_{ob}^{i+1}$. Essa também é uma etapa na qual
os algorítmos que necessitam de um modelo de classificação são inadequados, pois
é necessario desenvolver outra camada de software para processar a classificação
de cada robô de modo a determinar o próximo movinto dos robôs do time
adversário.

Também é desejável, mas não mandatório, que o modelo treinado possa ser
modificado e reutilizado para modelar diferentes IAs. Isso visa
agilizar a modelagem durante as partidas da SSL da Robocup, pois no intervalo
de um jogo é permitido aos times modificar suas respectivas IAs.

\subsection{Comparação dos Métodos}

Os algorítimos abordados se enquadram em duas classes: algorítmos que são
completos e que necessitam de uma estrutura de classificação que são o ACO, SA,
AG e a algorítmos que são não necessitam, que é o caso da ANN\@. O caso da
lógica fuzzy será discutido na próxima seção com mais detalhes. A
tabela~\ref{table:metodos} apresenta as principais diferenças dos métodos
apresentados anteriormente. Conforme pode ser visto nessa tabela, a rede neural
é a que se mais adequa às caraterísticas desejádas.


\section{Rede Neural}\label{cap:abordagem_rede_neural}

Essa aborgem consiste em usar uma Rede Neural que tem como entrada o estado do
jogo (todas as posições, orientações, velocidades e o comando do juiz) e como
saída o estado do time adversário (todas as posições, orientações e velocidades
dos robôs do time adversário), que visa prever as ações imediatas do adversário.
Essa rede deve ser treinada para prever um time específico usando os
\textit{logs} das partidas do torneio de 2013 da \textit{RoboCup}.

% vim: tw=80


 \begin{table}
   \begin{center}
     \begin{tabular}{|c|c|c|c|c|c|}
       \hline
                         &          &           &              &             &          \\
       Critério          &  ACO     &    SA     & Lógica Fuzzy & Rede Neural & Desejado \\
                         &          &           &              &             &          \\
       \hline                                                                           
                         &          &           &              &             &          \\
       Requer modelo     &   Sim    &    Sim    &     Sim      &    Não      &   Não    \\
       de classificação  &          &           &              &             &          \\
                         &          &           &              &             &          \\
       \hline                                                                           
                         &          &           &              &             &          \\
       Estrutura de      &   Sim    &    Sim    &     Sim      &    Não      &   Sim    \\
       fácil análise     &          &           &              &             &          \\
                         &          &           &              &             &          \\
       \hline                                                                           
                         &          &           &              &             &          \\
       Degradação devido &   Sim    &    Não    &     Não      &    Não      &   Não    \\
       a discretização   &          &           &              &             &          \\
                         &          &           &              &             &          \\
       \hline                                                                           
                         &          &           &              &             &          \\
       Fácil             &   Não    &    Não    &     Não      &    Sim      &   Sim    \\
       pré-processamento &          &           &              &             &          \\
                         &          &           &              &             &          \\
       \hline                                                                           
                         &          &           &              &             &          \\
       Fácil             &   Não    &    Não    &     Não      &    Sim      &   Sim    \\
       pós-processamento &          &           &              &             &          \\
                         &          &           &              &             &          \\
       \hline                                                                           
                         &          &           &              &             &          \\
       Característica    & Modelo   & Parte das & Parte das    & Somente     & Modelo   \\
       reutilizável      & treinado & regras    & regras       & topologia   & treinado \\
                         &          &           &              &             &          \\
       \hline
     \end{tabular}
   \caption{Tabela comparativa dos métodos}
   \label{table:metodos}
   \end{center}
 \end{table}

\section{Lógica Fuzzy}

O método da Lógica Fuzzy, conforme exposto anteriormente, necessita que um
conjunto de regras seja definido. Um exemplo dessas regras foi apresentado na
seção \ref{sec_regras}. Essas regras poder ser geradas a partir de uma
análise mais detalhada do problema, mas também podem ser obtidas utilizando
algum algorítimo de aprendizagem de regras. Para aplicar esse método ao problema
analisado neste trabalho é necessário definir as regras e as distribuições das
variáveis.

A vantagem dos conjuntos difusos é que eles tornam o modelo mais robusto. A
lógica fuzzy tenta melhorar a classificação e os sistemas de decisão.

A principal desvantagem deste método é a modelagem necessária para encaixar os
conceitos descritos acima. Isso, pois o conceito de conjuntos nebulosos ainda
estão em desenvolvimento para o problema abordado neste trabalho. Essa modelagem
não é imediata, pois o problema é de classificação temporal. Não basta que as
características do ambiente sejam associadas aos conjuntos nebulosos de
características. É necessário que regras sejam especificadas estática ou
dinamicamente. No caso estático, elas seriam incorporadas ao modelo através de
especialistas. No caso dinâmico, uma solução é utilizar um classificador para
deduzir as regras.

% XXX: Adicionar classificação temporal e viabilidade
% \begin{table}
%   \begin{center}
%     \begin{tabular}{|c|c|c|c|c|c|}
%       \hline
%                         &      &           &          &              &            \\
%       Critério          & CRF  & CRF + ACO & CRF + SA & Lógica Fuzzy & Rede Neural\\
%                         &      &           &          &              &            \\
%       \hline
%                         &      &           &          &              &            \\
%       Requer modelo     &  X   &     X     &    X     &      X       &      -     \\
%       de classificação  &      &           &          &              &            \\
%                         &      &           &          &              &            \\
%       \hline
%                         &      &           &          &              &            \\
%       Estrutura de      &  X   &     X     &    X     &      X       &      -     \\
%       fácil análise     &      &           &          &              &            \\
%                         &      &           &          &              &            \\
%       \hline
%                         &      &           &          &              &            \\
%       Degradação devido &  -   &     X     &    -     &      -       &      -     \\
%       a discretização   &      &           &          &              &            \\
%                         &      &           &          &              &            \\
%       \hline
%                         &      &           &          &              &            \\
%       Fácil             &  -   &     -     &    -     &      -       &      X     \\
%       pré-processamento &      &           &          &              &            \\
%                         &      &           &          &              &            \\
%       \hline
%                         &      &           &          &              &            \\
%       Fácil             &  -   &     -     &    -     &      -       &      X     \\
%       pós-processamento &      &           &          &              &            \\
%                         &      &           &          &              &            \\
%       \hline
%                         &          &          &           &           &          \\
%       Característica    & Modelo   & Modelo   & Parte das & Parte das & Somente  \\
%       reutilizável      & treinado & treinado & regras    & regras    & topologia\\
%                         &          &          &           &           &          \\
%       \hline
%     \end{tabular}
%   \caption{Tabela comparativa dos métodos}
%   \label{regras}
%   \end{center}
% \end{table}


% ---
% Finaliza a parte no bookmark do PDF, para que se inicie o bookmark na raiz
% ---
\bookmarksetup{startatroot}%
% ---

%\addcontentsline{toc}{chapter}{Conclusão}
\chapter{Conclusão}\label{cap:conclusao}

Este trabalho descreve uma modelagem pesquisada para o jogo de futebol de robôs da
categoria SSL da competição RoboCup e introduz métodos que podem
contribuir para a confecção da solução final. Concluiu-se que nem todas as variáveis
desejadas durante  processo de predição são observáveis, tornando o problema mais
complexo do que realmente se imaginava. A Rede Neural, ACO, SA, \textit{Lógica Fuzzy}
e o AG são abordagens apropriadas para o problema. Para a Rede Neural e \textit{Lógica Fuzzy}
não é necessário supor uma modelagem, uma vez que ela incorpora conhecimento
implicitamente em seus parâmetros. Esse já não é o caso dos métodos restantes, e por isso
foi necessário definir alguns parâmetros inicialmente. As heurísticas de otimização
serão de grande importância para a próxima etapa. Faz-se necessário continuar a
pesquisa para desenvolver os algoritmos que serão efetivamente utilizados.


% ----------------------------------------------------------
% ELEMENTOS PÓS-TEXTUAIS
% ----------------------------------------------------------
\postextual


% ----------------------------------------------------------
% Referências bibliográficas
% ----------------------------------------------------------
%\bibliographystyle{plainnat}%abbrvnat, unsrtnat, apsrev, rmpaps, IEEEtranN, achemso, rsc
\bibliography{referencias}

% ----------------------------------------------------------
% Glossário
% ----------------------------------------------------------
%
% Consulte o manual da classe abntex2 para orientações sobre o glossário.
%
%\glossary

% ----------------------------------------------------------
% Apêndices
% ----------------------------------------------------------

% ---
% Inicia os apêndices
% ---
%\begin{apendicesenv}

% Imprime uma página indicando o início dos apêndices
%\partapendices

% ----------------------------------------------------------
%\chapter{Quisque libero justo}
% ----------------------------------------------------------
%\end{apendicesenv}
% ---


% ----------------------------------------------------------
% Anexos
% ----------------------------------------------------------

% ---
% Inicia os anexos
% ---
%\begin{anexosenv}

% Imprime uma página indicando o início dos anexos
%\partanexos

% ---
%\chapter{Morbi ultrices rutrum lorem.}
% ---

%\end{anexosenv}

%---------------------------------------------------------------------
% INDICE REMISSIVO
%---------------------------------------------------------------------

\printindex

\end{document}
