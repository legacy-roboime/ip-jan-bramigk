\chapter*{Resumo}

A presente pesquisa tem por objetivo fazer um estudo sobre m�todos de armazenamento de informa��es, com enfoque em conectividade alg�brica e seus autovetores na classe das �rvores. Neste trabalho pesquisaremos ferramentas de bancos de dados para armazenar informa��es das �rvores geradas de forma in�dita pelo aluno de gradua��o Thiago. At� o momento n�o foi poss�vel concluir o trabalho, pois ainda estamos estudamos alguns modelos de banco de dados: o modelo relacional e o modelo orientado a grafos. O primeiro se baseia em tabelas e armazena as informa��es em tabelas, o �ltimo se baseia em grafos e armazenas as informa��es neste.

Propomos fazer a an�lise de qual a melhor maneira de armazenar essas informa��es levando em conta a grade quantidade de informa��es e tamb�m a possibilidade de armazenar imagens, al�m da query time para serem feitas as pesquisas. Analisando compara��es entre o tipo de armazenamento relacional e orientado a grafos, observamos que o segundo possue menor querytime, al�m de tamb�m possuir grande capacidade de armazenamento. Na presente fase da incia��o da pesquisa analisaremos as caracter�sticas das informa��es e verificaremos qual a melhor maneira de modelar os dados, pois embora aparentemente o modelo orientado a grafos pare�a ter vantagens sobre o modelo relacional, n�o necessariamente ele melhor se adequar� a necessidade da pesquisa.

\newpage

\chapter*{Abstract}

This research aims to make a study of methods of storing information, focusing on algebraic connectivity and its eigenvectors in the class of trees. In this work we will investigate tools databases to store information generated trees in an unprecedented manner by graduate student Thiago. Until now it was not possible to complete the work, as we are still studying some models database: the relational model and model-oriented graphs. The first is based on tables and stores information in tables, the latter is based on graphs and store the information on this.

We propose to do the analysis of what the best way to store this information taking into account the amount of grid information and the possibility to store images, plus the team query to be made ??the research. Analyzing comparisons between the type of relational storage and oriented graphs, we observe that the second smallest querytime possesses, and also has large storage capacity. At this stage of the research incia��o analyze the characteristics of the information and we will check what is the best way to model the data, for though apparently driven model graphs appears to have advantages over the relational model, not necessarily better it will suit the need of research.
